% This LaTeX document needs to be compiled with XeLaTeX.
\documentclass[10pt]{article}
\usepackage[utf8]{inputenc}
\usepackage{ucharclasses}
\usepackage{graphicx}
\usepackage[export]{adjustbox}
\graphicspath{ {./images/} }
\usepackage{amsmath}
\usepackage{amsfonts}
\usepackage{amssymb}
\usepackage[version=4]{mhchem}
\usepackage{stmaryrd}
\usepackage[fallback]{xeCJK}
\usepackage{polyglossia}
\usepackage{fontspec}
\setCJKmainfont{Noto Serif CJK KR}

\setmainlanguage{english}
\setotherlanguages{thai}
\newfontfamily\thaifont{Noto Serif Thai}
\newfontfamily\lgcfont{CMU Serif}
\setDefaultTransitions{\lgcfont}{}
\setTransitionsFor{Thai}{\thaifont}{\lgcfont}

\title{2025학년도 대학수학능력시험 }

\author{}
\date{}


\begin{document}
\maketitle
\begin{center}
\includegraphics[max width=\textwidth]{2024_07_25_adbb0085b5e6606b27f6g-01(2)}
\end{center}

\begin{itemize}
  \item 수학 영역 -
\end{itemize}

\section*{공통과목}
\begin{center}
\includegraphics[max width=\textwidth]{2024_07_25_adbb0085b5e6606b27f6g-01(3)}
\end{center}

\section*{해설}
\begin{enumerate}
  \item $\left(2^{2-\sqrt{2}}\right)^{1+\sqrt{2}}=2^{(2-\sqrt{2})(1+\sqrt{2})}$
\end{enumerate}

\[
\begin{aligned}
& =2^{\sqrt{2}(\sqrt{2}-1)(\sqrt{2}+1)} \\
& =2^{\sqrt{2}}
\end{aligned}
\]

\begin{enumerate}
  \setcounter{enumi}{1}
  \item $f(x)=2 x^{3}-3 x+1$ 에서
\end{enumerate}

$f^{\prime}(x)=6 x^{2}-3$ 이므로

$\lim _{h \rightarrow 0} \frac{f(1+h)-f(1)}{h}=f^{\prime}(1)=3$

\begin{enumerate}
  \setcounter{enumi}{2}
  \item $\sin ^{2} \theta=\frac{9}{25}$ 이므로 $\cos ^{2} \theta=1-\sin ^{2} \theta=\frac{16}{25}$ 이고
\end{enumerate}

$\tan ^{2} \theta=\frac{\sin ^{2} \theta}{\cos ^{2} \theta}=\frac{9}{16}$

$\pi<\theta<\frac{3}{2} \pi$ 일 때 $\tan \theta>0$ 이므로

$\tan (\pi+\theta)=\tan \theta=\frac{3}{4}$

\begin{enumerate}
  \setcounter{enumi}{3}
  \item $\lim _{x \rightarrow 0} f(x)=2, \lim _{x \rightarrow 1+} f(x)=-1$ 이므로 $\lim _{x \rightarrow 0} f(x)-\lim _{x \rightarrow 1+} f(x)=2-(-1)=3$

  \item 수열 $\left\{a_{n}\right\}$ 이 등비수열이므로 $a_{1} a_{5}=a_{3}^{2}$ 이고 $a_{1} a_{3} a_{5}=27$ 에서 $a_{3}^{3}=27$ 곧, $a_{3}=3$

\end{enumerate}

등비수열 $\left\{a_{n}\right\}$ 의 공비를 $r$ 이라 하면

$a_{4}=a_{3} r=3 r$ 이므로

$a_{3}+a_{4}=9$ 에서

$3+3 r=9$

곧, $r=2$

따라서

$a_{5}=a_{3} \times r^{2}$

$=3 \times 2^{2}$

$=12$

\begin{enumerate}
  \setcounter{enumi}{5}
  \item $\frac{6}{\log _{b} a+1}=4$ 에서
\end{enumerate}

$\log _{b} a=\frac{1}{2}$, 곧 $\log _{a} b=2$ 이고 $a^{2}=b$

따라서 $\frac{2 a+b}{\log _{a} b}=4$ 에서 $\frac{2 a+a^{2}}{2}=4$

$(a+4)(a-2)=0$

$a>0$ 이므로 $a=2$

이때 $b=a^{2}=4$

$a \times b=8$\\
7. $f(x)=x^{3}+3 x^{2}, g(x)=2 x^{2}+x+1$ 이라 하자. 두 곡선 $y=f(x), y=g(x)$ 가 만나는 점의 $x$ 좌표는 $x^{3}+3 x^{2}=2 x^{2}+x+1$ 에서

$x^{3}+x^{2}-x-1=0$

$(x+1)^{2}(x-1)=0$

$x=-1$ 또는 $x=1$

\begin{center}
\includegraphics[max width=\textwidth]{2024_07_25_adbb0085b5e6606b27f6g-01(1)}
\end{center}

닫힌구간 $[-1,1]$ 에서 $g(x) \geq f(x)$ 이므로 두 곡선 $y=f(x), y=g(x)$ 로 둘러싸인 부분의 넓이는

$\int_{-1}^{1}\{g(x)-f(x)\} d x$

$=\int_{-1}^{1}\left(-x^{3}-x^{2}+x+1\right) d x$

$=\int_{-1}^{1}\left(-x^{2}+1\right) d x\left(\because \int_{-1}^{1}\left(-x^{3}+x\right) d x=0\right)$.

$=2 \int_{0}^{1}\left(-x^{2}+1\right) d x$

$=2\left[-\frac{1}{3} x^{3}+x\right]_{0}^{1}$

$=2 \times \frac{2}{3}$

$=\frac{4}{3}$

\begin{enumerate}
  \setcounter{enumi}{7}
  \item $(x+3) f(x)=\left(x^{2}-x\right) g(x)$
\end{enumerate}

위의 식의 양변에 $x=3$ 을 대입하면

$6 f(3)=6 g(3)$, 곧 $f(3)=g(3)$

(ㄱ)의 양변을 $x$ 에 대하여 미분하면

$f(x)+(x+3) f^{\prime}(x)=(2 x-1) g(x)+\left(x^{2}-x\right) g^{\prime}(x)$ 위의 식의 양변에 $x=3$ 을 대입하면

$f(3)+6 f^{\prime}(3)=5 g(3)+6 g^{\prime}(3)$

$f(3)=g(3)=6, g^{\prime}(3)=3$ 이므로 (ㄴ)에서

$6+6 f^{\prime}(3)=5 \times 6+6 \times 3$

따라서 $f^{\prime}(3)=7$

\begin{enumerate}
  \setcounter{enumi}{8}
  \item $(\sin x-k)(\cos x-k)=0$ 에서
\end{enumerate}

$\sin x=k$ 또는 $\cos x=k$

곧, $x$ 에 대한 방정식 $(\sin x-k)(\cos x-k)=0$ 의 양의 실근은 직선 $y=k$ 가 $x>0$ 에서 두 곡선 $y=\sin x, y=\cos x$ 와 만나는 점의 $x$ 좌표와 같다.

$0<x \leq \frac{\pi}{4}$ 에서 $0<\sin x \leq \frac{\sqrt{2}}{2} \leq \cos x$ 이고

$\sin \alpha_{1}=k$ 이므로

$0<k \leq \frac{\sqrt{2}}{2}$

그런데 $k=\frac{\sqrt{2}}{2}$ 이면 $\alpha_{1}=\frac{\pi}{4}, \alpha_{2}=\frac{3}{4} \pi$ 가 되어

$\alpha_{2}-\alpha_{1}=\frac{\pi}{10}$ 를 만족시키지 않는다.

따라서 $0<k<\frac{\sqrt{2}}{2}$

\begin{center}
\includegraphics[max width=\textwidth]{2024_07_25_adbb0085b5e6606b27f6g-01}
\end{center}

위의 그림과 같이 직선 $y=k$ 가 두 곡선 $y=\sin x$ $y=\cos x$ 와 만나는 점은 구간 $[0,2 \pi]$ 에 4 개가 있다. 두 함수 $y=\sin x, y=\cos x$ 는 주기가 모두 $2 \pi$ 이므로 구간 $[2 \pi, 4 \pi],[4 \pi, 6 \pi], \cdots$ 에 이와 같은 교점 4 개가 반복적으로 나타난다.

따라서

\[
\begin{aligned}
\alpha_{8}-\alpha_{7} & =\left(\alpha_{4}+2 \pi\right)-\left(\alpha_{3}+2 \pi\right) \\
& =\alpha_{4}-\alpha_{3}
\end{aligned}
\]

그런데 곡선 $y=\cos x$ 는 직선 $x=\pi$ 에 대하여 대칭 이므로

$\alpha_{4}=2 \pi-\alpha_{2}$

또한 곡선 $y=\sin x$ 는 직선 $x=\frac{\pi}{2}$ 에 대하여 대칭

이므로

$\alpha_{3}=\pi-\alpha_{1}$

따라서

\[
\begin{aligned}
\alpha_{4}-\alpha_{3} & =\left(2 \pi-\alpha_{2}\right)-\left(\pi-\alpha_{1}\right) \\
& =\pi-\left(\alpha_{2}-\alpha_{1}\right) \\
& =\pi-\frac{\pi}{10} \\
& =\frac{9}{10} \pi
\end{aligned}
\]

\begin{enumerate}
  \setcounter{enumi}{9}
  \item 점 P 의 시각 $t$ 에서의 속도가 $v_{1}(t)=a t^{2}+b t$ 이고 조건 (가)에서 점 P 는 시각 $t=2$ 에서 운동 방향이 바뀌므로
\end{enumerate}

$v_{1}(2)=4 a+2 b=0$ 에서 $b=-2 a$

곧, $v_{1}(t)=a t(t-2)$

점 Q 의 시각 $t$ 에서의 속도는

$v_{2}(t)=t^{2}-5 t+4=(t-1)(t-4)$ 이므로

점 Q 는 시각 $t=1$ 에서 처음으로 운동 방향이 바퓌고 $t=4$ 에서 두 번째로 운동 방향이 바꿘다.

점 P 가 시각 $t=1$ 에서 $t=4$ 까지 움직인 거리는

$\int_{1}^{4}\left|v_{1}(t)\right| d t=\int_{1}^{4} a|t(t-2)| d t(\because a>0)$

\[
\begin{aligned}
& =-\int_{1}^{2} a t(t-2) d t+\int_{2}^{4} a t(t-2) d t \\
& =-a\left[\frac{1}{3} t^{3}-t^{2}\right]_{1}^{2}+a\left[\frac{1}{3} t^{3}-t^{2}\right]_{2}^{4} \\
& =\frac{22}{3} a
\end{aligned}
\]

점 Q 가 시각 $t=1$ 에서 $t=4$ 까지 움직인 거리는

\[
\begin{aligned}
\int_{1}^{4}\left|v_{2}(t)\right| d t & =\int_{1}^{4}|(t-1)(t-4)| d t \\
& =-\int_{1}^{4}(t-1)(t-4) d t \\
& =-\left[\frac{1}{3} t^{3}-\frac{5}{2} t^{2}+4 t\right]_{1}^{4} \\
& =\frac{9}{2}
\end{aligned}
\]

조건 (나)에서 시각 $t=1$ 에서 $t=4$ 까지 두 점 P , Q 가 각각 움직인 거리는 같으므로

$\int_{1}^{4}\left|v_{1}(t)\right| d t=\int_{1}^{4}\left|v_{2}(t)\right| d t$ 에서

$\frac{22}{3} a=\frac{9}{2}$, 곧 $a=\frac{27}{44}$

\begin{enumerate}
  \setcounter{enumi}{10}
  \item $S_{2 m}=a_{2 m}+3 a_{m}$
\end{enumerate}

$S_{2 m}-a_{2 m}=3 a_{m}$

곧, $S_{2 m-1}=3 a_{m}$

좌변의 $S_{2 m-1}$ 은

$S_{2 m-1}=(2 m-1) \times \frac{a_{1}+a_{2 m-1}}{2}=(2 m-1) \times a_{m}$

이므로 (ㄱ)에서

$(2 m-1) \times a_{m}=3 a_{m}$

$(2 m-4) \times a_{m}=0$

곧, $m=2$ 또는 $a_{m}=0$ 일 때 (ㄱ)을 만족시킨다.

그런데 주어진 조건에서 (ㄱ)을 만족시키는 모든

자연수 $m$ 의 값의 합이 8 이므로 $a_{m}=0$ 이 되도록 하는 자연수 $m$ 의 값은 6 이다. $\left(\because a_{m}=0\right.$ 이 되도록 하는 자연수 $m$ 의 개수는 1 이하이다.)

곧, $a_{6}=0$

등차수열 $\left\{a_{n}\right\}$ 의 공차를 $d$ 라 하면

$a_{1}=-10$ 이므로

$d=\frac{a_{6}-a_{1}}{5}=\frac{0-(-10)}{5}=2$

따라서

$a_{10}=a_{1}+9 d$

$=-10+9 \times 2$

$=8$

\begin{enumerate}
  \setcounter{enumi}{11}
  \item 두 이차함수 $f(x), g(x)$ 의 최고차항의 계수는 각각 $1,-1$ 이고, 조건 (가)에 의하여 두 곡선 $y=f(x), y=g(x)$ 는 모두 직선 $y=t x$ 와 원점에서 접하므로
\end{enumerate}

$f(x)-t x=x^{2}, g(x)-t x=-x^{2}$

따라서

$f(x) g(x)=\left(x^{2}+t x\right)\left(-\dot{x}^{2}+t x\right)$

\[
=-x^{2}(x+t)(x-t)
\]

곡선 $y=f(x) g(x)$ 와 직선 $y=x+t$ 가 모두 점 $(-t, 0)$ 을 지난다.

곡선 $y=f(x) g(x)$ 와 직선 $y=x+t$ 의 접점이 $(-t, 0)$ 인 경우와 접점이 $(-t, 0)$ 이 아닌 경우로 나누어 살펴보자.

( i ) 접점이 $(-t, 0)$ 인 경우

이 경우 곡선 $y=f(x) g(x)$ 위의 점 $(-t, 0)$ 에서의 접선의 기울기는 직선 $y=x+t$ 의 기울기인 1 과 같다.

$\{f(x) g(x)\}^{\prime}=-4 x^{3}+2 t^{2} x$ 이므로

$-4 \times(-t)^{3}+2 t^{2} \times(-t)=1$ 에서

$2 t^{3}=1$, 곧 $t=\frac{1}{\sqrt[3]{2}}$

\begin{center}
\includegraphics[max width=\textwidth]{2024_07_25_adbb0085b5e6606b27f6g-02(1)}
\end{center}

(ii) 접점이 $(-t, 0)$ 이 아닌 경우 접점의 $x$ 좌표를 $\alpha(\alpha \neq-t)$ 라 하면 곡선 $y=f(x) g(x)$ 위의 점 $\left(\alpha,-\alpha^{2}(\alpha+t)(\alpha-t)\right)$ 에서의 접선의 방정식은

$y=\left(-4 \alpha^{3}+2 t^{2} \alpha\right)(x-\alpha)-\alpha^{2}(\alpha+t)(\alpha-t)$ 이고 이 직선이 점 $(-t, 0)$ 을 지나므로

$0=\left(-4 \alpha^{3}+2 t^{2} \alpha\right)(-t-\alpha)-\alpha^{2}(\alpha+t)(\alpha-t)$ $\alpha(\alpha+t)^{2}(3 \alpha-2 t)=0$

$\alpha \neq-t$ 이므로 $\alpha=0$ 또는 $\alpha=\frac{2 t}{3}$\\
이때 접선의 기울기는 직선 $y=x+t$ 의 기울기인 1 과 같으므로 $-4 \alpha^{3}+2 t^{2} \alpha=1$

그런데 $\alpha=0$ 은 $-4 \alpha^{3}+2 t^{2} \alpha=1$ 을 만족시키지

않으므로 $\alpha=\frac{2 t}{3}$

$-4 \alpha^{3}+2 t^{2} \alpha=1$ 에서

$-4 \times\left(\frac{2 t}{3}\right)^{3}+2 t^{2} \times \frac{2 t}{3}=1$

$\frac{4 t^{3}}{27}=1$, 곧 $t=\frac{3}{\sqrt[3]{4}}$

\begin{center}
\includegraphics[max width=\textwidth]{2024_07_25_adbb0085b5e6606b27f6g-02}
\end{center}

( i ), ( ii)에 의하여 조건을 만족시키는 모든 $t$ 의 값의 곱은

$\frac{1}{\sqrt[3]{2}} \times \frac{3}{\sqrt[3]{4}}=\frac{3}{\sqrt[3]{8}} \doteq \frac{3}{2}$.

\section*{[다른 풀이]}
두 이차함수 $f(x), g(x)$ 의 최고차항의 계수는 각각 $1,-1$ 이고, 조건 (가)에 의하여 두 곡선 $y=f(x)$, $y=g(x)$ 는 모두 직선 $y=t x$ 와 원점에서 접하므로

$f(x)-t x=x^{2}, g(x)-t x=-x^{2}$

따라서

$f(x) g(x)=\left(x^{2}+t x\right)\left(-x^{2}+t x\right)$

\[
=-x^{2}(x+t)(x-t)
\]

곡선 $y=f(x) g(x)$ 가 직선 $y=x+t$ 와 접하므로

방정식 $f(x) g(x)=x+t$ 에서

$-x^{2}(x+t)(x-t)=x+t$

곧, $(x+t)\left(x^{3}-t x^{2}+1\right)=0$ 이 중근을 갖는다.

따라서 방정식 $x^{3}-t x^{2}+1=0$ 이 $-t$ 를 실근으로

갖거나 $x^{3}-t x^{2}+1=0$ 이 중근을 갖는다.

( i ) 방정식 $x^{3}-t x^{2}+1=0$ 이 $-t$ 를 실근으로 갖는 경우

\[
\begin{aligned}
& (-t)^{3}-t \times(-t)^{2}+1=0 \\
& 2 t^{3}=1, \text { 곧 } t=\frac{1}{\sqrt[3]{2}}
\end{aligned}
\]

(ii) 방정식 $x^{3}-t x^{2}+1=0$ 이 중근을 갖는 경우

\[
h(x)=x^{3}-t x^{2}+1 \text { 이라 하면 }
\]

\[
h^{\prime}(x)=3 x^{2}-2 t x=x(3 x-2 t) \text { 이므로 }
\]

삼차함수 $h(x)$ 는 두 극값

$h(0)=1, h\left(\frac{2 t}{3}\right)=-\frac{4 t^{3}}{27}+1$ 을 갖는다.

삼차방정식 $h(x)=0$ 이 중근을 가지려면 두 극값 중 하나가 0 이어야 하므로

$-\frac{4 t^{3}}{27}+1=0$ 에서 $t=\frac{3}{\sqrt[3]{4}}$

(i), (ii)에 의하여 조건을 만족시키는 모든 $t$ 의 값의 곱은

$\frac{1}{\sqrt[3]{2}} \times \frac{3}{\sqrt[3]{4}}=\frac{3}{\sqrt[3]{8}}=\frac{3}{2}$

\begin{enumerate}
  \setcounter{enumi}{12}
  \item $a_{n+1}= \begin{cases}3-a_{n} & \left(0 \leq a_{n} \leq 3\right) \\ \frac{a_{n}-3}{2} & \left(a_{n}>3\right)\end{cases}$
\end{enumerate}

만약 어떤 자연수 $m$ 에 대하여 $0 \leq a_{m} \leq 3$ 이면\\
(7ᄀ)에 의하여 $a_{m+1}=3-a_{m}$ 이 되어 $0 \leq a_{m+1} \leq 3$ 이고, 만약 어떤 자연수 $m$ 에 대하여 $a_{m}>3$ 이면

(ㄱ)에 의하여 $a_{m+1}=\frac{a_{m}-3}{2}$ 이 되어 $a_{m+1}>0$ 이다.

곧, $a_{m} \geq 0$ 이면 $a_{m+1} \geq 0$

그런데 $a_{1} \geq 0$ 이므로 모든 자연수 $n$ 에 대하여

$a_{n} \geq 0$

$a_{p}=a_{p+2}$ 가 성립할 조건을 $a_{p}$ 의 값의 범위에 따라 경우를 나누어 살펴보자.

( i ) $0 \leq a_{p} \leq 3$ 인 경우

(ㄱ)에 의하여 $a_{p+1}=3-a_{p}$

$0 \leq a_{p+1} \leq 3$ 이므로 (ㄱ)에 의하여

$a_{p+2}=3-a_{p+1}=a_{p}$

곧, $0 \leq a_{p} \leq 3$ 이면 $a_{p}=a_{p+2}$

(ii) $a_{p}>3$ 인 경우

(ㄱ)에 의하여 $a_{p+1}=\frac{a_{p}-3}{2}$

(ㄱ)에 의하여

$a_{p+2}=\frac{a_{p+1}-3}{2}$ 또는 $a_{p+2}=3-a_{p+1}$

곧, $a_{p+2}=\frac{a_{p}-9}{4}$ 또는 $a_{p+2}=\frac{9-a_{p}}{2}$

만약 $a_{p}=a_{p+2}$ 이면

$a_{p}=\frac{a_{p}-9}{4}$ 또는 $a_{p}=\frac{9-a_{p}}{2}$ 에서

$a_{p}=-3$ 또는 $a_{p}=3$ 이 되어

$a_{p}>3$ 을 만족시키지 않는다.

곧, $a_{p}>3$ 이면 $a_{p} \neq a_{p+2}$

( i), (ii)에 의하여

$a_{p}=a_{p+2}$ 이기 위한 필요충분조건은 $0 \leq a_{p} \leq 3$ 이다.

….. (ㄴ)

만약 $0 \leq a_{1} \leq 3$ 이면

(ㄱ)에 의하여 $a_{2}=3-a_{1}$ 이고 $0 \leq a_{2} \leq 3$ 이므로

(ㄴ)에 의하여 모든 자연수 $n$ 에 대하여 $a_{n}=a_{n+2}$ 가

되어 ' $a_{p} \neq a_{p+2}$ 인 자연수 $p$ 의 개수가 2 '라는 조건 을 만족시키지. 않는다.

곧, $a_{1}>3$

만약 $0 \leq a_{2} \leq 3$ 이면 마찬가지의 방법으로 2 이상의 모든 자연수 $n$ 에 대하여 $a_{n}=a_{n+2}$ 가 되어

' $a_{p} \neq a_{p+2}$ 인 자연수 $p$ 의 개수가 2 '라는 조건을

만족시키지 않는다.

곧, $a_{2}>3$

만약 $0 \leq a_{3} \leq 3$ 이면 마찬가지의 방법으로 3 이상의 모든 자연수 $n$ 에 대하여

$a_{n}=a_{n+2}$

따라서 $a_{1}>3, a_{2}>3,0 \leq a_{3} \leq 3$ 일 때

' $a_{p} \neq a_{p+2}$ 인 자연수 $p$ 의 개수가 2 '라는 조건을 만족시킨다.

이때 $a_{3}=a$ 라 하면

$0 \leq a_{3} \leq 3$ 이므로 (ㄱ)에 의하여 $a_{4}=3-a$ 이고,

$0 \leq a_{4} \leq 3$ 이므로 (ㄴ)에 의하여 모든 자연수 $k$ 에 대하여

$a_{2 k+1}=\dot{a}_{3}=a$,

$a_{2 k+2}=a_{4}=3-a$

또한 $a_{2}>3$ 이므로 $a_{3}=\frac{a_{2}-3}{2}$

곧, $a_{2}=2 a_{3}+3=2 a+3$

$a_{1}>3$ 이므로 $a_{2}=\frac{a_{1}-3}{2}$

곧, $a_{1}=2 a_{2}+3=4 a+9$

따라서

\section*{수학 영역}
\[
\begin{aligned}
\sum_{k=1}^{12} a_{k} & =a_{1}+a_{2}+\sum_{k=1}^{5}\left(a_{2 k+1}+a_{2 k+2}\right) \\
& =(4 a+9)+(2 a+3)+\sum_{k=1}^{5}\{a+(3-a)\} \\
& =6 a+12+\sum_{k=1}^{5} 3 \\
& =6 a+27
\end{aligned}
\]

주어진 조건에서 $\sum_{k=1}^{12} a_{k}=33$ 이므로

$6 a+27=33$ 에서 $a=1$

따라서 $a_{1}=4 a+9=13, a_{12}=3-a=2$ 이므로 $a_{1}+a_{12}=15$

\begin{enumerate}
  \setcounter{enumi}{13}
  \item 집합 $A=\{x|f(x)=| x \mid\}$ 에서
\end{enumerate}

방정식 $f(x)=|x|$ 는 $f(x)=x(x \geq 0)$ 또는 $f(x)=-x(x \leq 0)$ 과 같다.

곧, 집합 $A$ 의 원소는 곡선 $y=f(x)$ 가 두 반직선 $y=x(x \geq 0), y=-x(x \leq 0)$ 과 만나는 점의 $x$ 좌표 이다.

\begin{center}
\includegraphics[max width=\textwidth]{2024_07_25_adbb0085b5e6606b27f6g-03(5)}
\end{center}

또한 집합 $B=\{x|| f(x) \mid=x\}$ 에서

방정식 $|f(x)|=x$ 는

$f(x)=x(f(x) \geq 0)$ 또는 $-f(x)=x(f(x) \leq 0)$, 곧 $f(x)=x(x \geq 0)$ 또는 $f(x)=-x(x \geq 0)$ 과 같다.

곧, 집합 $B$ 의 원소는 곡선 $y=f(x)$ 가 두 반직선 $y=x(x \geq 0), y=-x(x \geq 0)$ 과 만나는 점의 $x$ 좌표 이다.

\begin{center}
\includegraphics[max width=\textwidth]{2024_07_25_adbb0085b5e6606b27f6g-03(1)}
\end{center}

집합 $A \cap B$ 의 원소는 곡선 $y=f(x)$ 가 반직선 $y=x(x \geq 0)$ 과 만나는 점의 $x$ 좌표이므로 조건 (가)의 집합 $(A \cup B)-(A \cap B)$ 의 원소는 곡선 $y=f(x)$ 가 두 반직선 $y=-x(x<0)$, $y=-x(x>0)$ 과 만나는 점의 $x$ 좌표이다. 조건 (가)에서 이 집합의 원소는 $-1,1$ 뿐이고 $f(0) \neq 0$ 이므로 곡선 $y=f(x)$ 와 직선 $y=-x$ 의 교점의 개수는 2 이고, 이 교점의 $x$ 좌표는 -1 또는 1 이다.

함수 $f(x)$ 는 최고차항의 계수가 1 인 삼차함수이므로

$f(x)=(x+1)^{2}(x-1)-x$ 또는

$f(x)=(x+1)(x-1)^{2}-x$

다음 그림과 같이 $f(x)=(x+1)^{2}(x-1)-x$ 일 때 곡선 $y=f(x)$ 와 반직선 $y=x(x \geq 0)$ 의 교점의 개수가 1 이고 $n(A)+n(B)=2+2=4$ 가 되어 조건 (나)를 만족시키지 않는다.

\begin{center}
\includegraphics[max width=\textwidth]{2024_07_25_adbb0085b5e6606b27f6g-03(4)}
\end{center}

다음 그림과 같이 $f(x)=(x+1)(x-1)^{2}-x$ 일 때 곡선 $y=f(x)$ 와 반직선 $y=x(x \geq 0)$ 의 교점의 개수가 2 이고 $n(A)+n(B)=3+3=6$ 이 되어 조건 (나)를 만족시킨다.

\begin{center}
\includegraphics[max width=\textwidth]{2024_07_25_adbb0085b5e6606b27f6g-03}
\end{center}

따라서 모든 조건을 만족시키는 함수 $f(x)$ 는

$f(x)=(x+1)(x-1)^{2}-x$ 이므로

$f(3)=13$

\begin{enumerate}
  \setcounter{enumi}{14}
  \item $\overline{\mathrm{AE}}: \overline{\mathrm{BE}}=5: 3$ 이므로
\end{enumerate}

$\overline{\mathrm{AE}}=5 a, \overline{\mathrm{BE}}=3 a(a>0)$ 으로 놓을 수 있다. 두 삼각형 $\mathrm{ABC}, \mathrm{AEC}$ 에서 사인법칙에 의하여 $\overline{\mathrm{AB}}=2 R_{1} \sin (\angle \mathrm{ACB}), \overline{\mathrm{AE}}=2 R_{2} \sin (\angle \mathrm{ACE})$ 이고 $\angle \mathrm{ACB}=\angle \mathrm{ACE}$ 이므로

$\overline{\mathrm{AB}}: \overline{\mathrm{AE}}=2 R_{1} \sin (\angle \mathrm{ACB}): 2 R_{2} \sin (\angle \mathrm{ACE})$

\[
\begin{aligned}
& =R_{1}: R_{2} \\
& =7: 5
\end{aligned}
\]

곧, $\overline{\mathrm{AB}}=7 a$

\begin{center}
\includegraphics[max width=\textwidth]{2024_07_25_adbb0085b5e6606b27f6g-03(3)}
\end{center}

삼각형 ABE 에서 코사인법칙에 의하여

$\cos (\angle \mathrm{AEB})=\frac{(3 a)^{2}+(5 a)^{2}-(7 a)^{2}}{2 \times 3 a \times 5 a}=-\frac{1}{2}$

곧, $\angle \mathrm{AEB}=\frac{2 \pi}{3}$

$\angle \mathrm{EAD}=\angle \mathrm{ECD}(\because$ 호 DE 의 원주각)이므로 두 삼각형 ABE 와 CBD 는 닮은 도형이다.

따라서 $\angle \mathrm{CDB}=\angle \mathrm{AEB}=\frac{2 \pi}{3}$ 이므로

$\angle \mathrm{ADC}=\frac{\pi}{3}$ 이고

$\overline{\mathrm{CB}}: \overline{\mathrm{BD}}=\overline{\mathrm{AB}}: \overline{\mathrm{BE}}=7: 3$ 이므로

$\overline{\mathrm{CB}}=7 b, \overline{\mathrm{BD}}=3 b(b>0)$ 으로 놓을 수 있다.

$\overline{\mathrm{BE}}: \overline{\mathrm{BD}}=11: 7$ 에서 $3 a: 3 b=11: 7$ 이므로

$a=\frac{11}{7} b$

$\overline{\mathrm{AD}}=\overline{\mathrm{AB}}-\overline{\mathrm{BD}}$

$=7 a-3 b$

$=8 b$\\
이고 $\overline{\mathrm{DC}}=5 b, \overline{\mathrm{AC}}=7, \angle \mathrm{ADC}=\frac{\pi}{3}$ 이므로

삼각형 ADC 에서 코사인법칙에 의하여

$7^{2}=(8 b)^{2}+(5 b)^{2}-2 \times 8 b \times 5 b \times \cos \frac{\pi}{3}$

$49=64 b^{2}+25 b^{2}-40 b^{2}$

$b=1(\because b>0)$

곧, $\overline{\mathrm{AD}}=8, \overline{\mathrm{DC}}=5$

따라서 삼각형 ADC 의 넓이는

$\frac{1}{2} \times 8 \times 5 \times \sin \frac{\pi}{3}=10 \sqrt{3}$

\begin{enumerate}
  \setcounter{enumi}{15}
  \item $\log _{3} 2 x$ 와 $\log _{3}(x-3)$ 에서 진수의 조건에 의하여 $2 x>0$ 이고 $x-3>0$ 이므로
\end{enumerate}

$x>3$

$\log _{3} 2 x \geq \log _{3}(x-3)+1$ 에서

$\log _{3} 2 x \geq \log _{3}(x-3)+\log _{3} 3$

$\log _{3} 2 x \geq \log _{3} 3(x-3)$

$2 x \geq 3 x-9$

$x \leq 9$

(ㄱ), (ㄴ)에서 $3<x \leq 9$ 이므로 부등식을 만족시키는 정수 $x$ 는 $4,5,6,7,8,9$ 이고 그 합은 39 이다.

\begin{enumerate}
  \setcounter{enumi}{16}
  \item $f^{\prime}(x)=4 x^{3}+4 x-1$ 에서
\end{enumerate}

$f(x)=x^{4}+2 x^{2}-x+C(C$ 는 적분상수 $)$

$f(0)=2$ 에서 $C=2$

따라서 $f(x)=x^{4}+2 x^{2}-x+2$ 이므로

$f(2)=24$

\begin{enumerate}
  \setcounter{enumi}{17}
  \item $\sum_{k=1}^{10}\left(a_{2 k-1}+a_{2 k}\right)^{2}=20$ 에서
\end{enumerate}

$\sum_{k=1}^{10}\left\{\left(a_{2 k-1}\right)^{2}+2 a_{2 k-1} a_{2 k}+\left(a_{2 k}\right)^{2}\right\}=20$\\
$\sum_{k=1}^{10}\left(a_{2 k-1}-a_{2 k}\right)^{2}=10$ 에서\\
$\sum_{k=1}^{10}\left\{\left(a_{2 k-1}\right)^{2}-2 a_{2 k-1} a_{2 k}+\left(a_{2 k}\right)^{2}\right\}=10$

(ㄱ), (ㄴ)을 변끼리 더하여 정리하면

$2 \sum_{k=1}^{10}\left\{\left(a_{2 k-1}\right)^{2}+\left(a_{2 k}\right)^{2}\right\}=20+10$

$\sum_{k=1}^{10}\left\{\left(a_{2 k-1}\right)^{2}+\left(a_{2 k}\right)^{2}\right\}=15$

따라서 $\sum_{k=1}^{20}\left(a_{k}\right)^{2}=\sum_{k=1}^{10}\left\{\left(a_{2 k-1}\right)^{2}+\left(a_{2 k}\right)^{2}\right\}=15$

\begin{enumerate}
  \setcounter{enumi}{18}
  \item 함수 $f(x)$ 를 $f(x)=3 x^{4}-16 x^{3}+18 x^{2}$ 이라 하자. 방정식 $3 x^{4}-16 x^{3}+18 x^{2}+k=0$ 의 실근은 곡선 $y=f(x)$ 와 직선 $y=-k$ 의 교점의 $x$ 좌표이다. $f^{\prime}(x)=12 x^{3}-48 x^{2}+36 x$
\end{enumerate}

\[
=12 x(x-1)(x-3)
\]

이고, $f(0)=0, f(1)=5, f(3)=-27$ 이므로

함수 $f(x)$ 의 그래프는 다음과 같다.

\begin{center}
\includegraphics[max width=\textwidth]{2024_07_25_adbb0085b5e6606b27f6g-03(2)}
\end{center}

\section*{수학 영역}
따라서 방정식 $3 x^{4}-16 x^{3}+18 x^{2}+k=0$ 이 서로 다른 두 개의 양의 실근을 가지려면

$-k=5$ 또는 $-27<-k \leq 0$ 이다.

곧, $k=-5$ 또는 $0 \leq k<27$

따라서 정수 $k$ 의 개수는 28 이다.

\begin{enumerate}
  \setcounter{enumi}{19}
  \item $G(x)=\int_{0}^{x} g(t) d t$ 라 하면
\end{enumerate}

$G(0)=0$ 이고 $G^{\prime}(x)=g(x)$

$|x-a| f(x)=G(x)$

(ㄱ)에 $x=0$ 을 대입하면 $|a| f(0)=0$ 에서

$a>0$ 이므로 $f(0)=0$ 이고,

(ㄱ)에 $x=a$ 를 대입하면 $G(a)=0$

또한 $x \neq a$ 일 때 $f(x)=\frac{G(x)}{|x-a|}$ 이고

함수 $f(x)$ 는 $x=a$ 에서 연속이므로

$\lim _{x \rightarrow a-} f(x)=\lim _{x \rightarrow a+} f(x)$

$\lim _{x \rightarrow a-} \frac{G(x)}{|x-a|}=\lim _{x \rightarrow a+} \frac{G(x)}{|x-a|}$

$\lim _{x \rightarrow a-}\left\{-\frac{G(x)-G(a)}{x-a}\right\}=\lim _{x \rightarrow a+} \frac{G(x)-G(a)}{x-a}$

$(\because G(a)=0)$

곧, $-g(a)=g(a)\left(\because G^{\prime}(a)=g(a)\right)$

따라서 $g(a)=0$ 이고 $\lim _{x \rightarrow a} f(x)=0$ 이므로

$f(a)=\lim _{x \rightarrow a} f(a)$ 에서 $f(a)=0$ 이고, $f(0)=0$ 이므로

$f(x)=x(x-a)$

따라서 $G(x)=x|x-a|(x-a)$ 이므로

$G(x)=\left\{\begin{array}{ll}-x(x-a)^{2} & (x \leq a) \\ x(x-a)^{2} & (x>a)\end{array}\right.$ 이고

$x$ 에 대하여 미분하면

$g(x)= \begin{cases}-3\left(x-\frac{a}{3}\right)(x-a) & (x \leq a) \\ 3\left(x-\frac{a}{3}\right)(x-a) & (x>a)\end{cases}$

곧, 함수 $g(x)$ 의 그래프의 개형은 그림과 같다.

\begin{center}
\includegraphics[max width=\textwidth]{2024_07_25_adbb0085b5e6606b27f6g-04(1)}
\end{center}

방정식 $g(x)=a$ 의 서로 다른 실근의 개수가 2 이고 $a>0$ 이므로 함수 $y=-3\left(x-\frac{a}{3}\right)(x-a)$ 의 최댓값이 $a$ 이다.

함수 $y=-3\left(x-\frac{a}{3}\right)(x-a)$ 의 최댓값은

$x=\frac{2 a}{3}$ 일 때 $-3\left(\frac{2 a}{3}-\frac{a}{3}\right)\left(\frac{2 a}{3}-a\right)=\frac{a^{2}}{3}$ 이므로

$\frac{a^{2}}{3}=a$ 에서 $a=3(\because a>0)$

따라서 $g(x)= \begin{cases}-3(x-1)(x-3) & (x \leq 3) \\ 3(x-1)(x-3) & (x>3)\end{cases}$

$g(2 a)=g(6)=3 \times 5 \times 3=45$

함수 $g(x)$ 가 실수 전체의 집합에서 연속이므로 함수 $\int_{0}^{x} g(t) d t$ 는 실수 전체의 집합에서 미분가능 하다.

곧, 함수 $|x-a| f(x)$ 가 $x=a$ 에서 미분가능하므로 이차함수 $f(x)$ 는 $x-a$ 를 인수로 갖는다.\\
21. 함수 $f(x)=\left\{\begin{array}{ll}2^{a-x}+a & (x<a) \\ 2^{1-x}+1 & (x \geq a)\end{array}\right.$ 의 증가, 감소에 대하여 살펴보자.

두 함수 $y=2^{a-x}+a, y=2^{1-x}+1$ 은 각각 실수 전체의 집합에서 감소한다.

또한 이 두 함수의 그래프는 각각 두 점 $(a, 1+a)$, $\left(a, 2^{1-a}+1\right)$ 을 지나는데

$a=1$ 일 때 $1+a=2^{1-a}+1$,

$a>1$ 일 때 $1+a>2^{1-a}+1$,

$a<1$ 일 때 $1+a<2^{1-a}+1$

이다.

함수 $f(x)$ 의 그래프 위의 점 $\mathrm{P}(k, f(k))$ 를 지나고 기울기가 1 인 직선을 $l_{k}$ 라 하자.

$a \geq 1$ 일 때 함수 $f(x)$ 는 실수 전체의 집합에서 감 소하므로 정수 $k$ 에 대하여 함수 $f(x)$ 의 그래프와 직선 $l_{k}$ 의 교점의 개수는 1 이다.

따라서 직선 $l_{k}$ 가 함수 $f(x)$ 의 그래프와 서로 다른 두 점에서 만나도록 하는 정수 $k$ 의 개수가 14 이려면 $a<1$ 이어야 한다.

( i ) $k<a$ 인 경우

\begin{center}
\includegraphics[max width=\textwidth]{2024_07_25_adbb0085b5e6606b27f6g-04}
\end{center}

이때 $f(k)=2^{a-k}+a$ 이므로

위의 그림과 같이 직선 $l_{k}$ 는 두 점 $\left(k, 2^{a-k}+a\right)$, $\left(a, 2^{a-k}+a+(a-k)\right)$ 를 지난다.

직선 $l_{k}$ 가 함수 $f(x)$ 의 그래프와 서로 다른 두

점에서 만나려면 점 $\left(a, 2^{a-k}+a+(a-k)\right)$ 가 점

$\left(a, 2^{1-a}+1\right)$ 보다 아래에 있거나 같아야 한다.

곧, $2^{a-k}+a+(a-k) \leq 2^{1-a}+1$

$2^{a-k}+(a-k) \leq 2^{1-a}+(1-a)$

이때 함수 $y=2^{x}+x$ 는 증가함수이므로

$a-k \leq 1-a$

따라서 $2 a-1 \leq k<a$

(ii) $k \geq a$ 인 경우

\begin{center}
\includegraphics[max width=\textwidth]{2024_07_25_adbb0085b5e6606b27f6g-04(3)}
\end{center}

이때 $f(k)=2^{1-k}+1$ 이므로

위의 그림과 같이 직선 $l_{k}$ 는 두 점 $\left(k, 2^{1-k}+1\right)$, $\left(a, 2^{1-k}+1-(k-a)\right)$ 를 지난다.

직선 $l_{k}$ 가 함수 $f(x)$ 의 그래프와 서로 다른 두

점에서 만나려면 점. $\left(a, 2^{1-k}+1-(k-a)\right)$ 가 점 $(a, 1+a)$ 보다 위에 있어야 한다.

곧, $2^{1-k}+1-(k-a)>a+1$

$2^{1-k}+(1-k)>1$

$2^{1-k}+(1-k)>2^{0}+0$

이 때 함수 $y=2^{x}+x$ 는 증가함수이므로

$1-k>0$

따라서 $a \leq k<1$

( i ), (ii)에 의하여 $2 a-1 \leq k<1$ 이므로

정수 $k$ 의 개수는 $1-(2 a-1)=2-2 a$ 이다.

따라서 $2-2 a=14$ 에서 $a=-6$ 이고 $f(x)= \begin{cases}2^{-6-x}-6 & (x<-6) \\ 2^{1-x}+1 & (x \geq-6)\end{cases}$

$f(a)=f(-6)=2^{1-(-6)}+1=129$

따라서 $a+f(a)=-6+129=123$

\section*{[다른 풀이]}
곡선 $y=2^{1-x}+1$ 을 $x$ 축의 방향으로 $a-1$ 만큼, $y$ 축의 방향으로 $a-1$ 만큼 평행이동하면 곡선

$y=2^{a-x}+a$

...... (ㄱ)

와 일치한다.

곡선 $y=2^{a-x}+a$ 는 점 $(a, a+1)$ 을 지나고

곡선 $y=2^{1-x}+1$ 은 점 $(1,2)$ 를 지나므로

두 곡선은 직선 $y=x+1$ 과 각각 이 두 점에서 만난다.

$a$ 의 값에 따라 경우를 나누어 함수 $f(x)$ 의 그래프를 살펴보자.

( i ) $a=1$ 인 경우

$f(x)=\left\{\begin{array}{ll}2^{a-x}+a & (x<a) \\ 2^{1-x}+1 & (x \geq a)\end{array}\right.$ 에서

$f(x)=2^{1-x}+1$ 이므로 함수 $f(x)$ 의 그래프와

기울기가 1 인 직선의 교점의 개수는 항상 1 이 되어 조건을 만족시키지 않는다.

(ii) $a>1$ 인 경우

$a-1>0$ 이므로 (ㄱ)에 의하여 함수 $f(x)$ 의 그래 프의 개형은 그림과 같다.

\begin{center}
\includegraphics[max width=\textwidth]{2024_07_25_adbb0085b5e6606b27f6g-04(4)}
\end{center}

이때 함수 $f(x)$ 의 그래프와 기울기가 1 인 직선 의 교점의 개수는 0 또는 1 이 되어 조건을 만족 시키지 않는다.

(iii) $a<1$ 인 경우

$a-1<0$ 이므로 (ㄱ)에 의하여 함수 $f(x)$ 의 그래프 의 개형은 그림과 같다.

\begin{center}
\includegraphics[max width=\textwidth]{2024_07_25_adbb0085b5e6606b27f6g-04(2)}
\end{center}

위의 그림과 같이 곡선 $y=2^{1-x}+1$ 위의 점 중 $x$ 좌표가 $a$ 인 점을 지나고 기울기가 1 인 직선이

곡선 $y=2^{a-x}+a$ 와 만나는 점의 $x$ 좌표는

$a+(a-1)=2 a-1$ 이다. ( $\because$ (ㄱ)

따라서 조건

'점 $\mathrm{P}(k, f(k))$ 를 지나고 기울기가 1 인 직선이 함수 $f(x)$ 의 그래프와 서로 다른 두 점에서 만난다.'

를 만족시키는 점 P 중에서

곡선 $y=2^{a-x}+a$ 위에 있는 모든 점의 $x$ 좌표 $k$ 는 $2 a-1,2 a, \cdots, a-1$ 이고,

곡선 $y=2^{1-x}+1$ 위에 있는 모든 점의 $x$ 좌표 $k$ 는 $a, a+1, \cdots, 0$ 이다.

\section*{정수 $k$ 의 개수는}
$0-(2 a-1)+1=2-2 a$

$2-2 a=14$ 에서 $a=-6$

( i ), (ii), (iii)에 의하여 $a=-6$ 이고

$f(x)=\left\{\begin{array}{ll}2^{-6-x}-6 & (x<-6) \\ 2^{1-x}+1 & (x \geq-6)\end{array}\right.$ 이므로

$f(a)=f(-6)=2^{1-(-6)}+1=129$

따라서 $a+f(a)=-6+129=123$

\begin{enumerate}
  \setcounter{enumi}{21}
  \item '두 함수 $g(x)-h(x)$ 와 $g(x)+h(x)$ 가 불연속인 점의 개수가 각각 1 이다.'
\end{enumerate}

...... (7)

함수 $f(x)$ 는 최고차항의 계수가 1 인 사차함수이고 조건 (가)에 의하여

$f(1)=0, f(2)=0, f(3)=0$ 이므로

$f(x)=(x-1)(x-2)(x-3)(x-\alpha)(\alpha$ 는 상수)로 놓을 수 있다.

이때 조건 (가)에서 모든 실수 $x$ 에 대하여

$f(x)=g(x)(x-1)(x-2)$ 이므로

$x \neq 1, x \neq 2$ 일 때 $g(x)=(x-3)(x-\alpha) \quad$...... (ㄴ)

이고 함수 $g(x)$ 는 $x=1$ 과 $x=2$ 를 제외한 모든 점 에서 연속이다.

마찬가치의 방법으로 조건 (가)에 의하여

$x \neq 1, x \neq 3$ 일 때 $h(x)=(x-2)(x-\alpha) \quad$...... (ㄷ)

이고 함수 $h(x)$ 는 $x=1$ 과 $x=3$ 을 제외한 모든 점 에서 연속이다.

(i) 함수 $g(x)$ 가 $x=2$ 에서 불연속인 경우

함수 $h(x)$ 는 $x=2$ 에서 연속이므로

두 함수 $g(x)-h(x), g(x)+h(x)$ 는 각각 $x=2$ 에서 불연속이다.

이때 (ㄱ)에 의하여

두 함수 $g(x)-h(x), g(x)+h(x)$ 는 $x=2$ 를 제외한 모든 점에서 각각 연속이다.

곧, 두 함수 $g(x), h(x)$ 는 $x=2$ 를 제외한 모든 점에서 각각 연속이다.

조건 (나)에 의하여 $g(1)=1$ 이고, (ㄴ)에서

$\lim _{x \rightarrow 1} g(x)=\lim _{x \rightarrow 1}(x-3)(x-\alpha)=2 \alpha-2$

함수 $g(x)$ 는 $x=1$ 에서 연속이므로

$g(1)=\lim _{x \rightarrow 1} g(x)$

$1=2 \alpha-2$ 에서 $\alpha=\frac{3}{2}$

이때 조건 (나)에 의하여 $h(3)=1$ 이고 (ㄷ)에서 $\lim _{x \rightarrow 3} h(x)=\lim _{x \rightarrow 3}(x-2)(x-\alpha)$

\[
\begin{aligned}
& =3-\alpha \\
& =\frac{3}{2}
\end{aligned}
\]

$h(3) \neq \lim _{x \rightarrow 3} h(x)$

곧, 함수 $h(x)$ 는 $x=3$ 에서 불연속이 되어 모순 이다.

따라서 함수 $g(x)$ 는 $x=2$ 에서 연속이다.

(ii) 함수 $h(x)$ 가 $x=3$ 에서 불연속인 경우

함수 $g(x)$ 는 $x=3$ 에서 연속이므로

두 함수 $g(x)-h(x), g(x)+h(x)$ 는 각각 $x=3$ 에서 불연속이다.

조건 (ㄱ)에 의하여

두 함수 $g(x)-h(x), g(x)+h(x)$ 는 $x=3$ 을 제외한 모든 점에서 각각 연속이다.

곧, 두 함수 $g(x), h(x)$ 는 $x=3$ 을 제외한 모든 점에서, 각각 연속이다.

(ㄴ)에서

\[
\begin{aligned}
g(2) & =\lim _{x \rightarrow 2} g(x) \\
& =\lim _{x \rightarrow 2}(x-3)(x-\alpha) \\
& =\alpha-2
\end{aligned}
\]

(ㄷ)에서

$h(1)=\lim _{x \rightarrow 1} h(x)$

\[
\begin{aligned}
& =\lim _{x \rightarrow 1}(x-2)(x-\alpha) \\
& =\alpha-1
\end{aligned}
\]

조건 (나)에서 $g(2)+h(1)=1$ 이므로

$(\alpha-2)+(\alpha-1)=1$

곧, $\alpha=2$

이때 (ㄷ)에서

$\lim _{x \rightarrow 3} h(x)=\lim _{x \rightarrow 3}(x-2)(x-\alpha)=1$

조건 (나)에서 $h(3)=1$

곧, $h(3)=\lim _{x \rightarrow 3} h(x)$ 가 되어 모순이다.

따라서 함수 $h(x)$ 는 $x=3$ 에서 연속이다.

(i), (ii)에 의하여 두 함수 $g(x), h(x)$ 는 각각 $x=2, x=3$ 에서 연속이다.

조건 (나)에서 $h(3)=1$ 이고, (ㄷ)에서

$\lim _{x \rightarrow 3} h(x)=\lim _{x \rightarrow 3}(x-2)(x-\alpha)^{*}=3-\alpha$ 이므로

$h(3)=\lim _{x \rightarrow 3} h(x)$ 에서

$1=(3-2)(3-\alpha)$

곧, $\alpha=2$

$f(x)=(x-1)(x-2)^{2}(x-3)$ 이므로

$f(4)=12$

함수 $g(x)$ 는 $x=2$ 에서 연속이므로 (ㄴ)에서

$g(2)=\lim _{x \rightarrow 2} g(x)$

\[
\begin{aligned}
& =\lim _{x \rightarrow 2}(x-3)(x-2) \\
& =0
\end{aligned}
\]

조건 (나)에서 $g(2)+h(1)=1$ 이므로

$h(1)=1$

따라서 $f(4)+g(2)-h(1)=12+0-1=11$

\section*{[다른 풀이]}
함수 $f(x)$ 는 최고차항의 계수가 1 인 사차함수이고 조건 (가)에서 $f(1)=0, f(2)=0, f(3)=0$ 이므로 $f(x)=(x-1)(x-2)(x-3)(x-\alpha)$ ( $\alpha$ 는 상수 $)$ 로 놓을 수 있다.

이때 조건 (가)에 의하여 모든 실수 $x$ 에 대하여 $(x-1)(x-2)(x-3)(x-\alpha)=g(x)(x-1)(x-2)$ 이고 조건 (나)에 의하여 $g(1)=1$ 이므로 $g(2)=m$ 이라 하면

$g(x)= \begin{cases}(x-3)(x-\alpha) & (x \neq 1,2) \\ 1 & (x=1) \\ m & (x=2)\end{cases}$

마찬가지의 방법으로 조건 (가)에 의하여 모든 실수 $x$ 에 대하여

$(x-1)(x-2)(x-3)(x-\alpha)=h(x)(x-1)(x-3)$

이고 조건 (나)에 의하여

$h(3)=1, h(1)=1-g(2)=1-m$ 이므로.

$h(x)= \begin{cases}(x-2)(x-\alpha) & (x \neq 1,3) \\ 1-m & (x=1) \\ 1 & (x=3)\end{cases}$

따라서

$g(x)= \begin{cases}(x-3)(x-\alpha) & (x \neq 1,2,3) \\ 1 & (x=1) \\ m & (x=2) \\ 0 & (x=3)\end{cases}$\\
이고

$h(x)= \begin{cases}(x-2)(x-\alpha) & (x \neq 1,2,3) \\ 1-m & (x=1) \\ 0 & (x=2) \\ 1 & (x=3)\end{cases}$

$A(x)=g(x)-h(x)$ 라 하면

$A(x)= \begin{cases}-(x-\alpha) & (x \neq 1,2,3) \\ m & (x=1) \\ m & (x=2) \\ -1 & (x=3)\end{cases}$

함수 $A(x)$ 가 $x=k(k=1,2,3)$ 에서 연속일 조건을 표로 정리하면 다음과 같다.

\begin{center}
\begin{tabular}{|c|c|c|}
\hline
$k$ & $A(k)=\lim _{x \rightarrow k} A(x)$ &  \\
\hline
1 & $m=\alpha-1$ & $\cdots \cdots \cdots$ (a) \\
\hline
2 & $m=\alpha-2$ & $\cdots \cdots .$. (b) \\
\hline
3 & $-1=\alpha-3$ & $\cdots \cdots \cdots$ (c) \\
\hline
\end{tabular}
\end{center}

함수 $A(x)$ 의 불연속점의 개수는 1 이므로 세 등식 (a), (b), (c) 중에서 두 등식은 성립하고 한 등식은 성립하지 않는다.

그런데 두 등식 (a), (b)는 동시에 성립할 수 없으 므로 등식 (c)는 성립하고 두 등식 (a), (b) 중에 하나만 성립한다.

곧, (c)에서

$\alpha=2$

이고, (a) 또는 (b)에서

$m=1$ 또는 $m=0$

마찬가지의 방법으로

$B(x)=g(x)+h(x)$ 라 하면

$B(x)= \begin{cases}(2 x-5)(x-2) & (x \neq 1,2,3) \\ 2-m & (x=1) \\ m & (x=2) \\ 1 & (x=3)\end{cases}$

함수 $B(x)$ 가 $x=k(k=1,2,3)$ 에서 연속일 조건을 표로 정리하면 다음과 같다.

\begin{center}
\begin{tabular}{|c|c|c|}
\hline
$k$ & $B(k)=\lim _{x \rightarrow k} B(x)$ &  \\
\hline
1 & $2-m=3$ & $\cdots \cdots . .(\mathrm{d})$ \\
\hline
2 & $m=0$ & $\cdots \cdots . .(\mathrm{e})$ \\
\hline
3 & $1=1$ & $\cdots \cdots . .(\mathrm{f})$ \\
\hline
\end{tabular}
\end{center}

함수 $B(x)$ 의 불연속점의 개수는 1 인데 위와 같이 함수 $B(x)$ 는 $x=3$ 에서 연속이므로 두 등식 (d), (e) 중에 하나만 성립한다.

곧, (d) 또는 (e)에서

$m=-1$ 또는 $m=0$

(ㄴ), (ㄷ)에서 $m=0$

따라서 $g(2)=m=0, h(1)=1-m=1$ 이고

(ㄱ)에서 $\alpha=2$ 이므로

$f(x)=(x-1)(x-2)^{2}(x-3)$

$f(4)=12$

$f(4)+g(2)-h(1)=12+0-1=11$

\section*{수학 영역}
\section*{확률과 통계}
\section*{해설}
\begin{enumerate}
  \setcounter{enumi}{22}
  \item $\mathrm{V}(X)=50 \times \frac{1}{5} \times\left(1-\frac{1}{5}\right)=8$

  \item 여섯 자리의 자연수가 짝수이므로 일의 자리의 수가 2 또는 4 이어야 한다.

\end{enumerate}

( i ) 일의 자리에 2 가 오는 경우

남은 다섯 자리에 $1,1,2,4,5$ 를 일렬로 나열 하는 경우의 수는

\[
\frac{5!}{2!}=60
\]

(ii) 일의 자리에 4 가 오는 경우

남은 다섯 자리에 $1,1,2,2,5$ 를 일렬로 나열 하는 경우의 수는

\[
\frac{5!}{2!\times 2!}=30
\]

( i), (ii)에 의하여 구하는 경우의 수는 $60+30=90$

\begin{enumerate}
  \setcounter{enumi}{24}
  \item $\mathrm{P}\left(A \cup B^{C}\right)=\frac{5}{6}$ 이므로
\end{enumerate}

\[
\begin{aligned}
\mathrm{P}\left(A^{C} \cap B\right) & =1-\mathrm{P}\left(A \cup B^{C}\right) \\
& =1-\frac{5}{6} \\
& =\frac{1}{6}
\end{aligned}
\]

주어진 조건에서 $\mathrm{P}(A \mid B)=\frac{1}{2}$ 이고

$\mathrm{P}(A \cap B)=\mathrm{P}(B)-\mathrm{P}\left(A^{C} \cap B\right)=\mathrm{P}(B)-\frac{1}{6}$ 이므로

$\mathrm{P}(A \mid B)=\frac{\mathrm{P}(A \cap B)}{\mathrm{P}(B)}$ 에서

$\frac{1}{2}=\frac{\mathrm{P}(B)-\frac{1}{6}}{\mathrm{P}(B)}$

따라서 $\mathrm{P}(B)=\frac{1}{3}$

\begin{enumerate}
  \setcounter{enumi}{25}
  \item 이 공장에서 생산하는 분필 중에서 $n$ 개를 임의추출 하여 얻은 표본평균이 $\overline{x_{1}}$ 일 때, 이를 이용하여 구한 모평균 $m$ 에 대한 신뢰도 $95 \%$ 의 신뢰구간은
\end{enumerate}

$\overline{x_{1}}-1.96 \times \frac{\sigma}{\sqrt{n}} \leq m \leq \overline{x_{1}}+1.96 \times \frac{\sigma}{\sqrt{n}}$

이 신뢰구간이 $80.51 \leq m \leq 81.49$ 이므로

$\overline{x_{1}}-1.96 \times \frac{\sigma}{\sqrt{n}}=80.51$

$\overline{x_{1}}+1.96 \times \frac{\sigma}{\sqrt{n}}=81.49$

(ㄴ)에서 (ㄱ)을 변끼리 빼면

$2 \times 1.96 \times \frac{\sigma}{\sqrt{n}}=81.49-80.51$

곧, $\frac{\sigma}{\sqrt{n}}=\frac{1}{4}$

이 공장에서 생산하는 분필 중에서 $4 n$ 개를 임의추출 하여 얻은 표본평균이 $\overline{x_{2}}$ 일 때, 이를 이용하여 구한 모평균 $m$ 에 대한 신뢰도 $99 \%$ 의 신뢰구간은

$\overline{x_{2}}-2.58 \times \frac{\sigma}{\sqrt{4 n}} \leq m \leq \overline{x_{2}}+2.58 \times \frac{\sigma}{\sqrt{4 n}}$\\
곧, $\overline{x_{2}}-1.29 \times \frac{1}{4} \leq m \leq \overline{x_{2}}+1.29 \times \frac{1}{4}$

이 신뢰구간이 $a \leq m \leq b$ 이므로

$a=\overline{x_{2}}-1.29 \times \frac{1}{4}$

$b=\overline{x_{2}}+1.29 \times \frac{1}{4}$

$b-a=\left(\overline{x_{2}}+1.29 \times \frac{1}{4}\right)-\left(\overline{x_{2}}-1.29 \times \frac{1}{4}\right)$

$=1.29 \times \frac{1}{4} \times 2$

$=0.645$

\begin{enumerate}
  \setcounter{enumi}{26}
  \item 최초에 상자에는 검은 공이 3 개 들어 있으므로 을이 꺼낸 두 공이 모두 검은색이려면 갑이 꺼낸 두 공 중 적어도 하나는 흰색이어야 한다. 갑이 꺼낸 두 공이 모두 흰색인 사건을 $A_{1}$ 갑이 꺼낸 두 공의 색이 서로 다른 사건을 $A_{2}$, 을이 꺼낸 두 공이 모두 검은색인 사건을 $B$ 라 하면 $\mathrm{P}(B)=\mathrm{P}\left(A_{1} \cap B\right)+\mathrm{P}\left(A_{2} \cap B\right)$ 이므로 구하는 확률은
\end{enumerate}

$\mathrm{P}\left(A_{1} \mid B\right)=\frac{\mathrm{P}\left(A_{1} \cap B\right)}{\mathrm{P}(B)}=\frac{\mathrm{P}\left(A_{1} \cap B\right)}{\mathrm{P}\left(A_{1} \cap B\right)+\mathrm{P}\left(A_{2} \cap B\right)}$

$\mathrm{P}\left(A_{1} \cap B\right)=\frac{{ }_{4} \mathrm{C}_{2}}{{ }_{7} \mathrm{C}_{2}} \times \frac{{ }_{3} \mathrm{C}_{2}}{{ }_{5} \mathrm{C}_{2}}=\frac{18}{{ }_{7} \mathrm{C}_{2} \times{ }_{5} \mathrm{C}_{2}}$,

$\mathrm{P}\left(A_{2} \cap B\right)=\frac{{ }_{4} \mathrm{C}_{1} \times{ }_{3} \mathrm{C}_{1}}{{ }_{7} \mathrm{C}_{2}} \times \frac{{ }_{2} \mathrm{C}_{2}}{{ }_{5} \mathrm{C}_{2}}=\frac{12}{{ }_{7} \mathrm{C}_{2} \times{ }_{5} \mathrm{C}_{2}}$ 이므로

구하는 확률은

$\mathrm{P}\left(A_{1} \mid B\right)=\frac{18}{18+12}=\frac{3}{5}$

\begin{enumerate}
  \setcounter{enumi}{27}
  \item 조건 (다)에 의하여 짝수 개의 검은색 볼펜을 받은 학생은 빨간색 볼펜을 받을 수 없다.
\end{enumerate}

곧, 홀수 개의 검은색 볼펜을 받은 학생만 빨간색 볼펜을 받을 수 있다.

...... (ㄱ)

검은색 볼펜의 개수가 10 이므로 홀수 개의 검은색 볼펜을 받는 학생의 수는 2 또는 4 이다.

( i ) 홀수 개의 검은색 볼펜을 받는 학생의 수가 4 인 경우

$\mathrm{A}, \mathrm{B}, \mathrm{C}, \mathrm{D}$ 가 받는 검은색 볼펜의 개수를 각각 $2 a+1,2 b+1,2 c+1,2 d+1$ ( $a, b, c, d$ 는 음이 아닌 정수)라 하면

$(2 a+1)+(2 b+1)+(2 c+1)+(2 d+1)=10$ $a+b+c+d=3$

순서쌍 $(a, b, c, d)$ 의 개수는

${ }_{4} \mathrm{H}_{3}={ }_{6} \mathrm{C}_{3}=20$

이때 (ㄱ)에 의하여 네 학생 모두 빨간색 볼펜을 받을 수 있고 홀수 개의 볼펜을 받은 학생이 존재 하게 되어 조건 (나)를 만족시킨다.

4 명에게 2 개의 빨간색 볼펜을 나누어 주는 경우의 수는

${ }_{4} \mathrm{H}_{2}={ }_{5} \mathrm{C}_{2}=10$

따라서 이 경우의 수는

$20 \times 10=200$

(ii) 홀수 개의 검은색 볼펜을 받는 학생의 수가 2 인 경우

홀수 개의 검은색 볼펜을 받는 학생을 정하는 경우의 수는

${ }_{4} \mathrm{C}_{2}=6$

예를 들어 홀수 개의 검은색 볼펜을 받는 학생이 $\mathrm{A}, \mathrm{B}$ 인 경우 (ㄱ)에 의하여 $\mathrm{A}, \mathrm{B}$ 만 빨간색 볼펜을 받을 수 있다.

그런데 $\mathrm{A}, \mathrm{B}$ 가 빨간색 볼펜을 각각 1 개씩 받게 되면 네 학생 모두 짝수 개의 볼펜을 받게 되어 조건 (나)를 만족시키지 않는다.\\
따라서 $\mathrm{A}, \mathrm{B}$ 중 한 명이 빨간색 볼펜 2 개를 모두 받아야 한다.

빨간색 볼펜을 2 개 받는 학생을 정하는 경우의 수는 ${ }_{2} \mathrm{C}_{1}=2$ 이다.

$\mathrm{C}, \mathrm{D}$ 는 짝수 개의 검은색 볼펜을 받아야 하므로 $\mathrm{A}, \mathrm{B}, \mathrm{C}, \mathrm{D}$ 가 받는 검은색 볼펜의 개수를 각각 $2 a+1,2 b+1,2 c+2,2 d+2(a, b, c, d$ 는 음이 아닌 정수)라 하면

$(2 a+1)+(2 b+1)+(2 c+2)+(2 d+2)=10$,

$a+b+c+d=2$

순서쌍 $(a, b, c, d)$ 의 개수는

${ }_{4} \mathrm{H}_{2}={ }_{5} \mathrm{C}_{2}=10$

따라서 이 경우의 수는

$6 \times 2 \times 10=120$

( i ), (ii)에 의하여 구하는 경우의 수는

$200+120=320$

\section*{[다른 풀이]}
빨간색 볼펜을 받는 학생의 수는 1 또는 2 이다.

( i ) 빨간색 볼펜을 받는 학생의 수가 1 인 경우 이때 한 명의 학생이 빨간색 볼펜 2 개를 받는다. 빨간색 볼펜을 받는 학생을 정하는 경우의 수는 ${ }_{4} \mathrm{C}_{1}=4$

예를 들어 빨간색 볼펜을 받는 학생을 A 라 할 때, 조건 (다)에 의하여 A 는 홀수 개의 검은색 볼펜을 받고 A 가 받는 모든 색의 볼펜의 개수는 홀수이다.

따라서 $\mathrm{B}, \mathrm{C}, \mathrm{D}$ 가 받는 볼펜의 개수와 관계없이 조건 (나)를 만족시킨다.

$\mathrm{A}, \mathrm{B}, \mathrm{C}, \mathrm{D}$ 가 받는 검은색 볼펜의 개수를 각각 $2 a+1, b+1, c+1, d+1(a, b, c, d$ 는 음이 아닌 정수)라 하면

$(2 a+1)+(b+1)+(c+1)+(d+1)=10$

$b+c+d=6-2 a$

$a=0$ 일 때 $b+c+d=6$ 이고

순서쌍 $(b, c, d)$ 의 개수는 ${ }_{3} \mathrm{H}_{6}={ }_{8} \mathrm{C}_{6}=28$, $a=1$ 일 때 $b+c+d=4$ 이고

순서쌍 $(b, c, d)$ 의 개수는 ${ }_{3} \mathrm{H}_{4}={ }_{6} \mathrm{C}_{4}=15$,

$a=2$ 일 때 $b+c+d=2$ 이고

순서쌍 $(b, c, d)$ 의 개수는 ${ }_{3} \mathrm{H}_{2}={ }_{4} \mathrm{C}_{2}=6$,

$a=3$ 일 때 $b+c+d=0$ 이고

순서쌍 $(b, c, d)$ 의 개수는 ${ }_{3} \mathrm{H}_{0}={ }_{2} \mathrm{C}_{0}=1$

따라서 이 경우의 수는

$4 \times(28+15+6+1)=200$

(ii) 빨간색 볼펜을 받는 학생의 수가 2 인 경우 두 명의 학생이 빨간색 볼펜을 각각 1 개씩 받는다. 빨간색 볼펜을 받는 학생을 정하는 경우의 수는 ${ }_{4} \mathrm{C}_{2}=6$

예를 들어 빨간색 볼펜을 받는 두 학생을 $\mathrm{A}, \mathrm{B}$ 라 할 때, 조건 (다)에 의하여 $\mathrm{A}, \mathrm{B}$ 는 검은색 볼펜을 홀수 개 받으므로 $\mathrm{A}, \mathrm{B}$ 가 받는 볼펜의 개수는 모두 짝수이다.

조건 (나)에 의하여 $\mathrm{C}, \mathrm{D}$ 중 적어도 한 명이 검은색 볼펜을 홀수 개 받아야 하고 검은색 볼펜이 10 개 이므로 $\mathrm{C}, \mathrm{D}$ 모두 검은색 볼펜을 홀수 개 받아야 한다.

$\mathrm{A}, \mathrm{B}, \mathrm{C}, \mathrm{D}$ 가 받는 검은색 볼펜의 개수를 각각 $2 a+1,2 b+1,2 c+1,2 d+1$ ( $a, b, c, d$ 는 음이 아닌 정수)라 하면

$(2 a+1)+(2 b+1)+(2 c+1)+(2 d+1)=10$,

$a+b+c+d=3$

순서쌍 $(a, b, c, d)$ 의 개수는

${ }_{4} \mathrm{H}_{3}={ }_{6} \mathrm{C}_{3}=20$

\section*{수학 영역}
따라서 이 경우의 수는

$6 \times 20=120$

( i ), (ii)에 의하여 구하는 경우의 수는 $200+120=320$

\begin{enumerate}
  \setcounter{enumi}{28}
  \item 확률변수 $X$ 는 정규분포 $\mathrm{N}\left(m, 2^{2}\right)$ 을 따르므로 $\mathrm{P}(x \leq X \leq x+4)$ 의 값은 $\frac{x+(x+4)}{2}=m$, 곧 $x=m-2$ 일 때 최댓값 $\mathrm{P}(m-2 \leq X \leq m+2)=\mathrm{P}(-1 \leq Z \leq 1)$
\end{enumerate}

\[
=2 \mathrm{P}(0 \leq Z \leq 1)
\]

을 갖는다.

또한 확률변수 $Y$ 는 정규분포 $\mathrm{N}\left(4, \sigma^{2}\right)$ 을 따르므로 $\mathrm{P}(y \leq Y \leq y+6)$ 의 값은

$\frac{y+(y+6)}{2}=4$, 곧 $y=1$ 일 때 최댓값

$\mathrm{P}(1 \leq Y \leq 7)=\mathrm{P}\left(-\frac{3}{\sigma} \leq Z \leq \frac{3}{\sigma}\right)$

\[
=2 \mathrm{P}\left(0 \leq Z \leq \frac{3}{\sigma}\right)
\]

을 갖는다.

따라서 $\mathrm{P}(x \leq X \leq x+4)+\mathrm{P}(y \leq Y \leq y+6)$ 은 $x=m-2$ 이고 $y=1$ 일 때

최댓값 $2 \mathrm{P}(0 \leq Z \leq 1)+2 \mathrm{P}\left(0 \leq Z \leq \frac{3}{\sigma}\right)$ 을 갖는다.

$x=y=a$ 에서

$m-2=1=a$, 곧 $m=3, a=1$

$2 \mathrm{P}(0 \leq Z \leq 1)+2 \mathrm{P}\left(0 \leq Z \leq \frac{3}{\sigma}\right)=4 \mathrm{P}(0 \leq Z \leq 1)$

에서

$\mathrm{P}\left(0 \leq Z \leq \frac{3}{\sigma}\right)=\mathrm{P}(0 \leq Z \leq 1)$

곧, $\sigma=3$

따라서 $m+\sigma+a=3+3+1=7$

\begin{enumerate}
  \setcounter{enumi}{29}
  \item 한 개의 동전을 5 번 던질 때, 앞면이 연속해서 나오는 경우가 있는 사건을 $A$, 뒷면이 연속해서 나오는 경우가 있는 사건을 $B$ 라 하자.
\end{enumerate}

이 때 앞면이 연속해서 나오는 경우가 있거나 뒷면이 연속해서 나오는 경우가 없는 사건은 $A \cup B^{C}$ 이므로 구하는 확률은 $\mathrm{P}\left(A \cup B^{C}\right)$ 이다.

그런데 $\left(A \cup B^{C}\right)^{C}=A^{C} \cap B$ 이므로

$\mathrm{P}\left(A \cup B^{C}\right)=1-\mathrm{P}\left(A^{C} \cap B\right)$

곧, $\mathrm{P}\left(A^{C} \cap B\right)$ 의 값을 구하고 이를 이용하여 $\mathrm{P}\left(A \cup B^{C}\right)$ 의 값을 구하자.

사건 $A^{C} \cap B$ 는 앞면이 연속해서 나오는 경우가 없고 뒷면이 연속해서 나오는 경우가 있는 사건이다.

한 개의 동전을 5 번 던질 때 나타나는 모든 경우의 수는

$2^{5}=32$

곧, $\mathrm{P}\left(A^{C} \cap B\right)=\frac{n\left(A^{C} \cap B\right)}{32}$

뒷면이 나오는 횟수에 따라 경우를 나누어 $n\left(A^{C} \cap B\right)$ 의 값을 구하자.

뒷면이 나오는 휫수가 1 이하인 경우에는 뒷면이 연속해서 나올 수 없으므로 사건 $A^{C} \cap B$ 는 일어나지 않는다.

(i) 뒷면이 2 번, 앞면이 3 번 나오는 경우 이때 사건 $A^{C} \cap B$ 는 일어나지 않는다.

(ii) 뒷면이 3 번, 앞면이 2 번 나오는 경우 이때 사건 $A^{C} \cap B$ 가 일어나는 경우는 다음과 같다. $\vee$ 뒤 $\vee$ 뒤 $\vee$ 뒤 $\vee$

위의 ' $V$ ' 표시한 곳 중 두 곳에 앞면이 들어가는 경우의 수는 ${ }_{4} \mathrm{C}_{2}=6$

이 중에서 다음과 같이 뒷면이 연속하지 않는 경우는 제외해야 한다.

뒤 앞 뒤 앞 뒤

따라서 뒷면이 3 번, 앞면이 2 번 나오고 사건 $A^{C} \cap B$ 가 일어나는 경우의 수는

$6-1=5$

(iii) 뒷면이 4 번, 앞면이 1 번 나오는 경우

이때 사건 $A^{C} \cap B$ 는 항상 일어나므로

이 경우의 수는

$\frac{5!}{4!}=5$

(iv) 뒷면이 5 번 나오는 경우

이때 사건 $A^{C} \cap B$ 는 항상 일어나므로

이 경우의 수는 1 이다.

( i ) (iv)에 의하여

$n\left(A^{C} \cap B\right)=0+5+5+1=11$ 이고

$\mathrm{P}\left(A^{C} \cap B\right)=\frac{n\left(A^{C} \cap B\right)}{32}=\frac{11}{32}$

따라서

$\mathrm{P}\left(A \cup B^{C}\right)=1-\mathrm{P}\left(A^{C} \cap B\right)$

\[
\begin{aligned}
& =1-\frac{11}{32} \\
& =\frac{21}{32}
\end{aligned}
\]

$p=\frac{21}{32}$ 이므로 $32 \times p=21$

\section*{미적분}
\textbackslash section*\{\begin{tabular}{|l|l|l|l|l|l|l|l|l|l|}
\hline
23 & (4) & 24 & (4) & 25 & (2) & 26 & (4) & 27 & (2) \\
\hline
\end{tabular}\} \begin{tabular}{l|l|l|l|l|l|}
\hline
28 & (3) & 29 & 2 & 30 & 50 \\
\hline
\end{tabular}

해설

\begin{enumerate}
  \setcounter{enumi}{22}
  \item $\sqrt{n^{2}+2 n}-\sqrt{n^{2}-6 n}$ $=\frac{\left(\sqrt{n^{2}+2 n}-\sqrt{n^{2}-6 n}\right)\left(\sqrt{n^{2}+2 n}+\sqrt{n^{2}-6 n}\right)}{\sqrt{n^{2}+2 n}+\sqrt{n^{2}-6 n}}$
\end{enumerate}

$=\frac{\left(n^{2}+2 n\right)-\left(n^{2}-6 n\right)}{\sqrt{n^{2}+2 n}+\sqrt{n^{2}-6 n}}$

$=\frac{8 n}{\sqrt{n^{2}+2 n}+\sqrt{n^{2}-6 n}}$

$=\frac{8}{\sqrt{1+\frac{2}{n}}+\sqrt{1-\frac{6}{n}}}$

이므로

$\lim _{n \rightarrow \infty}\left(\sqrt{n^{2}+2 n}-\sqrt{n^{2}-6 n}\right)$

$=\lim _{n \rightarrow \infty} \frac{8}{\sqrt{1+\frac{2}{n}}+\sqrt{1-\frac{6}{n}}}$

$=\frac{8}{\sqrt{1+0}+\sqrt{1-0}}$

$=4$

\begin{enumerate}
  \setcounter{enumi}{23}
  \item $x=t \ln t-t$ 에서
\end{enumerate}

$\frac{d x}{d t}=\ln t+t \times \frac{1}{t}-1=\ln t$

$y=2^{1-t}$ 에서

$\frac{d y}{d t}=2^{1-t} \times \ln 2 \times(-1)$

$\frac{d y}{d x}=\frac{\frac{d y}{d t}}{\frac{d x}{d t}}=\frac{2^{1-t} \times \ln 2 \times(-1)}{\ln t}($ 단, $t>1$ )

따라서 $t=2$ 일 때

$\frac{d y}{d x}=\frac{2^{-1} \times \ln 2 \times(-1)}{\ln 2}=-\frac{1}{2}$

\begin{enumerate}
  \setcounter{enumi}{24}
  \item 다음과 같이 경우를 나누어 $f(x)$ 를 살펴보자
\end{enumerate}

( i ) $|x|>a$ 인 경우

\[
\begin{aligned}
f(x) & =\lim _{n \rightarrow \infty} \frac{3|x|^{n}+a^{n}}{|x|^{n}+a^{n}} \\
& =\lim _{n \rightarrow \infty} \frac{3+\left(\frac{a}{|x|}\right)^{n}}{1+\left(\frac{a}{|x|}\right)^{n}} \\
& =3\left(\because 0<\frac{a}{|x|}<1\right)
\end{aligned}
\]

(ii) $|x|=a$ 인 경우

\[
\begin{aligned}
f(x) & =\lim _{n \rightarrow \infty} \frac{3|x|^{n}+a^{n}}{|x|^{n}+a^{n}} \\
& =\lim _{n \rightarrow \infty} \frac{3 a^{n}+a^{n}}{a^{n}+a^{n}} \\
& =2
\end{aligned}
\]

(iii) $|x|<a$ 인 경우

\[
\begin{aligned}
f(x) & =\lim _{n \rightarrow \infty} \frac{3|x|^{n}+a^{n}}{|x|^{n}+a^{n}} \\
& =\lim _{n \rightarrow \infty} \frac{3 \times\left(\frac{|x|}{a}\right)^{n}+1}{\left(\frac{|x|}{a}\right)^{n}+1} \\
& =1\left(\because 0<\frac{|x|}{a}<1\right)
\end{aligned}
\]

(i), (ii), (iii)에 의하여

$f(x)= \begin{cases}3 & (|x|>a) \\ 2 & (|x|=a) \\ 1 & (|x|<a)\end{cases}$

방정식 $f(x)=|x|$ 의 서로 다른 실근의 개수가 6 이 려면 함수 $f(x)$ 의 그래프와 함수 $y=|x|$ 의 그래프 의 교점의 개수가 다음 그림과 같이 6 이어야 한다.

\begin{center}
\includegraphics[max width=\textwidth]{2024_07_25_adbb0085b5e6606b27f6g-08(1)}
\end{center}

따라서 직선 $y=x$ 가 점 $(a, 2)$ 를 지나야 한다. 곧, $a=2$

\begin{enumerate}
  \setcounter{enumi}{25}
  \item $\int_{1}^{4} f(\sqrt{t}) d t$ 에서
\end{enumerate}

$\sqrt{t}=u$ 라 하면 $t=u^{2}$ 이므로 $1=2 u \times \frac{d u}{d t}$.

$t=1$ 일 때 $u=1$ 이고,

$t=4$ 일 때 $u=2$ 이므로

$\int_{1}^{4} f(\sqrt{t}) d t=\int_{1}^{2} 2 u f(u) d u$

$k=\int_{1}^{2} 2 u f(u) d u$ 라 하면

$f(x)=x \ln x+\int_{1}^{4} f(\sqrt{t}) d t$

$=x \ln x+k$

이므로

$k=\int_{1}^{2} 2 u f(u) d u$

$=\int_{1}^{2} 2 u(u \ln u+k) d u$

$=\int_{.1}^{2} 2 u^{2} \ln u d u+\int_{1}^{2} 2 k u d u$

$=\left[\frac{2}{3} u^{3} \times \ln u\right]_{1}^{2}-\int_{1}^{2}\left(\frac{2}{3} u^{3} \times \frac{1}{u}\right) d u+\left[k u^{2}\right]_{1}^{2}$

$=\left[\frac{2}{3} u^{3} \ln u\right]_{1}^{2}-\left[\frac{2}{9} u^{3}\right]_{1}^{2}+\left[k u^{2}\right]_{1}^{2}$

$=\frac{16}{3} \ln 2-\frac{14}{9}+3 k$

따라서 $f(1)=k=\frac{7}{9}-\frac{8}{3} \ln 2$

\begin{enumerate}
  \setcounter{enumi}{26}
  \item $y=e^{x}+a$ 에서 $y^{\prime}=e^{x}$ 이므로
\end{enumerate}

이 곡선 위의 점 $\mathrm{A}\left(t, e^{t}+a\right)$ 에서의 접선 $l$ 의 기울기는 $e^{t}$ 이다.

따라서 점 $\mathrm{A}\left(t, e^{t}+a\right)$ 를 지나고 직선 $l$ 에 수직인 직선의 방정식은

$y=-\frac{1}{e^{t}}(x-t)+e^{t}+a$

$y=0$ 일 때 $0=-\frac{1}{e^{t}}(x-t)+e^{t}+a$ 에서

$x=e^{2 t}+a e^{t}+t$ 이므로

$f(t)=e^{2 t}+a e^{t}+t$,

$x=0$ 일 때 $y=-\frac{1}{e^{t}}(0-t)+e^{t}+a$ 에서

$y=e^{t}+a+\frac{t}{e^{t}}$ 이므로

$g(t)=e^{t}+a+\frac{t}{e^{t}}$

이때 $\lim _{t \rightarrow 0} \frac{f(t)-g(t)}{t}$

$=\lim _{t \rightarrow 0} \frac{\left(e^{2 t}+a e^{t}+t\right)-\left(e^{t}+a+\frac{t}{e^{t}}\right)}{t}$

$=\lim _{t \rightarrow 0} \frac{\left(e^{2 t}-e^{t}\right)+\left(a e^{t}-a\right)+\left(t-\frac{t}{e^{t}}\right)}{t}$

$=\lim _{t \rightarrow 0}\left\{\frac{e^{t}\left(e^{t}-1\right)}{t}+\frac{a\left(e^{t}-1\right)}{t}+\left(1-\frac{1}{e^{t}}\right)\right\}$

$=1+a+0$

$=1+a$

이고 주어진 조건에서 $\lim _{t \rightarrow 0} \frac{f(t)-g(t)}{t}=3$ 이므로

$1+a=3$ 에서 $a=2$

\begin{enumerate}
  \setcounter{enumi}{27}
  \item 조건 (나)에서 정수 $n$ 에 대하여
\end{enumerate}

닫힌구간 $[2 n \pi,(2 n+1) \pi]$ 에서

$\left|f^{\prime}(x)\right|=2 \sin x$,

닫힌구간 $[(2 n-1) \pi, 2 n \pi]$ 에서

$\left|f^{\prime}(x)\right|=0$

또한 도함수 $f^{\prime}(x)$ 는 실수 전체의 집합에서 연속 이므로

닫힌구간 $[2 n \pi,(2 n+1) \pi]$ 에서

$f^{\prime}(x)=2 \sin x$ 또는 $f^{\prime}(x)=-2 \sin x$,

닫힌구간 $[(2 n-1) \pi, 2 n \pi]$ 에서

$f^{\prime}(x)=0$

조건 (다)에서

$\int_{0}^{t} f(x) d x=\int_{4 \pi}^{t+4 \pi}\{k-f(x)\} d x$ 의

양변을 $t$ 에 대하여 미분하면

$f(t)=k-f(t+4 \pi)$

곧, 모든 실수 $x$ 에 대하여

$f(x)=k-f(x+4 \pi)$ 이고

$f(x+4 \pi)=-f(x)+k$

(ㄷ)에 $x=0$ 을 대입하면

$f(4 \pi)=-f(0)+k=k>0$

만약 함수 $f^{\prime}(x)$ 가 닫힌구간 $[0, \pi]$ 에서

$f^{\prime}(x)=-2 \sin x$ 이면

$f(4 \pi)=\int_{0}^{4 \pi} f^{\prime}(x) d x(\because$ 조건 (가) $f(0)=0)$

\[
\begin{aligned}
& =\int_{0}^{\pi}(-2 \sin x) d x+\int_{\pi}^{2 \pi} 0 d x \\
& \quad \quad \quad \int_{2 \pi}^{3 \pi} f^{\prime}(x) d x+\int_{3 \pi}^{4 \pi} 0 d x(\because \text { (ㄴ) } \\
& \leq \int_{0}^{\pi}(-2 \sin x) d x+\int_{2 \pi}^{3 \pi} 2 \sin x d x(\because \text { (ㄱ) }) \\
& =0
\end{aligned}
\]

곧, $f(4 \pi) \leq 0$ 이 되어 (ㄹ)을 만족시키지 않는다.

따라서 닫힌구간 $[0, \pi]$ 에서

$f^{\prime}(x)=2 \sin x$

만약 함수 $f^{\prime}(x)$ 가 닫힌구간 $[2 \pi, 3 \pi]$ 에서 $f^{\prime}(x)=-2 \sin x$ 이면

$f(4 \pi)=\int_{0}^{4 \pi} f^{\prime}(x) d x(\because$ 조건 (가) $f(0)=0)$

\[
\begin{aligned}
& =\int_{0}^{\pi} 2 \sin x d x+\int_{\pi}^{2 \pi} 0 d x \\
& \quad \quad+\int_{2 \pi}^{3 \pi}(-2 \sin x) d x+\int_{3 \pi}^{4 \pi} 0 d x(\because \text { (ㄴ) } \\
& =0\left(\because \int_{2 \pi}^{3 \pi}(-2 \sin x) d x=\int_{0}^{\pi}(-2 \sin x) d x\right)
\end{aligned}
\]

곧, $f(4 \pi)=0$ 이 되어 (ㄹ)을 만족시키지 않는다.\\
따라서 닫힌구간 $[2 \pi, 3 \pi]$ 에서

$f^{\prime}(x)=2 \sin x$

정리하면

$f^{\prime}(x)= \begin{cases}2 \sin x & (0 \leq x<\pi) \\ 0 & (\pi \leq x<2 \pi) \\ 2 \sin x & (2 \pi \leq x<3 \pi) \\ 0 & (3 \pi \leq x<4 \pi)\end{cases}$

이므로

$f(x)= \begin{cases}-2 \cos x+C_{1} & (0 \leq x<\pi) \\ C_{2} & (\pi \leq x<2 \pi) \\ -2 \cos x+C_{3} & (2 \pi \leq x<3 \pi) \\ C_{4} & (3 \pi \leq x \leq 4 \pi)\end{cases}$

$\left(C_{1}, C_{2}, C_{3}, C_{4}\right.$ 는 적분상수 $)$

$f(0)=0$ 에서

$-2 \times \cos 0+C_{1}=0$, 곧 $C_{1}=2$

함수 $f(x)$ 는 $x=\pi$ 에서 연속이므로

$-2 \times \cos \pi+C_{1}=C_{2}$, 곧 $C_{2}=4$

함수 $f(x)$ 는 $x=2 \pi$ 에서 연속이므로

$C_{2}=-2 \times \cos 2 \pi+C_{3}$, 곧 $C_{3}=6$

함수 $f(x)$ 는 $x=3 \pi$ 에서 연속이므로

$-2 \times \cos 3 \pi+C_{3}=C_{4}$, 곧 $C_{4}=8$

따라서

$f(x)= \begin{cases}-2 \cos x+2 & (0 \leq x<\pi) \\ 4 & (\pi \leq x<2 \pi) \\ -2 \cos x+6 & (2 \pi \leq x<3 \pi) \\ 8 & (3 \pi \leq x \leq 4 \pi)\end{cases}$

\begin{center}
\includegraphics[max width=\textwidth]{2024_07_25_adbb0085b5e6606b27f6g-08}
\end{center}

이때 $f(4 \pi)=8$ 이므로 (ㄹ)에서

$k=8$

(ㄷ)에 $x$ 대신 $x+4 \pi$ 를 대입하여 정리하면

$f(x+8 \pi)=-f(x+4 \pi)+k$

\[
\begin{aligned}
& =-\{-f(x)+k\}+k \\
& =f(x)
\end{aligned}
\]

곧, $f(x+8 \pi)=f(x)$

함수 $f(x)$ 의 주기가 $8 \pi$ 이므로

$\int_{8 \pi}^{\frac{25}{2} \pi} f(x) d x=\int_{8 \pi}^{\frac{25}{2} \pi} f(x+8 \pi) d x$

\[
=\int_{0}^{\frac{9}{2} \pi} f(x) d x
\]

\[
=\int_{0}^{4 \pi} f(x) d x+\int_{4 \pi}^{\frac{9}{2} \pi} f(x) d x
\]

( i ) $\int_{0}^{4 \pi} f(x) d x$ 의 값을 구하자.

\[
\begin{aligned}
& \int_{0}^{4 \pi} f(x) d x \\
& =\int_{0}^{\pi} f(x) d x+\int_{\pi}^{2 \pi} f(x) d x \\
& \quad+\int_{2 \pi}^{3 \pi} f(x) d x+\int_{3 \pi}^{4 \pi} f(x) d x \\
& =\int_{0}^{\pi}(-2 \cos x+2) d x+\int_{\pi}^{2 \pi} 4 d x \\
& \quad+\int_{2 \pi}^{3 \pi}(-2 \cos x+6) d x+\int_{3 \pi}^{4 \pi} 8 d x
\end{aligned}
\]

\section*{수학 영역}
\[
\begin{aligned}
& =[-2 \sin x+2 x]_{0}^{\pi}+[4 x]_{\pi}^{2 \pi} \\
& +[-2 \sin x+6 x]_{2 \pi}^{3 \pi}+[8 x]_{3 \pi}^{4 \pi}
\end{aligned}
\]

\begin{center}
\includegraphics[max width=\textwidth]{2024_07_25_adbb0085b5e6606b27f6g-09}
\end{center}

\begin{center}
\includegraphics[max width=\textwidth]{2024_07_25_adbb0085b5e6606b27f6g-09(2)}
\end{center}

(ii) $\int_{4 \pi}^{\frac{9}{2} \pi} f(x) d x$ 의 값을 구하자.

$\int_{4 \pi}^{\frac{9}{2} \pi} f(x) d x$

$=\int_{4 \pi}^{\frac{9}{2} \pi}\{-f(x+4 \pi)+8\} d x(\because$ (ㄷ), (ㅁ) $)$

$=\int_{0}^{\frac{1}{2} \pi}\{-f(x+8 \pi)+8\} d x$

$=\int_{0}^{\frac{1}{2} \pi}\{-f(x)+8\} d x(\because f(x+8 \pi)=f(x))$

$=\int_{0}^{\frac{1}{2} \pi}\{-(-2 \cos x+2)+8\} d x$

$=\int_{0}^{\frac{1}{2} \pi}(2 \cos x+6) d x$

$=[2 \sin x+6 x]_{0}^{\frac{1}{2} \pi}$

$=2+3 \pi$

( i ), (ii)에 의하여

$\int_{0}^{\frac{9}{2} \pi} f(x) d x=\int_{0}^{4 \pi} f(x) d x+\int_{4 \pi}^{\frac{9}{2} \pi} f(x) d x$

\[
=20 \pi+(2+3 \pi)
\]

\[
=23 \pi+2
\]

따라서

$k+\int_{8 \pi}^{\frac{25}{2} \pi} f(x) d x=8+(23 \pi+2)=23 \pi+10$

\begin{enumerate}
  \setcounter{enumi}{28}
  \item 등차수열 $\left\{a_{n}\right\}$ 의 첫째항을 $a$ ( $a$ 는 자연수), 공차를 $d$ ( $d$ 는 자연수)라 하자.
\end{enumerate}

$a_{n}=a+(n-1) d=(a-d)+d n$ 이므로

$r^{a_{n}}=r^{(a-d)+d n}=r^{a-d} \times\left(r^{d}\right)^{n}$

$|r|>1$ 이면 $\left|r^{d}\right|>1$ 이므로

$\lim _{n \rightarrow \infty} r^{n}$ 과 $\lim _{n \rightarrow \infty}\left\{r^{a-d} \times\left(r^{d}\right)^{n}\right\}$ 의 값이 모두 존재하지

않는다.

$|r|<1$ 이면 $\left|r^{d}\right|<1$ 이므로

$\lim _{n \rightarrow \infty} r^{n}=0, \lim _{n \rightarrow \infty}\left\{r^{a-d} \times\left(r^{d}\right)^{n}\right\}=0$

$r=1$ 이면

$\lim _{n \rightarrow \infty} r^{n}=1, \lim _{n \rightarrow \infty} r^{a_{n}}=1$

따라서 두 수열 $\left\{r^{n}\right\},\left\{r^{a_{n}}\right\}$ 중 하나만 수렴하려면 $r=-1$ 이어야 한다.

이때 수열 $\left\{r^{n}\right\}$, 곧 $\left\{(-1)^{n}\right\}$ 은 발산하므로

수열 $\left\{r^{a_{n}}\right\}$, 곧 $\left\{(-1)^{a-d} \times(-1)^{d n}\right\}$ 이 수렴한다. 따라서 $d$ 는 짝수이다.

그런데 $a_{2}=10$ 이고 $a_{1}=a$ 가. 자연수이므로

가능한 $d$ 의 값은 $2,4,6,8$ 이다.

한편,

\[
\begin{aligned}
\sum_{n=1}^{\infty} \frac{1}{a_{n+1} a_{n}} & =\frac{1}{d} \sum_{n=1}^{\infty}\left(\frac{1}{a_{n}}-\frac{1}{a_{n+1}}\right) \\
& =\frac{1}{d} \lim _{n \rightarrow \infty} \sum_{k=1}^{n}\left(\frac{1}{a_{k}}-\frac{1}{a_{k+1}}\right) \\
& =\frac{1}{d} \lim _{n \rightarrow \infty}\left(\frac{1}{a_{1}}-\frac{1}{a_{n+1}}\right)
\end{aligned}
\]

\[
\begin{aligned}
& =\frac{1}{d} \lim _{n \rightarrow \infty}\left(\frac{1}{a}-\frac{1}{a+d n}\right) \\
& =\frac{1}{d} \times\left(\frac{1}{a}-0\right) \\
& =\frac{1}{d a} \\
& =\frac{1}{d(10-d)}\left(\because a_{2}=10\right)
\end{aligned}
\]

$d$ 의 값이 $2,4,6,8$ 일 때

$d(10-d)$ 의 값은 각각 $16,24,24,16$ 이므로 $\frac{1}{d(10-d)}$ 은 $d=4$ 또는 $d=6$ 일 때 최솟값 $\frac{1}{24}$ 을 갖는다.

곧, $\sum_{n=1}^{\infty} \frac{1}{a_{n+1} a_{n}}$ 의 최솟값은 $\frac{1}{24}$ 이다.

$m=\frac{1}{24}$ 이므로 $48 \times m=2$

\begin{enumerate}
  \setcounter{enumi}{29}
  \item $h(x)=\frac{x^{3}-3 x^{2}+a}{x^{2}+1}$ 라 하자.
\end{enumerate}

곡선 $y=h(x)$ 와 직선 $y=t$ 의 교점 중 $x$ 좌표가 최대인 점의 $x$ 좌표가 $f(t)$ 이고 $x$ 좌표가 최소인 점의 $x$ 좌표가 $g(t)$ 이다. 함수 $h(x)$ 의 그래프의 개형에 대하여 살펴보자.

\[
h^{\prime}(x)=\frac{\left(3 x^{2}-6 x\right)\left(x^{2}+1\right)-\left(x^{3}-3 x^{2}+a\right) \times 2 x}{\left(x^{2}+1\right)^{2}}
\]

\[
=\frac{x\left(x^{3}+3 x-2 a-6\right)}{\left(x^{2}+1\right)^{2}}
\]

함수 $y=x^{3}+3 x-2 a-6$ 에서

$y^{\prime}=3 x^{2}+3>0$ 이므로

이 함수는 실수 전체의 집합에서 증가한다.

곧, 방정식 $x^{3}+3 x-2 a-6=0$ 의 실근의 개수는 1 이다.

이 실근을 $\beta$ 라 하면

$\beta^{3}+3 \beta-2 a-6=0$

만약 $\beta=0$, 곧 $a=-3$ 이면

$h^{\prime}(x)=\frac{x^{2}\left(x^{2}+3\right)}{\left(x^{2}+1\right)^{2}} \geq 0$ 이므로

함수 $h(x)$ 는 실수 전체의 집합에서 증가한다.

이때 모든 실수 $t$ 에 대하여 $f(t)=g(t)$ 이므로

' $f(t) \neq g(t)$ 를 만족시키는 $t$ 의 값의 범위가 $0 \leq t \leq k(k>0)$ 이다.'를 만족시키지 않는다. 따라서 $\beta \neq 0$ 이다.

방정식 $h^{\prime}(x)=0$ 은 두 실근 $0, \beta$ 를 가지므로

함수 $h(x)$ 는 극댓값과 극솟값을 갖는다.

이 두 극값을 각각 $M, m$ 이라 하자.

$\lim _{x \rightarrow-\infty} h(x)=-\infty, \lim _{x \rightarrow \infty} h(x)=\infty$ 이므로

함수 $h(x)$ 의 그래프의 개형은 다음과 같다.

\begin{center}
\includegraphics[max width=\textwidth]{2024_07_25_adbb0085b5e6606b27f6g-09(3)}
\end{center}

따라서

$t<m$ 또는 $t>M$ 일 때 $f(t)=g(t)$ 이고, $m \leq t \leq M$ 일 때 $f(t) \neq g(t)$ 이다.

이때 $m \leq t \leq M$ 과 $0 \leq t \leq k$ 가 같으므로 $m=0, M=k$

곧, ' $h(0)=0, h(\beta)=k$ ' 또는 ' $h(0)=k, h(\beta)=0$ 만약 $h(0)=0, h(\beta)=k$ 이면 $\beta<0$ 이고, $h(0)=a$ 에서 $a=0$

이 때 $h(x)=\frac{x^{3}-3 x^{2}}{x^{2}+1}$ 이므로

$h(\beta)=\frac{\beta^{3}-3 \beta^{2}}{\beta^{2}+1}=\frac{\beta^{2}(\beta-3)}{\beta^{2}+1}<0$

곧, $k<0$ 이 되어 모순이다. 따라서 $h(0)=k, h(\beta)=0$

$h(\beta)=\frac{\beta^{3}-3 \beta^{2}+a}{\beta^{2}+1}=0$ 에서

$\beta^{3}-3 \beta^{2}+a=0$

(ㄱ), (ㄴ)을 연립하여 풀면 $\beta=2, a=4$

따라서 $h(x)=\frac{x^{3}-3 x^{2}+4}{x^{2}+1}$

이때 $h(0)=4, h(2)=0$ 이므로 $k=4$

또한 $t=4$ 일 때 방정식 $h(x)=4$ 에서

$\frac{x^{3}-3 x^{2}+4}{x^{2}+1}=4$

$x^{3}-3 x^{2}+4=4 x^{2}+4$

$x^{2}(x-7)=0$

$x=0$ 또는 $x=7$

...... (ㄷ)

따라서 함수 $h(x)$ 의 그래프의 개형은 다음과 같다.

\begin{center}
\includegraphics[max width=\textwidth]{2024_07_25_adbb0085b5e6606b27f6g-09(1)}
\end{center}

$k=4$ 이므로 $f^{\prime}(k)=f^{\prime}(4)$ 이고,

(ㄷ)에서 방정식 $h(x)=4$ 의 두 실근이 0 과 7 이므로

$f(4)=7$ 이고 $g(4)=0$

$t>0$ 에서 $h(f(t))=t(\because f(t)>2)$

양변을 $t$ 에 대하여 미분하면

$h^{\prime}(f(t)) f^{\prime}(t)=1$

$t=4$ 를 대입하면

$h^{\prime}(f(4)) f^{\prime}(4)=1$

$f^{\prime}(4)=\frac{1}{h^{\prime}(7)}$

$h^{\prime}(x)=\frac{x\left(x^{3}+3 x-14\right)}{\left(x^{2}+1\right)^{2}}$ 에서

$h^{\prime}(7)=\frac{49}{50}$ 이므로

$f^{\prime}(4)=\frac{50}{49}$

따라서 $49 \times f^{\prime}(k)=49 \times f^{\prime}(4)=50$

\section*{ะ:}
$h(x)=\frac{x^{3}-3 x^{2}+a}{x^{2}+1}=x-3-\frac{x-a-3}{x^{2}+1}$

에서

$x \rightarrow \infty$ 일 때 $-\frac{x-a-3}{x^{2}+1} \rightarrow 0-$,

$x \rightarrow-\infty$ 일 때 $-\frac{x-a-3}{x^{2}+1} \rightarrow 0+$ 이므로

곡선 $y=h(x)$ 는 직선 $y=x-3$ 을 점근선으로 갖는다.

\section*{수학 영역}
기하

\begin{center}
\begin{tabular}{|l|l|l|l|l|l|l|l|l|l|}
\hline
23 & $(5)$ & 24 & $(3)$ & 25 & $(2)$ & 26 & $(1)$ & 27 & $(2)$ \\
\hline
28 & $(2$ & 29 & 52 & 30 & 30 &  &  &  &  \\
\hline
\end{tabular}
\end{center}

\begin{center}
\begin{tabular}{l|l|l|l|l|l|}
\hline
28 & (2) & 29 & 52 & 30 & 30 \\
\hline
\end{tabular}
\end{center}

\section*{해설}
\begin{enumerate}
  \setcounter{enumi}{22}
  \item 점 $\mathrm{A}(1,-4,3)$ 을 $y z$ 평면에 대하여 대칭이동한 점은 $\mathrm{B}(-1,-4,3)$ 이고 원점에 대하여 대칭이동한 점은 $\mathrm{C}(-1,4,-3)$ 이므로
\end{enumerate}

$\overline{\mathrm{BC}}$

$=\sqrt{\{(-1)-(-1)\}^{2}+\{4-(-4)\}^{2}+\{(-3)-3\}^{2}}$

$=10$

\begin{enumerate}
  \setcounter{enumi}{23}
  \item 쌍곡선 $\frac{x^{2}}{a^{2}}-\frac{y^{2}}{b^{2}}=-1$ 의 두 초점의 좌표를
\end{enumerate}

$(0, c),(0,-c)(c>0)$ 이라 하자.

두 초점 사이의 거리가 20 이므로

$2 c=20$, 곧 $c=10$

따라서 $a^{2}+b^{2}=10^{2}$

$\frac{b}{a}=\frac{4}{3}$

(ㄱ), (ㄴ)을 연립하여 풀면

$a=6, b=8(\because a>0, b>0)$

따라서 쌍곡선의 주축의 길이는 $2 b=16$

\begin{enumerate}
  \setcounter{enumi}{24}
  \item 포물선 $y^{2}=9 x$ 위의 점 A 의 좌표를
\end{enumerate}

$\mathrm{A}(9, a)(a>0)$ 이라 하면

$a^{2}=9 \times 9$ 에서 $a=9$

점 $\mathrm{A}(9,9)$ 에서의 접선의 방정식은

$9 y=\frac{9}{2}(x+9)$

곧, $y=\frac{1}{2}(x+9)$

포물선 $y^{2}=9 x$ 위의 점 B 의 좌표를 $\mathrm{B}(b, c)$ 라 하자. 점 B 에서의 접선의 방정식은

$c y=\frac{9}{2}(x+b)$

곧, $y=\frac{9}{2 c}(x+b)$

두 직선 (ㄱ), (ㄴ)은 서로 수직이므로

$\frac{1}{2} \times \frac{9}{2 c}=-1$ 에서 $c=-\frac{9}{4}$

곧, 점 B 의 $y$ 좌표는 $-\frac{9}{4}$ 이다.

\begin{enumerate}
  \setcounter{enumi}{25}
  \item 선분 AD 의 중점을 H 라 하면 $\overline{\mathrm{AC}} / / \overline{\mathrm{HN}}$ 이므로 두 직선 $\mathrm{MN}, \mathrm{AC}$ 가 이루는 예각의 크기는
\end{enumerate}

두 직선 $\mathrm{MN}, \mathrm{HN}$ 이 이루는 예각의 크기와 같다. 곧, $\angle \mathrm{MNH}=\theta$

\begin{center}
\includegraphics[max width=\textwidth]{2024_07_25_adbb0085b5e6606b27f6g-10}
\end{center}

한편, 선분 AB 의 중점을 $\mathrm{M}^{\prime}$ 이라 하면 $\overline{\mathrm{EA}} / / \overline{\mathrm{MM}^{\prime}}$ 이고 $\overline{\mathrm{EA}} \perp$ (평면 ABC )이므로 점 M 에서 평면 ABC 에 내린 수선의 발은 $\mathrm{M}^{\prime}$ 이다.

이 때 $\overline{\mathrm{AM}^{\prime}}=\overline{\mathrm{AH}}=1$ 이고 $\angle \mathrm{M}^{\prime} \mathrm{AH}=\frac{2 \pi}{3}$ 이므로

이등변삼각형 $\mathrm{M}^{\prime} \mathrm{AH}$ 에서 $\angle \mathrm{AHM}^{\prime}=\frac{\pi}{6}$

$\overline{\mathrm{ND}}=\overline{\mathrm{DH}}=\overline{\mathrm{HN}}=1$ 이므로

정삼각형 DHN 에서 $\angle \mathrm{DHN}=\frac{\pi}{3}$

따라서 $\angle \mathrm{M}^{\prime} \mathrm{HN}=\frac{\pi}{2}$

\begin{center}
\includegraphics[max width=\textwidth]{2024_07_25_adbb0085b5e6606b27f6g-10(1)}
\end{center}

$\overline{\mathrm{MM}^{\prime}} \perp$ (평면 ABC )이고 $\overline{\mathrm{M}^{\prime} \mathrm{H}} \perp \overline{\mathrm{NH}}$ 이므로

삼수선의 정리에 의하여 $\overline{\mathrm{MH}} \perp \overline{\mathrm{NH}}$

\begin{center}
\includegraphics[max width=\textwidth]{2024_07_25_adbb0085b5e6606b27f6g-10(3)}
\end{center}

또한 $\overline{\mathrm{MM}^{\prime}}=\frac{1}{2} \overline{\mathrm{AE}}=1$ 이고 $\overline{\mathrm{M}^{\prime} \mathrm{N}}=\overline{\mathrm{AD}}=2$ 이므로 직각삼각형 $\mathrm{MM}^{\prime} \mathrm{N}$ 에서

$\overline{\mathrm{MN}}=\sqrt{1^{2}+2^{2}}=\sqrt{5}$

따라서 직각삼각형 MNH 에서

$\cos \theta=\cos (\angle \mathrm{MNH})$

\[
\begin{aligned}
& =\frac{\overline{\mathrm{NH}}}{\overline{\mathrm{MN}}} \\
& =\frac{1}{\sqrt{5}} \\
& =\frac{\sqrt{5}}{5}
\end{aligned}
\]

\begin{enumerate}
  \setcounter{enumi}{26}
  \item $|\overrightarrow{\mathrm{OP}}|^{2}=\overrightarrow{\mathrm{OP}} \cdot \overrightarrow{\mathrm{OA}}$ 에서
\end{enumerate}

$|\overrightarrow{\mathrm{OP}}|^{2}-\overrightarrow{\mathrm{OP}} \cdot \overrightarrow{\mathrm{OA}}=0$

$\overrightarrow{\mathrm{OP}} \cdot(\overrightarrow{\mathrm{OP}}-\overrightarrow{\mathrm{OA}})=0$

$\overrightarrow{\mathrm{OP}} \cdot \overrightarrow{\mathrm{AP}}=0$

따라서 점 P 는 선분 OA 를 지름으로 하는 원 위의 점이다.

이 원을 $C$ 라 하자.

점 $\mathrm{B}(3,9)$ 에 대하여 두 벡터 $\overrightarrow{\mathrm{OB}}, \overrightarrow{\mathrm{PB}}$ 가 이루는 각의 크기를. $\theta(0 \leq \theta \leq \pi)$ 라 하면

$\overrightarrow{\mathrm{OB}} \cdot \frac{\overrightarrow{\mathrm{PB}}}{|\overrightarrow{\mathrm{PB}}|}$

$=|\overrightarrow{\mathrm{OB}}| \times\left|\frac{\overrightarrow{\mathrm{PB}}}{|\overrightarrow{\mathrm{PB}}|}\right| \times \cos \theta$

$=3 \sqrt{10} \cos \theta(\because|\overrightarrow{\mathrm{OB}}|=3 \sqrt{10})$

이므로 $\cos \theta$ 의 값이 최소일 때 $\overrightarrow{\mathrm{OB}} \cdot \frac{\overrightarrow{\mathrm{PB}}}{|\overrightarrow{\mathrm{PB}}|}$ 의 값도

최소이다.

한편, 원 $C$ 의 중심을 C 라 하면 점 C 는 선분 OA 의

중점이므로 좌표가 $\mathrm{C}(1,3)$ 이고 네 점 $\mathrm{O}, \mathrm{C}(1,3)$, $\mathrm{A}(2 ; 6), \mathrm{B}(3,9)$ 가 한 직선 위에 있으므로 두 벡터 $\overrightarrow{\mathrm{CB}}, \overrightarrow{\mathrm{PB}}$ 가 이루는 각의 크기도 $\theta$ 이다 따라서 $\theta$ 의 값은 다음 그림과 같이 직선 PB 가 원 $C$ 에 접할 때 최대이고 이때 $\cos \theta$ 의 값은 최소이다.

\begin{center}
\includegraphics[max width=\textwidth]{2024_07_25_adbb0085b5e6606b27f6g-10(4)}
\end{center}

이때 접하는 점 중 하나를 $\mathrm{P}_{0}$ 이라 하면 직각삼각형 $\mathrm{CBP}_{0}$ 에서

$\overline{\mathrm{CP}_{0}}=\overline{\mathrm{CA}}=\sqrt{10}, \overline{\mathrm{CB}}=2 \sqrt{10}$ 이므로

$\angle \mathrm{CBP}_{0}=\frac{\pi}{6}$

곧, $\cos \theta$ 의 최솟값은 $\cos \frac{\pi}{6}=\frac{\sqrt{3}}{2}$

따라서 $\overrightarrow{\mathrm{OB}} \cdot \frac{\overrightarrow{\mathrm{PB}}}{|\overrightarrow{\mathrm{PB}}|}$, 곧 $3 \sqrt{10} \cos \theta$ 의 최솟값은

$3 \sqrt{10} \times \frac{\sqrt{3}}{2}=\frac{3 \sqrt{30}}{2}$

\begin{enumerate}
  \setcounter{enumi}{27}
  \item 평행사변형 PQF F에서
\end{enumerate}

$\overline{\mathrm{F}^{\prime} \mathrm{Q}}=4, \overline{\mathrm{F}^{\prime} \mathrm{S}}=1$ 이므로 $\overline{\mathrm{SQ}}=3$

또한 $\overline{\mathrm{FP}}=\overline{\mathrm{F}^{\prime} \mathrm{Q}}=4$ 이므로 $\overline{\mathrm{FR}}=k$ 라 하면

$\overline{\mathrm{RP}}=4-k$

\begin{center}
\includegraphics[max width=\textwidth]{2024_07_25_adbb0085b5e6606b27f6g-10(2)}
\end{center}

평행한 두 직선 $\mathrm{F}^{\prime} \mathrm{Q}, \mathrm{FP}$ 사이의 거리를 $h$ 라 하면 두 사다리꼴 $\mathrm{PQSR}, \mathrm{RSF}^{\prime} \mathrm{F}$ 의 넓이는 각각

$S_{1}=\frac{1}{2} \times h \times(\overline{\mathrm{SQ}}+\overline{\mathrm{RP}})$

$=\frac{1}{2} h\{3+(4-k)\}$

$=\frac{1}{2} h(7-k)$

$S_{2}=\frac{1}{2} \times h \times\left(\overline{\mathrm{F}^{\prime} \mathrm{S}}+\overline{\mathrm{FR}}\right)$

$=\frac{1}{2} h(1+k)$

$S_{1}: S_{2}=13: 3$ 이므로

$\frac{1}{2} h(7-k): \frac{1}{2} h(1+k)=13: 3$ 에서

$(7-k):(1+k)=13: 3$

$13(1+k)=3(7-k)$

곧, $k=\frac{1}{2}$

따라서 $\overline{\mathrm{FR}}=k=\frac{1}{2}$

\section*{수학 영역}
한편, 타원 $C$ 의 장축의 길이를 $2 a$ 라 하면

$\overline{\mathrm{F}^{\prime} \mathrm{S}}=1$ 이므로 $\overline{\mathrm{FS}}=2 a-1$

$\overline{\mathrm{FR}}=\frac{1}{2}$ 이므로 $\overline{\mathrm{F}^{\prime} \mathrm{R}}=2 a-\frac{1}{2}$

직각삼각형 $\mathrm{QF}^{\prime} \mathrm{O}$ 에서 $\angle \mathrm{QF}$ 等= $=\theta$ 라. 하면. $\cos \theta=\frac{\overline{\mathrm{OF}^{\prime}}}{\overline{\mathrm{F}^{\prime} \mathrm{Q}}}=\frac{c}{4}$

\begin{center}
\includegraphics[max width=\textwidth]{2024_07_25_adbb0085b5e6606b27f6g-11(1)}
\end{center}

삼각형 $\mathrm{SF}^{\prime} \mathrm{F}$ 에서 코사인법칙에 의하여

$\overline{\mathrm{FS}}^{2}={\overline{\mathrm{F}^{\prime} \mathrm{S}}}^{2}+\overline{\mathrm{FF}}^{2}-2 \times \overline{\mathrm{F}^{\prime} \mathrm{S}} \times \overline{\mathrm{FF}^{\prime}} \times \cos \theta$ $(2 a-1)^{2}=1^{2}+(2 c)^{2}-2 \times 1 \times 2 c \times \frac{c}{4}$

$4 a^{2}-4 a=3 c^{2}$

\begin{center}
\includegraphics[max width=\textwidth]{2024_07_25_adbb0085b5e6606b27f6g-11}
\end{center}

$\angle \mathrm{RFF}^{\prime}=\pi-\theta$ 이므로

삼각형 $\mathrm{RFF}^{\prime}$ 에서 코사인법칙에 의하여

\begin{center}
\includegraphics[max width=\textwidth]{2024_07_25_adbb0085b5e6606b27f6g-11(2)}
\end{center}

$\left(2 a-\frac{1}{2}\right)^{2}=\left(\frac{1}{2}\right)^{2}+(2 c)^{2}-2 \times \frac{1}{2} \times 2 c \times\left(-\frac{c}{4}\right)$

$4 a^{2}-2 a=\frac{9}{2} c^{2}$

(ㄱ), (ㄴ)에서

$\frac{3}{2} \times\left(4 a^{2}-4 a\right)=4 a^{2}-2 a$

$4 a^{2}=8 a$ 이므로 $a=2(\because a>0)$

따라서 타원 $C$ 의 장축의 길이는 $2 a=4$

$\overline{\mathrm{FR}}=\frac{1}{2}$ 을 구한 후 타원 $C$ 의 장축의 길이를 다음과 같이 구할 수도 있다.

직선 $\mathrm{F}^{\prime} \mathrm{Q}$ 가 타원과 만나는 점 중 S 가 아닌 점을 T 라 하면 두 평행선 $\mathrm{F}^{\prime} \mathrm{Q}, \mathrm{FP}$ 가 원점에 대하여 대칭이고 타원 $C$ 도 원점에 대하여 대칭이므로 두 선분 $\mathrm{F}^{\prime} \mathrm{T}, \mathrm{FR}$ 도 원점에 대하여 대칭이다.

곧, $\overline{\mathrm{F}^{\prime} \mathrm{T}}=\overline{\mathrm{FR}}=\frac{1}{2}$

또한 타원 $C$ 는 $y$ 축에 대하여 대칭이므로

$\overline{\mathrm{FQ}}=\overline{\mathrm{F}^{\prime} \mathrm{Q}}=4$

$\overline{\mathrm{FS}}=l$ 이라 하면 타원의 정의에 의하여

$\overline{\mathrm{FS}}+\overline{\mathrm{F}^{\prime} \mathrm{S}}=\overline{\mathrm{FT}}+\overline{\mathrm{F}^{\prime} \mathrm{T}}=$ (타원 $C$ 의 장축의 길이 $)$ 이므로

$l+1=\overline{\mathrm{FT}}+\frac{1}{2}$ 에서 $\overline{\mathrm{FT}}=l+\frac{1}{2}$

\begin{center}
\includegraphics[max width=\textwidth]{2024_07_25_adbb0085b5e6606b27f6g-11(5)}
\end{center}

두 삼각형 $\mathrm{FQS}, \mathrm{FQT}$ 에서 코사인법칙에 의하여

$\cos (\angle \mathrm{FQS})=\frac{\overline{\mathrm{FQ}}^{2}+\overline{\mathrm{SQ}}^{2}-\overline{\mathrm{FS}}^{2}}{2 \times \overline{\mathrm{FQ}} \times \overline{\mathrm{SQ}}}$

$\cos (\angle \mathrm{FQT})=\frac{\overline{\mathrm{FQ}}^{2}+\overline{\mathrm{TQ}}^{2}-\overline{\mathrm{FT}}^{2}}{2 \times \overline{\mathrm{FQ}} \times \overline{\mathrm{TQ}}}$

$\cos (\angle \mathrm{FQS})=\cos (\angle \mathrm{FQT})$ 이므로

$\frac{4^{2}+3^{2}-l^{2}}{2 \times 4 \times 3}=\frac{4^{2}+\left(\frac{9}{2}\right)^{2}-\left(l+\frac{1}{2}\right)^{2}}{2 \times 4 \times \frac{9}{2}}$ 에서

$(l-3)(l+1)=0$

$l=3(\because l>0)$

따라서 타원 $C$ 의 장축의 길이는

$l+1=4$

\begin{enumerate}
  \setcounter{enumi}{28}
  \item 조건 (가)에서 $\overrightarrow{\mathrm{OP}}=(1-s) \overrightarrow{\mathrm{OA}}+s \overrightarrow{\mathrm{OB}}$ 이고 $0 \leq s \leq 1$ 이므로 P 는 선분 AB 위의 점이다.
\end{enumerate}

조건 (나)에서

$\overrightarrow{\mathrm{OB}} \cdot \overrightarrow{\mathrm{OP}}=\overrightarrow{\mathrm{OB}} \cdot \overrightarrow{\mathrm{OQ}}$

$\overrightarrow{\mathrm{OB}} \cdot \overrightarrow{\mathrm{OP}}-\overrightarrow{\mathrm{OB}} \cdot \overrightarrow{\mathrm{OQ}}=0$

$\overrightarrow{\mathrm{OB}} \cdot(\overrightarrow{\mathrm{OP}}-\overrightarrow{\mathrm{OQ}})=0$

곧, $\overrightarrow{\mathrm{OB}} \cdot \overrightarrow{\mathrm{QP}}=0$ 이고 $\overrightarrow{\mathrm{OB}} \neq \overrightarrow{0}, \overrightarrow{\mathrm{QP}} \neq \overrightarrow{0}$ 이므로 두 벡터 $\overrightarrow{\mathrm{OB}}, \overrightarrow{\mathrm{PQ}}$ 는 서로 수직이다.

정리하면 점 P 는 선분 AB 위에 있고, 점 Q 는 점 P 를 지나고 직선 OB 에 수직인 직선 위에 있다.

또한 조건 (다)에서 $|\overrightarrow{\mathrm{PQ}}|=2 \sqrt{2}$ 이므로 가능한 점 Q 의 위치는 다음과 같이 $\mathrm{Q}_{1}$ 또는 $\mathrm{Q}_{2}$ 이다.

\begin{center}
\includegraphics[max width=\textwidth]{2024_07_25_adbb0085b5e6606b27f6g-11(3)}
\end{center}

좌표평면에서 세 점 $\mathrm{O}, \mathrm{A}, \mathrm{C}$ 의 좌표를 각각 $\mathrm{O}(0,0), \mathrm{A}(2,0), \mathrm{C}(0,2)$ 로 놓으면

조건 (가)에 의하여 $\overrightarrow{\mathrm{AP}}=s \overrightarrow{\mathrm{AB}}=(0,2 s)$ 이고 $\overrightarrow{\mathrm{PQ}_{1}}=(2,-2), \overrightarrow{\mathrm{PQ}_{2}}=(-2,2)$ 이므로

세 점 $\mathrm{P}, \mathrm{Q}_{1}, \mathrm{Q}_{2}$ 의 좌표는

$\mathrm{P}(2,2 s), \mathrm{Q}_{1}(4,2 s-2), \mathrm{Q}_{2}(0,2 s+2)$ 이다.

점 Q 가 점 $\mathrm{Q}_{1}(4,2 s-2)$ 이면

$\overrightarrow{\mathrm{OP}} \cdot \overrightarrow{\mathrm{OQ}_{1}}=2 \times 4+2 s \times(2 s-2)$

\[
\begin{aligned}
& =4\left(s-\frac{1}{2}\right)^{2}+7 \\
& \geq 7\left(\text { 등호는 } s=\frac{1}{2} \text { 일 때 성립 }\right)
\end{aligned}
\]

이므로 조건 (다)를 만족시키지 않는다.

따라서 점 Q 는 점 $\mathrm{Q}_{2}(0,2 s+2)$ 이다.\\
조건 (다)의 $\overrightarrow{\mathrm{OP}} \cdot \overrightarrow{\mathrm{OQ}_{2}}=3$ 에서

$2 \times 0+2 s \times(2 s+2)=3$

$4 s^{2}+4 s-3=0$

$(2 s-1)(2 s+3)=0$

곧, $s=\frac{1}{2}(\because 0 \leq s \leq 1)$

이때 $\mathrm{P}(2,1)$ 이므로

$t=\overrightarrow{\mathrm{OB}} \cdot \overrightarrow{\mathrm{OP}}$

$=2 \times 2+2 \times 1$

$=6$

따라서 $8(s+t)=8\left(\frac{1}{2}+6\right)=52$

\begin{enumerate}
  \setcounter{enumi}{29}
  \item 점 A 를 지나고 $z$ 축에 수직인 평면을 $\alpha$ 라 하자. 직선 AP 는 $z$ 축과 수직이므로 점 P 는 평면 $\alpha$ 위에 있다.
\end{enumerate}

또한 점 P 가 구 $S$ 위의 점이므로 점 P 는 평면 $\alpha$ 와 구 $S$ 가 만나서 생기는 원 위에 있다.

이 원을 $C$ 라 하자.

원 $C$ 는 점 $\mathrm{A}(3, a, 3)$ 을 중심으로 하고 반지름의 길이가 3 인 원이다.

한편, 평면 $\alpha$ 는 $x y$ 평면과 평행하므로

(평면 OAP 와 평면 $\alpha$ 가 이루는 각의 크기)

$=($ 평면 OAP 와 $x y$ 평면이 이루는 각의 크기)

$=\theta$

이때 평면 OAP 와 평면 $\alpha$ 의 교선은 직선 AP 이다. 평면 OAP 위의 점 O 에서 평면 $\alpha$ 에 내린 수선의 발을 H 라 하면 $\mathrm{H}(0,0,3)$ 이고 $\overline{\mathrm{OH}}=3$ 이다.

점 H 가 직선 AP 위의 점일 때,

(평면 OAP$) \perp($ 평면 $\alpha)$ 가 되어 $\sin \theta=1$ 이고, 이때의 점 P 는 $\mathrm{P}_{0}$ 이 아니다.

점 H 가 직선 AP 위의 점이 아닐 때, 점 H 에서 직선 AP 에 내린 수선의 발을 I 라 하면 삼수선의 정리에 의하여 $\overline{\mathrm{OI}} \perp$ (직선 AP )이므로

$\angle \mathrm{OIH}=\theta$

\begin{center}
\includegraphics[max width=\textwidth]{2024_07_25_adbb0085b5e6606b27f6g-11(6)}
\end{center}

직각삼각형 OHI 에서 $\tan \theta=\frac{\overline{\mathrm{OH}}}{\overline{\mathrm{HI}}}=\frac{3}{\overline{\mathrm{HI}}}$

$\sin \theta$ 의 최솟값이 $\frac{\sqrt{21}}{7}$ 이므로

$\tan \theta$ 의 최솟값은 $\frac{\sqrt{3}}{2}$ 이다.

곧, $\tan \theta=\frac{3}{\overline{\mathrm{HI}}} \geq \frac{\sqrt{3}}{2}$ 에서 $\overline{\mathrm{HI}} \leq 2 \sqrt{3}$

그런데 선분 HI 의 길이, 곧 점 H 와 직선 AP 사이의 거리가 최대인 경우는 점 I 가 점 A 와 같을 때이다. 따라서 $\overline{\mathrm{HA}}=2 \sqrt{3}$ 이고,

점 I 가 점 A 와 같을 때 $\overline{\mathrm{HA}} \perp \overline{\mathrm{AP}_{0}}$ 이다.

이때 점 $\mathrm{P}_{0}$ 의 위치는 다음 그림과 같다.

\begin{center}
\includegraphics[max width=\textwidth]{2024_07_25_adbb0085b5e6606b27f6g-11(4)}
\end{center}

수학 영역

\[
\begin{aligned}
& \text { 직각삼각형 } \mathrm{OHA} \text { 에서 } \\
& \overline{\mathrm{OA}}^{2}=\overline{\mathrm{OH}}^{2}+\overline{\mathrm{HA}}^{2} \\
&=3^{2}+(2 \sqrt{3})^{2} \\
&=21
\end{aligned}
\]

삼수선의 정리에 의하여 $\overline{\mathrm{OA}} \perp \overline{\mathrm{AP}_{0}}$ 이므로 직각삼각형 $\mathrm{OAP}_{0}$ 에서

$\overrightarrow{\mathrm{OP}_{0}}=\sqrt{\overline{\mathrm{OA}}^{2}+{\overline{\mathrm{AP}_{0}}}^{2}}$

$=\sqrt{21+3^{2}}$

$=\sqrt{30}$

따라서 ${\overline{\mathrm{OP}_{0}}}^{2}=30$


\end{document}