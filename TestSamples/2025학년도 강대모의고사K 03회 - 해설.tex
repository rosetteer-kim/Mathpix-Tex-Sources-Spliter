% This LaTeX document needs to be compiled with XeLaTeX.
\documentclass[10pt]{article}
\usepackage[utf8]{inputenc}
\usepackage{amsmath}
\usepackage{amsfonts}
\usepackage{amssymb}
\usepackage[version=4]{mhchem}
\usepackage{stmaryrd}
\usepackage{graphicx}
\usepackage[export]{adjustbox}
\graphicspath{ {./images/} }
\usepackage[fallback]{xeCJK}
\usepackage{polyglossia}
\usepackage{fontspec}
\setCJKmainfont{Noto Serif CJK KR}

\setmainlanguage{english}
\setmainfont{CMU Serif}

\begin{document}
\section*{- 수학 영역 $\cdot$}
\section*{공통과목}
\begin{center}
\includegraphics[max width=\textwidth]{2024_08_01_b4f29a259b9b99d8349ag-1(1)}
\end{center}

해설

\begin{enumerate}
  \item $4^{\sqrt{2}} \times\left(2^{1-\sqrt{2}}\right)^{2}=4^{\sqrt{2}} \times\left(2^{2}\right)^{1-\sqrt{2}}$
\end{enumerate}

$=4^{\sqrt{2}+(1-\sqrt{2})}$

$=4$

\begin{enumerate}
  \setcounter{enumi}{1}
  \item $f(x)=\frac{1}{3} x^{3}-\frac{1}{2} x^{2}+3 x-2$ 에서 $f^{\prime}(x)=x^{2}-x+3$ 이므로 $f^{\prime}(2)=5$

  \item $\cos ^{2}\left(\frac{\pi}{2}+\theta\right)=(-\sin \theta)^{2}=\frac{1}{9}$ 이고

\end{enumerate}

$\pi<\theta<2 \pi$ 에서 $\sin \theta<0$ 이므로

$\sin \theta=-\frac{1}{3}$

또한 $\cos ^{2} \theta=1-\sin ^{2} \theta=\frac{8}{9}$ 이므로

$\cos ^{2} \theta-\sin \theta=\frac{8}{9}-\left(-\frac{1}{3}\right)=\frac{11}{9}$

\begin{enumerate}
  \setcounter{enumi}{3}
  \item $\lim _{x \rightarrow 1+} f(x)+\lim _{x \rightarrow-1-} f(x)=1+2=3$

  \item 등차수열 $\left\{a_{n}\right\}$ 의 공차를 $d(d \neq 0)$ 이라 하면 $a_{4}=5$ 이므로

\end{enumerate}

$a_{1}=a_{4}-3 d=5-3 d$,

$a_{2}=a_{4}-2 d=5-2 d$,

$a_{3}=a_{4}-d=5-d$

이때 $a_{1} a_{2}=a_{3}^{2}$ 에서

$(5-3 d)(5-2 d)=(5-d)^{2}$

$5 d^{2}-15 d=0$

$d \neq 0$ 이므로 $d=3$

따라서 $a_{5}=a_{4}+d=5+3=8$

\begin{enumerate}
  \setcounter{enumi}{5}
  \item $f^{\prime}(x)=3 x^{2}+f(2) x+f(1)$ 의 양변을 적분하면 $f(x)=x^{3}+\frac{f(2)}{2} x^{2}+f(1) x+C(C$ 는 적분상수 $)$ 이때 $f(0)=0$ 이므로 $C=0$
\end{enumerate}

곧, $f(x)=x^{3}+\frac{f(2)}{2} x^{2}+f(1) x$

위의 식의 양변에 $x=1$ 을 대입하면

$f(1)=1+\frac{f(2)}{2}+f(1)$ 이므로

$f(2)=-2$

곧, $f(x)=x^{3}-x^{2}+f(1) x$

위의 식의 양변에 $x=2$ 를 대입하면

$f(2)=2^{3}-2^{2}+2 f(1)$ 이고

$f(2)=-2$ 이므로 $-2=2^{3}-2^{2}+2 f(1)$ 에서 $f(1)=-3$

곧, $f(x)=x^{3}-x^{2}-3 x$

따라서 $f(-1)=(-1)^{3}-(-1)^{2}-3 \times(-1)=1$

\begin{enumerate}
  \setcounter{enumi}{6}
  \item 곡선 $y=3 x^{2}+k$ 가 곡선 $y=x^{3}-3 x^{2}$ 과 서로 다른 세 점에서 만나려면 삼차방정식 $3 x^{2}+k=x^{3}-3 x^{2}$, 곧 $x^{3}-6 x^{2}=k$ 가 서로 다른 세 실근을 가져야 하고, 곡선 $y=x^{3}-6 x^{2}$ 과 직선 $y=k$ 가 서로 다른 세 점 에서 만나야 한다.
\end{enumerate}

이매 $f(x)=x^{3}-6 x^{2}$ 이라 하면

$f^{\prime}(x)=3 x^{2}-12 x=3 x(x-4)$ 이므로

삼차함수 $f(x)$ 는

$x=0$ 에서 극댓값 0 을 갖고

$x=4$ 에서 극솟값 -32 를 갖는다.

따라서 직선 $y=k$ 가 곡선 $y=f(x)$ 와 서로 다른

세 점에서 만나려면 $-32<k<0$ 이어야 한다.

곧, 정수 $k$ 는 $-31,-30, \cdots,-1$ 이고,

그 개수는 31 이다.

\begin{enumerate}
  \setcounter{enumi}{7}
  \item 원점을 O 라 하면 직선 OA 의 기울기는 1 이고 직선 AC 의 기울기는 -1 이므로
\end{enumerate}

$\angle \mathrm{OAC}=\frac{\pi}{2}$

또한 $\angle \mathrm{AOC}=\frac{\pi}{4}$ 이므로 삼각형 OAC 는 직각이등변 삼각형이다.

한편, 점 A 는 직선 $y=x$ 위에 있으므로

$\mathrm{A}(p, p)(p>0)$ 으로 놓을 수 있다.

이때 (ㄱ)에 의하여 $\mathrm{C}(0,2 p)$

점 B 는 점 C 와 $y$ 좌표가 같고 직선 $y=x$ 위에

있으므로 $\mathrm{B}(2 p, 2 p)$

\begin{center}
\includegraphics[max width=\textwidth]{2024_08_01_b4f29a259b9b99d8349ag-1(3)}
\end{center}

두 점 $\mathrm{A}, \mathrm{B}$ 는 모두 곡선 $y=\log _{a} x$ 위의 점이므로

$\log _{a} p=p$

$\log _{a} 2 p=2 p$

(ㄷ)에서 (ㄴ)을 변끼리 빼면

$\log _{a} 2 p-\log _{a} p=2 p-p$

$\log _{a} 2=p$

따라서 $\log _{a} p=\log _{a} 2$ 에서 $p=2$ 이고

$\log _{a} 2=2$

곧, $a^{2}=2$

$a>1$ 이므로 $a=\sqrt{2}$

\begin{enumerate}
  \setcounter{enumi}{8}
  \item 두 곡선 $y=x^{3}-x^{2}, y=a\left(x^{2}-x\right)$ 의 교점을 찾아 보자.
\end{enumerate}

방정식 $x^{3}-x^{2}=a\left(x^{2}-x\right)$ 에서

$\left(x^{3}-x^{2}\right)-a\left(x^{2}-x\right)=0$

$x(x-a)(x-1)=0$

곧, $x=0, x=a, x=1$

$0<a<1$ 이므로

$0 \leq x \leq a$ 일 때 $\left(x^{3}-x^{2}\right)-a\left(x^{2}-x\right) \geq 0$ 이고,

$a \leq x \leq 1$ 일 때 $\left(x^{3}-x^{2}\right)-a\left(x^{2}-x\right) \leq 0$

\begin{center}
\includegraphics[max width=\textwidth]{2024_08_01_b4f29a259b9b99d8349ag-1}
\end{center}

따라서

$A=\int_{0}^{a}\left\{\left(x^{3}-x^{2}\right)-a\left(x^{2}-x\right)\right\} d x$,

$B=-\int_{a}^{1}\left\{\left(x^{3}-x^{2}\right)-a\left(x^{2}-x\right)\right\} d x$ 이므로

$A-B=\int_{0}^{a}\left\{x^{3}-(a+1) x^{2}+a x\right\} d x$

$+\int_{a}^{1}\left\{x^{3}-(a+1) x^{2}+a x\right\} d x$

$=\int_{0}^{1}\left\{x^{3}-(a+1) x^{2}+a x\right\} d x$

$=\left[\frac{1}{4} x^{4}-\frac{a+1}{3} x^{3}+\frac{a}{2} x^{2}\right]_{0}^{1}$

$=\frac{1}{4}-\frac{a+1}{3}+\frac{a}{2}$

$=\frac{a}{6}-\frac{1}{12}$

\begin{center}
\includegraphics[max width=\textwidth]{2024_08_01_b4f29a259b9b99d8349ag-1(4)}
\end{center}

$A-B=\frac{1}{36}$ 이므로

$\frac{a}{6}-\frac{1}{12}=\frac{1}{36}$ 에서 $a=\frac{2}{3}$

\begin{enumerate}
  \setcounter{enumi}{9}
  \item 조건 (가)의 $\frac{a_{n+2}}{a_{n}}=\frac{a_{n+1}}{a_{n+3}}$ 에서
\end{enumerate}

$a_{n} a_{n+1}=a_{n+2} a_{n+3}$ 이므로

$a_{1} a_{2}=a_{3} a_{4}=a_{5} a_{6}=\cdots$ 이고,

$a_{2} a_{3}=a_{4} a_{5}=a_{6} a_{7}=\cdots$

$a_{2}=a$ 라 하면 $a_{1}=1, a_{3}=2$ 이므로

위의 규칙에 따라 수열 $\left\{a_{n}\right\}$ 을 나열하면 다음과 같다.

\begin{center}
\includegraphics[max width=\textwidth]{2024_08_01_b4f29a259b9b99d8349ag-1(2)}
\end{center}

곧, $a_{7}=8, a_{8}=\frac{a}{8}$

$a_{7}=a_{8}$ 에서 $8=\frac{a}{8}$, 곧 $a=64$

따라서 $a_{2}=64$

\section*{[다른 풀이]}
조건 (가)에서 모든 자연수 $n$ 에 대하여

$\frac{a_{n+2}}{a_{n}}=\frac{a_{n+1}}{a_{n+3}}$

이므로 $n$ 에 $n+1$ 을 대입하면

$\frac{a_{n+3}}{a_{n+1}}=\frac{a_{n+2}}{a_{n+4}}$

곧, $\frac{a_{n+1}}{a_{n+3}}=\frac{a_{n+4}}{a_{n+2}}$

(ㄱ), (ㄴ)에서 $\frac{a_{n+2}}{a_{n}}=\frac{a_{n+4}}{a_{n+2}}$

따라서 두 수열 $\left\{a_{2 n-1}\right\},\left\{a_{2 n}\right\}$ 은 모두 등비수열이다.

조건 (나)에서 $a_{1}=1, a_{3}=2$ 이므로

등비수열 $\left\{a_{2 n-1}\right\}$ 의 공비는 2 이다.

곧, $a_{7}=a_{1} \times 2^{3}=2^{3}$

조건 (나)에서 $a_{8}=a_{7}=2^{3}$

(ㄱ)에 $n=1$ 을 대입하면

$\frac{a_{3}}{a_{1}}=\frac{a_{2}}{a_{4}}$, 곧 $\frac{a_{4}}{a_{2}}=\frac{1}{2}$ 이므로

등비수열 $\left\{a_{2 n}\right\}$ 의 공비는 $\frac{1}{2}$ 이다.

따라서

$a_{2}=a_{8} \times 2^{3}$

$=2^{3} \times 2^{3}$

$=64$

\begin{enumerate}
  \setcounter{enumi}{10}
  \item $f(x)=a x^{3}+b x^{2}$ 에서
\end{enumerate}

$f^{\prime}(x)=3 a x^{2}+2 b x$

$f(-1)=f^{\prime}(-1)$ 에서

$-a+b=3 a-2 b$, 곧 $b=\frac{4}{3} a$

따라서 $f(x)=a x^{2}\left(x+\frac{4}{3}\right)$

곡선 $y=f(x)$ 위의 점 $\mathrm{A}(-1, f(-1))$ 에서의 접선이

곡선 $y=f(x)$ 와 A 가 아닌 점 B 에서 만나므로

(선분 AB 의 기울기) $=f^{\prime}(-1)$

곧, 점 B 의 $x$ 좌표를 $t(t \neq-1)$ 이라 하면

$\frac{f(t)-f(-1)}{t-(-1)}=f^{\prime}(-1)$

(ㄱ)에서 $\frac{a t^{2}\left(t+\frac{4}{3}\right)-a \times \frac{1}{3}}{t+1}=\frac{1}{3} a$

$\frac{t^{2}\left(t+\frac{4}{3}\right)-\frac{1}{3}}{t+1}=\frac{1}{3}(\because a>0)$

$t^{2}\left(t+\frac{4}{3}\right)-\frac{1}{3}=\frac{1}{3}(t+1)$

$\left(t^{2}+2 t+1\right)(3 t-2)=0$

$t=\frac{2}{3}(\because t \neq-1)$

따라서 점 B 의 좌표는 $\left(\frac{2}{3}, f\left(\frac{2}{3}\right)\right)$ 이다.

두 직선 $\mathrm{OA}, \mathrm{OB}$ 가 수직이므로

$($ 직선 $O A$ 의 기울기 $) \times($ 직선 $O B$ 의 기울기 $)=-1$

$\frac{f(-1)-0}{-1-0} \times \frac{f\left(\frac{2}{3}\right)-0}{\frac{2}{3}-0}=-1$

$\left(-\frac{1}{3} a\right) \times \frac{4}{3} a=-1$

$a^{2}=\frac{9}{4}$

$a=\frac{3}{2}(\because a>0)$

따라서 $f(x)=\frac{3}{2} x^{2}\left(x+\frac{4}{3}\right)$ 이므로

$f(1)=\frac{7}{2}$

\begin{enumerate}
  \setcounter{enumi}{11}
  \item 주어진 수열 $\left\{a_{n}\right\}$ 의 모든 항은 자연수이다.
\end{enumerate}

부등식

$\log _{3}\left(x-a_{k}\right)<\log _{9} x$

...... (7)

에서 로그의 진수 조건은

$x-a_{k}>0$ 이고 $x>0$, 곧 $x>a$

부등식 (ㄱ)에서

$\log _{9}\left(x-a_{k}\right)^{2}<\log _{9} x$ $\left(x-a_{k}\right)^{2}<x$

따라서 $f(x)=\left(x-a_{k}\right)^{2}-x\left(x>a_{k}\right)$ 라 하면 부등식 (ㄱ)은 부등식 $f(x)<0\left(x>a_{k}\right)$ 와 같다. ….. (ㄴ)

\begin{center}
\includegraphics[max width=\textwidth]{2024_08_01_b4f29a259b9b99d8349ag-2}
\end{center}

$f\left(a_{k}\right)=-a_{k}<0$ 이므로

(ㄴ)을 만족시키는 자연수 $x$ 의 개수가 1 이려면 $f\left(a_{k}+1\right)=1-\left(a_{k}+1\right)<0$, 곧 $a_{k}>0$ 이고,

$f\left(a_{k}+2\right)=4-\left(a_{k}+2\right) \geq 0$, 곧 $a_{k} \leq 2$ 이므로

$a_{k}$ 의 값으로 가능한 것은 1,2 이다.

( i ) $a_{k}=1$ 일 때

$a_{k}=\log _{3} k$ 에서 $k=3^{1}=3$

$a_{k}=k$ 에서 $k=1$

(ii) $a_{k}=2$ 일 때

$a_{k}=\log _{3} k$ 에서 $k=3^{2}=9$,

$a_{k}=k$ 에서 $k=2$

(i), (ii)에 의하여 구하는 모든 자연수 $k$ 의 값의 합은

$(3+1)+(9+2)=15$

\section*{줄죠}
수열 $\left\{a_{n}\right\}$ 의 항의 값을 표로 나타내면 다음과 같다.

\begin{center}
\begin{tabular}{|c|c|c|c|c|c|c|c|c|c|c|}
\hline
$n$ & 1 & 2 & 3 & 4 & 5 & 6 & 7 & 8 & 9 & $\cdots$ \\
\hline
$a_{n}$ & 1 & 2 & 1 & 4 & 5 & 6 & 7 & 8 & 2 & $\cdots$ \\
\hline
\end{tabular}
\end{center}

\begin{enumerate}
  \setcounter{enumi}{12}
  \item $\lim _{x \rightarrow t-} f(x)-\lim _{x \rightarrow t+} f(x)=(t-1)(t-2)^{2}$
\end{enumerate}

이 방정식의 실근은 다음과 같이 두 가지의 경우가 있다.

( i ) $t \neq a$ 인 경우

함수 $f(x)$ 는 $x=t$ 에서 연속이므로 방정식 (ㄱ)은 방정식 $0=(t-1)(t-2)^{2}$ 과 같다.

이 방정식의 실근은 $t=1$ 또는 $t=2$ 이다.

이때 1 과 2 중에서 $a$ 와 값이 다른 것만 방정식 (ㄱ)의 실근이다.

따라서 실수 $a$ 에 대하여 이 경우의 방정식 (ㄱ)의 서로 다른 실근의 개수를 $g_{1}(a)$ 라 하면

$g_{1}(a)= \begin{cases}2 & (a \neq 1, a \neq 2) \\ 1 & (a=1 \text { 또는 } a=2\end{cases}$

(ii) $t=a$ 인 경우

$\lim _{x \rightarrow t-} f(x)=t^{2}, \lim _{x \rightarrow t+} f(x)=t$ 이므로

방정식 (ㄱ)은 방정식 $t^{2}-t=(t-1)(t-2)^{2}$, 곧 $(t-1)^{2}(t-4)=0$ 과 같다.

이 방정식의 실근은 $t=1$ 또는 $t=4$ 이다

이때 1 과 4 중에서 $a$ 와 값이 같은 것만 방정식 (ㄱ)의 실근이다.

따라서 실수 $a$ 에 대하여 이 경우의 방정식 (ㄱ)의 서로 다른 실근의 개수를 $g_{2}(a)$ 라 하면

$g_{2}(a)= \begin{cases}0 & (a \neq 1, a \neq 4) \\ 1 & (a=1 \text { 또는 } a=4\end{cases}$

(i), (ii)에 의하여 방정식 (ㄱ)의 서로 다른 실근의 개수는

$g_{1}(a)+g_{2}(a)= \begin{cases}2 & (a \neq 1, a \neq 2, a \neq 4) \\ 2 & (a=1) \\ 1 & (a=2) \\ 3 & (a=4)\end{cases}$\\
따라서 $g_{1}(a)+g_{2}(a)$ 의 값이 홀수가 되도록 하는 모든 상수 $a$ 의 값의 합은

$2+4=6$

\section*{루료}
$h(t)=\lim _{x \rightarrow t-} f(x)-\lim _{x \rightarrow t+} f(x)$ 라 하면

$h(t)=\left\{\begin{array}{ll}0 & (t \neq a) \\ t^{2}-t & (t=a)\end{array}\right.$ 이고,

함수 $y=h(t)$ 의 그래프는 다음과 같다.

\begin{center}
\includegraphics[max width=\textwidth]{2024_08_01_b4f29a259b9b99d8349ag-2(2)}
\end{center}

이때 방정식 (ㄱ)의 서로 다른 실근의 개수는 두 함수 $y=h(t), y=(t-1)(t-2)^{2}$ 의 그래프의 교점의 개수 와 같다.

다음 그림은 $a=2$ 인 경우 두 함수 $y=h(t)$

$y=(t-1)(t-2)^{2}$ 의 그래프의 교점을 ■로 표시한 것이다.

\begin{center}
\includegraphics[max width=\textwidth]{2024_08_01_b4f29a259b9b99d8349ag-2(4)}
\end{center}

다음 그림은 $a=4$ 인 경우 두 함수 $y=h(t)$

$y=(t-1)(t-2)^{2}$ 의 그래프의 교점을 ■로 표시한 것이다.

\begin{center}
\includegraphics[max width=\textwidth]{2024_08_01_b4f29a259b9b99d8349ag-2(3)}
\end{center}

\begin{enumerate}
  \setcounter{enumi}{13}
  \item 조건 (가)에 의하여 $f(0)=0$ 이고 삼차함수 $f(x)$ 의 그래프는 다음 그림에서 색칠된 부분에 존재한다.
\end{enumerate}

\begin{center}
\includegraphics[max width=\textwidth]{2024_08_01_b4f29a259b9b99d8349ag-2(1)}
\end{center}

따라서 삼차함수 $f(x)$ 의 최고차항의 계수는 양수이므로

' $f(k)=1$ 이고 함수 $f(x)$ 는 $x=k$ 의 좌우에서 증가 한다.

를 만족시키는 실수 $k(0<k \leq 1)$ 이 존재한다

함수 $g(x)=\left\{\begin{array}{ll}\frac{f(x)}{x} & (f(x)>1) \\ x & (f(x) \leq 1)\end{array}\right.$ 이

$x=k$ 에서 연속이므로

$\frac{f(k)}{k}=k$, 곧 $f(k)=k^{2}$

그런데 (ㄱ)에 의하여 $f(k)=1$ 이므로

$k^{2}=1$ 에서 $k=1(\because k>0)$

따라서 삼차함수 $f(x)$ 의 그래프는 다음 그림과 같이 점 $(0,0)$ 을 지나고 점 $(1,1)$ 에서 직선 $y=x$ 와 접한다.

\begin{center}
\includegraphics[max width=\textwidth]{2024_08_01_b4f29a259b9b99d8349ag-3(3)}
\end{center}

$f(x)=a x(x-1)^{2}+x(a$ 는 상수, $a>0)$ 으로

놓을 수 있고 (ㄱ)을 만족시키는 $k$ 가 1 뿐이므로 $x \leq 1$ 일 때 $f(x) \leq 1$

곧, $a x(x-1)^{2}+x \leq 1$

정리하면 $(x-1)\{a x(x-1)+1\} \leq 0$

따라서 $x \leq 1$ 일 때 $a x(x-1)+1 \geq 0$

함수 $y=a x(x-1)+1(x \leq 1)$, 곧

$y=a\left(x-\frac{1}{2}\right)^{2}-\frac{a}{4}+1(x \leq 1)$ 은 최솟값 $-\frac{a}{4}+1$ 을

가지므로 $-\frac{a}{4}+1 \geq 0$ 에서

$a \leq 4$

따라서 $0<a \leq 4$

이상에서 조건을 만족시키는 함수 $f(x)$ 는

$f(x)=a x(x-1)^{2}+x$ ( $a$ 는 상수, $\left.0<a \leq 4\right)$ 이고 $f(3)=12 a+3 \leq 51$ 이므로

$f(3)$ 의 최댓값은 51 이다.

\begin{enumerate}
  \setcounter{enumi}{14}
  \item 함수 $f(x)=a \cos \left(a \pi x+\frac{\pi}{b}\right)-b(a, b$ 는 자연수 $)$ 의 최댓값은 $a-b$ 이므로 주어진 조건에 의하여
\end{enumerate}

$a-b=3 a k$, 곧 $a=3 a k+b$

방정식 $f(x)=f(0)$ 에서

$a \cos \left(a \pi x+\frac{\pi}{b}\right)-b=a \cos \frac{\pi}{b}-b$

$a$ 는 자연수이므로

$\cos \left(a \pi x+\frac{\pi}{b}\right)=\cos \frac{\pi}{b}$

이때 $t=a \pi x+\frac{\pi}{b}$ 라 하면

$\cos t=\cos \frac{\pi}{b}$

$x>0$ 에서 방정식 $\cos \left(a \pi x+\frac{\pi}{b}\right)=\cos \frac{\pi}{b}$ 의 실근 중 가장 작은 것이 $k$ 이므로 $t>\frac{\pi}{b}$ 에서 방정식 $\cos t=\cos \frac{\pi}{b}$ 의 실근 중 가장 작은 것이 $a \pi k+\frac{\pi}{b}$ 이다.

곧, $t>\frac{\pi}{b}$ 에서 곡선 $y=\cos t$ 와 직선 $y=\cos \frac{\pi}{b}$ 의 교점의 $t$ 좌표 중 가장 작은 것이 $a \pi k+\frac{\pi}{b}$ 이다. 자연수 $b$ 의 값에 따라 경우를 나누어 살펴보자.\\
( i ) $b=1$ 인 경우

이 경우 $\cos \frac{\pi}{b}=-1$ 이고

$a \pi k+\frac{\pi}{b}=a \pi k+\pi$

$t>\pi$ 에서 곡선 $y=\cos t$ 와 직선 $y=\cos \frac{\pi}{b}$,

곧 $y=-1$ 의 교점의 $t$ 좌표 중 가장 작은 것이 $a \pi k+\pi$ 이다.

그런데 곡선 $y=\cos t$ 와 직선 $y=-1$ 은 $t=\pi$ 에서 만나고 함수 $y=\cos t$ 의 주기가 $2 \pi$ 이므로 $t>\pi$ 에서 곡선 $y=\cos t$ 와 직선 $y=-1$ 의 교점의 $t$ 좌표 중 가장 작은 것은 $\pi+2 \pi=3 \pi$ 이다.

\begin{center}
\includegraphics[max width=\textwidth]{2024_08_01_b4f29a259b9b99d8349ag-3(2)}
\end{center}

$a \pi k+\pi=3 \pi$ 에서 $a k=2$

(ㄱ)에서

$a=3 a k+b$

$=3 \times 2+1$

$=7$

(ii) $b \geq 2$ 인 경우

이 경우 $0<\frac{\pi}{b} \leq \frac{\pi}{2}$ 이므로

$0 \leq \cos \frac{\pi}{b}<1$

$t>\frac{\pi}{b}$ 에서 곡선 $y=\cos t$ 와 직선 $y=\cos \frac{\pi}{b}$ 의

교점의 $t$ 좌표 중 가장 작은 것은 $a \pi k+\frac{\pi}{b}$ 이다.

그런데 곡선 $y=\cos t$ 와 직선 $y=\cos \frac{\pi}{b}$ 는 $t=\frac{\pi}{b}$

에서 만나고 곡선 $y=\cos t$ 는 직선 $t=\pi$ 에 대하여 대칭이므로

$\frac{\frac{\pi}{b}+\left(a \pi k+\frac{\pi}{b}\right)}{2}=\pi$, 곧 $a k=2-\frac{2}{b}$

\begin{center}
\includegraphics[max width=\textwidth]{2024_08_01_b4f29a259b9b99d8349ag-3}
\end{center}

(ㄱ)에서

$a=3 a k+b$

$=3\left(2-\frac{2}{b}\right)+b$

$=6-\frac{6}{b}+b$

$a, b(b \geq 2)$ 가 자연수이므로 가능한 자연수 $b$ 의

값은 $2,3,6$ 뿐이고 이때 $a$ 는 각각 $5,7,11$ 이다.

(i), (ii)에 의하여 순서쌍 $(a, b)$ 는

$(7,1),(5,2),(7,3),(11,6)$ 이다.

따라서 $M=11 \times 6=66, m=7 \times 1=7$ 이므로

$M-m=59$

\begin{enumerate}
  \setcounter{enumi}{15}
  \item 방정식 $4^{x-15}=\left(\frac{1}{8}\right)^{x}$ 에서
\end{enumerate}

$2^{2 x-30}=2^{-3 x}$

$2 x-30=-3 x$

따라서 $x=6$

\begin{enumerate}
  \setcounter{enumi}{16}
  \item 시각 $t=0$ 에서 $t=3$ 까지 점 P 가 움직인 거리는 $\int_{0}^{3}|v(t)| d t$ 이다. $v(t)=6 t(t-1)$ 이므로
\end{enumerate}

$0 \leq t \leq 1$ 일 때 $v(t) \leq 0$ 이고,

$t \geq 1$ 일 때 $v(t) \geq 0$ 이다.

$\int_{0}^{3}|v(t)| d t$

$=\int_{0}^{1}|v(t)| d t+\int_{1}^{3}|v(t)| d t$

$=-\int_{0}^{1} v(t) d t+\int_{1}^{3} v(t) d t$

$=-\left[2 t^{3}-3 t^{2}\right]_{0}^{1}+\left[2 t^{3}-3 t^{2}\right]_{1}^{3}$

$=-(-1)+28$

$=29$

\begin{enumerate}
  \setcounter{enumi}{17}
  \item $\sum_{k=1}^{5}\left(2 a_{k}+3 k\right)=45$ 에서
\end{enumerate}

$2 \sum_{k=1}^{5} a_{k}+3 \sum_{k=1}^{5} k=45$

$2 \sum_{k=1}^{5} a_{k}+3 \times \frac{5 \times 6}{2}=45$

곧, $\sum_{k=1}^{5} a_{k}=0$

$\sum_{k=1}^{6}\left(a_{k}-k^{2}\right)=0$ 에서

$\sum_{k=1}^{6} a_{k}-\sum_{k=1}^{6} k^{2}=0$

$\sum_{k=1}^{6} a_{k}-\frac{6 \times 7 \times 13}{6}=0$

곧, $\sum_{k=1}^{6} a_{k}=91$

따라서

$a_{6}=\sum_{k=1}^{6} a_{k}-\sum_{k=1}^{5} a_{k}$

$=91-0$

$=91$

\begin{enumerate}
  \setcounter{enumi}{18}
  \item 만약 $f(0) \neq 0$ 이면
\end{enumerate}

$\lim _{x \rightarrow 0} \frac{\{f(x)\}^{2}+x^{2}}{f(x)+x}=\frac{\{f(0)\}^{2}+0}{f(0)+0}=f(0)$ 이 되어

주어진 조건을 만족시키지 않는다.

따라서 $f(0)=0$ 이므로

$f(x)=x(x-k)(k$ 는 상수)

로 놓을 수 있다.

$\lim _{x \rightarrow 0} \frac{\{f(x)\}^{2}+x^{2}}{f(x)+x}$

$=\lim _{x \rightarrow 0} \frac{x^{2}\left\{(x-k)^{2}+1\right\}}{x(x-k+1)}$

$=\lim _{x \rightarrow 0} \frac{x\left\{(x-k)^{2}+1\right\}}{x-k+1}$

$=a$

이때 (극한값) $=a \neq 0$ 이고

$x \rightarrow 0$ 일 때 (분자) $\rightarrow 0$ 이므로 (분모) $\rightarrow 0$

곧, $k=1$

따라서 $f(x)=x(x-1)$

\[
\begin{aligned}
a & =\lim _{x \rightarrow 0} \frac{x\left\{(x-k)^{2}+1\right\}}{x-k+1} \\
& =\lim _{x \rightarrow 0}\left\{(x-1)^{2}+1\right\} \\
& =2 \\
& f(2 a)=f(4)=12
\end{aligned}
\]

\begin{center}
\includegraphics[max width=\textwidth]{2024_08_01_b4f29a259b9b99d8349ag-3(1)}
\end{center}

\section*{수학 영역}
\begin{enumerate}
  \setcounter{enumi}{19}
  \item $\overline{\mathrm{AB}}=\overline{\mathrm{CD}}=a, \overline{\mathrm{AC}}=b, \angle \mathrm{ACD}=\theta$ 라 하자.
\end{enumerate}

\begin{center}
\includegraphics[max width=\textwidth]{2024_08_01_b4f29a259b9b99d8349ag-4}
\end{center}

삼각형 ACD 의 넓이는

$\frac{1}{2} \times \overline{\mathrm{CD}} \times \overline{\mathrm{AC}} \times \sin (\angle \mathrm{ACD})=\frac{1}{2} a b \sin \theta$ 이고, 주어진 삼각형 ACD 의 넓이가 8 이므로 $\frac{1}{2} a b \sin \theta=8$ 에서

$\sin \theta=\frac{16}{a b}$

직각삼각형 ABC 에서

$\overline{\mathrm{AB}}^{2}+\overline{\mathrm{AC}}^{2}=\overline{\mathrm{BC}}^{2}$

곧, $a^{2}+b^{2}=8^{2}$ 이므로

삼각형 ACD 에서 코사인법칙에 의하여

\begin{center}
\includegraphics[max width=\textwidth]{2024_08_01_b4f29a259b9b99d8349ag-4(2)}
\end{center}

\[
=\frac{a^{2}+b^{2}-10^{2}}{2 a b}
\]

\[
\begin{aligned}
& =\frac{8^{2}-10^{2}}{2 a b} \\
& =\frac{-18}{a b}
\end{aligned}
\]

$\sin ^{2} \theta+\cos ^{2} \theta=1$ 이므로

$\left(\frac{16}{a b}\right)^{2}+\left(\frac{-18}{a b}\right)^{2}=1$ 에서

$a^{2} b^{2}=580$, 곧 $a b=2 \sqrt{145}$

따라서 삼각형 ABC 의 넓이는

$\frac{1}{2} a b=\frac{1}{2} \times 2 \sqrt{145}=\sqrt{145}$

이상에서 $p=16, q=-18, r=\sqrt{145}$ 이므로

$p+q+r^{2}=143$

\begin{enumerate}
  \setcounter{enumi}{20}
  \item 공차가 2 인 등차수열 $\left\{a_{n}\right\}$ 의 첫째항부터 제 $n$ 항 까지의 합은
\end{enumerate}

\[
S_{n}=\frac{n\left\{2 a_{1}+(n-1) \times 2\right\}}{2}
\]

\[
=n\left(n+a_{1}-1\right)
\]

따라서 $S_{2}, S_{4}, S_{6}, S_{8}, S_{10}$ 은 자연수 $a_{1}$ 의 값에 관계 없이 짝수이고

$S_{1}, S_{3}, S_{5}, S_{7}, S_{9}$ 는 $a_{1}$ 이 짝수일 때 짝수, $a_{1}$ 이 홀수일 때 홀수이다.

(i) $a_{1}$ 이 짝수일 때

모든 $S_{n}$ 이 짝수이므로

$\sum_{k=1}^{10}\left\{(-1)^{S_{k}} \times S_{k}\right\}$

$=\sum_{k=1}^{10} S_{k}$

$=\sum_{k=1}^{10}\left\{k^{2}+\left(a_{1}-1\right) k\right\}$

$=\sum_{k=1}^{10} k^{2}+\left(a_{1}-1\right) \sum_{k=1}^{10} k$

$=\frac{10 \times 11 \times 21}{6}+\left(a_{1}-1\right) \times \frac{10 \times 11}{2}$

$=55 a_{1}+330$

$\leq 550$

따라서 $a_{1} \leq 4$ 인 짝수 $a_{1}$ 의 개수는 2 이다

(ii) $a_{1}$ 이 홀수일 때

$S_{1}, S_{3}, S_{5}, S_{7}, S_{9}$ 는 홀수이고

$S_{2}, S_{4}, S_{6}, S_{8}, S_{10}$ 은 짝수이므로

\[
\begin{aligned}
& \sum_{k=1}^{10}\left\{(-1)^{S_{k}} \times S_{k}\right\} \\
& =-S_{1}+S_{2}-S_{3}+S_{4}-S_{5}+S_{6} \\
& \quad-S_{7}+S_{8}-S_{9}+S_{10} \\
& =a_{2}+a_{4}+a_{6}+a_{8}+a_{10}\left(\because-S_{n}+S_{n+1}=a_{n+1}\right) \\
& =5 a_{6}\left(\because a_{2}+a_{10}=a_{4}+a_{8}=2 a_{6}\right) \\
& =5\left(a_{1}+5 \times 2\right) \\
& =5 a_{1}+50 \\
& \leq 550 \\
& \text { 따라서 } a_{1} \leq 100 \text { 인 홀수 } a_{1} \text { 의 개수는 } 50 \text { 이다. }
\end{aligned}
\]

\begin{center}
\includegraphics[max width=\textwidth]{2024_08_01_b4f29a259b9b99d8349ag-4(3)}
\end{center}

\begin{center}
\includegraphics[max width=\textwidth]{2024_08_01_b4f29a259b9b99d8349ag-4(1)}
\end{center}

( i ), (ii)에 의하여 조건을 만족시키는 자연수 $a_{1}$ 의 개수는 $2+50=52$

\begin{enumerate}
  \setcounter{enumi}{21}
  \item $h(x)=f(x)-\int_{-2}^{x}(t-1) f^{\prime}(t) d t$ 라 하자.
\end{enumerate}

이때 부등식 $f(x) \leq \int_{-2}^{x}(t-1) f^{\prime}(t) d t$ 는

부등식 $h(x) \leq 0$ 과 같다.

실수 $a(a \neq 0)$ 에 대하여 삼차함수 $f(x)$ 의 최고차항 을 $a x^{3}$ 이라 하면

$(t-1) f^{\prime}(t)$ 의 최고차항은 $3 a t^{3}$ 이고,

$\int_{-2}^{x}(t-1) f^{\prime}(t) d t$ 의 최고차항은 $\frac{3}{4} a x^{4}$ 이므로

$h(x)$ 의 최고차항은 $-\frac{3}{4} a x^{4}$ 이다.

곧, $h(x)$ 는 최고차항의 계수가 $-\frac{3}{4} a$ 인 사차함수 이다.

그런데 부등식 $h(x) \leq 0$ 을 만족시키는 실수 $x$ 의 값이 $g(4)$ 뿐이므로 모든 실수 $x$ 에 대하여 $h(x) \geq 0$ 이다.

따라서 $-\frac{3}{4} a>0$, 곧 $a<0$

이때 삼차함수 $f(x)$ 가 역함수 $g(x)$ 를 가지므로

$f(x)$ 는 감소함수이다.

….. (7)

한편,

$h^{\prime}(x)=f^{\prime}(x)-(x-1) f^{\prime}(x)$

\[
=-(x-2) f^{\prime}(x)
\]

이고 (ㄱ)에 의하여 모든 실수 $x$ 에 대하여 $f^{\prime}(x) \leq 0$ 이므로 함수 $h(x)$ 는 $x=2$ 에서만 극소이고, 이때 최솟값을 갖는다.

따라서 부등식 $h(x) \leq 0$ 을 만족시키는 유일한 실수 $x$ 의 값은 2 이므로

$g(4)=2$

곧, $f(2)=4$ 이고 $h(g(4))=h(2)=0 \quad$ …… (ㄴ)

따라서 $h(2)=f(2)-\int_{-2}^{2}(t-1) f^{\prime}(t) d t=0$

곧, $f(2)-\int_{-2}^{2} t f^{\prime}(t) d t+\int_{-2}^{2} f^{\prime}(t) d t=0 \quad$...... (ㄷ)

함수 $f(x)$ 는 최고차항의 계수가 $a(a<0)$ 이고 이차항의 계수가 0 인 삼차함수이므로

$f(x)=a x^{3}+b x+c(b, c$ 는 상수 $)$

로 놓을 수 있다.

이때 $f^{\prime}(x)=3 a x^{2}+b$ 이므로

$\int_{-2}^{2} f^{\prime}(t) d t=2 \int_{0}^{2} f^{\prime}(t) d t=2\{f(2)-f(0)\}$

$x f^{\prime}(x)=3 a x^{3}+b x$ 이므로

$\int_{-2}^{2} t f^{\prime}(t) d t=0$

따라서 (ㄷ)에서 $f(2)-0+2\{f(2)-f(0)\}=0$

이때 (ㄴ)에서 $f(2)=4$ 이므로 $f(0)=6$

또한 (ㄱ)에 의하여 $g(x)$ 는 감소함수이고

$g(4)=2$ 이므로 $g(0)>2$\\
따라서 $|f(0)|=2|g(0)|$ 에서 $6=2 g(0)$ 이므로

$g(0)=3$, 곧 $f(3)=0$

$f(x)=a x^{3}+b x+c$ 에 대하여

$f(2)=4$ 이므로 $8 a+2 b+c=4$,

$f(0)=6$ 이므로 $c=6$,

$f(3)=0$ 이므로 $27 a+3 b+c=0$

위의 세 식을 연립하여 풀면

$a=-\frac{1}{5}, b=-\frac{1}{5}, c=6$

곧, $f(x)=-\frac{1}{5} x^{3}-\frac{1}{5} x+6$ 이므로

$f(1)=\frac{28}{5}$

따라서 $10 \times f(1)=56$

\section*{수학 영역}
\section*{확률과 통계}
\begin{center}
\includegraphics[max width=\textwidth]{2024_08_01_b4f29a259b9b99d8349ag-5}
\end{center}

\section*{해설}
\begin{enumerate}
  \setcounter{enumi}{22}
  \item $\mathrm{V}(X)=100 \times \frac{2}{5} \times\left(1-\frac{2}{5}\right)=24$

  \item $\mathrm{P}(A \cap B)=\frac{1}{11}, \mathrm{P}(A \cup B)-\mathrm{P}(A)=\frac{7}{11}$ 이므로 $\mathrm{P}(A \cup B)=\mathrm{P}(A)+\mathrm{P}(B)-\mathrm{P}(A \cap B)$ 에서 $\mathrm{P}(B)=\mathrm{P}(A \cap B)+\{\mathrm{P}(A \cup B)-\mathrm{P}(A)\}$

\end{enumerate}

\[
\begin{aligned}
& =\frac{1}{11}+\frac{7}{11} \\
& =\frac{8}{11}
\end{aligned}
\]

두 사건 $A, B$ 가 서로 독립이므로

$\mathrm{P}(A \cap B)=\mathrm{P}(A) \mathrm{P}(B)=\frac{1}{11}$

따라서 $\mathrm{P}(A)=\frac{1}{8}$

25.

\begin{center}
\includegraphics[max width=\textwidth]{2024_08_01_b4f29a259b9b99d8349ag-5(4)}
\end{center}

위의 그림과 같이 네 지점 $\mathrm{P}_{1}, \mathrm{P}_{2}, \mathrm{P}_{3}, \mathrm{P}_{4}$ 에 대하여 지점 A 에서 지점 B 까지 도로를 따라 최단 거리로 갈 때

지점 $\mathrm{P}_{1}$ 을 지나는 경우의 수는 1 ,

지점 $\mathrm{P}_{2}$ 를 지나는 경우의 수는 1 ,

지점 $\mathrm{P}_{1}$ 과 $\mathrm{P}_{2}$ 를 지나지 않는

경우 $\left(\mathrm{A} \rightarrow \mathrm{P}_{3} \rightarrow \mathrm{P}_{4} \rightarrow \mathrm{B}\right)$ 의 수는 $\frac{7!}{3!4!}=35$

따라서 구하는 경우의 수는 $1+1+35=37$

\begin{enumerate}
  \setcounter{enumi}{25}
  \item 모표준편차가 50 이므로 표본의 크기가 49 인 표본평균이 $\overline{x_{1}}$ 일 때 $m$ 에 대한 신뢰도 $95 \%$ 의 신뢰구간은
\end{enumerate}

$\overline{x_{1}}-1.96 \times \frac{50}{\sqrt{49}} \leq m \leq \overline{x_{1}}+1.96 \times \frac{50}{\sqrt{49}}$

이 신뢰구간이 $a \leq m \leq b$ 와 같으므로

$b-a=2 \times 1.96 \times \frac{50}{\sqrt{49}}=28$

또한 표본의 크기가 36 인 표본평균이 $\overline{x_{2}}$ 일 때 $m$ 에 대한 신뢰도 $99 \%$ 의 신뢰구간은

$\overline{x_{2}}-2.58 \times \frac{50}{\sqrt{36}} \leq m \leq \overline{x_{2}}+2.58 \times \frac{50}{\sqrt{36}}$

이 신뢰구간이 $b-20 \leq m \leq c$ 와 같으므로

$c-(b-20)=2 \times 2.58 \times \frac{50}{\sqrt{36}}=43$

곧, $c-b=23$

따라서

$c-a=(b-a)+(c-b)$

$=28+23$

$=51$\\
27. 한 번의 시행에서 꺼낸 두 장의 카드에 적힌 두 수의 합은 3 또는 4 이다.

두 수의 합이 3 인 사건을 $A$ 라 하면 사건 $A^{C}$ 은 두 수의 합이 4 인 사건이다.

이때 $\mathrm{P}(A)=\frac{2}{3}$ 이고, $\mathrm{P}\left(A^{C}\right)=\frac{1}{3}$

5 번의 시행 중 사건 $A$ 가 일어난 횟수를

$a(0 \leq a \leq 5)$ 라 하면 사건 $A^{C}$ 이 일어난 횟수는 $5-a$ 이므로 기록한 5 개의 수의 합은

$3 \times a+4 \times(5-a)=20-a$

기록한 5 개의 수의 합이 17 이려면

$20-a=17$ 에서 $a=3$

곧, 기록한 5 개의 수의 합이 17 이려면 5 번의 시행 중 사건 $A$ 가 3 번 일어나고 사건 $A^{C}$ 이 2 번 일어나 면 되므로 이 확률은

${ }_{5} \mathrm{C}_{3} \times\left(\frac{2}{3}\right)^{3}\left(\frac{1}{3}\right)^{2}=\frac{80}{243}$

\begin{enumerate}
  \setcounter{enumi}{27}
  \item 확률변수 $Z$ 가 표준정규분포를 따른다고 하자. 두-확률변수 $X, Y$ 가 각각 정규분포 $\mathrm{N}\left(0,5^{2}\right)$, $\mathrm{N}\left(t, 5^{2}\right)$ 을 따르므로
\end{enumerate}

$\mathrm{P}(0 \leq X \leq 1) \leq \mathrm{P}(t \leq Y \leq 5)$

$\mathrm{P}\left(0 \leq \frac{X}{5} \leq \frac{1}{5}\right) \leq \mathrm{P}\left(\frac{t-t}{5} \leq \frac{Y-t}{5} \leq \frac{5-t}{5}\right)$

$\mathrm{P}\left(0 \leq Z \leq \frac{1}{5}\right) \leq \mathrm{P}\left(0 \leq Z \leq \frac{5-t}{5}\right)$

곧, $\frac{1}{5} \leq \frac{5-t}{5}$ 에서 $0<t \leq 4$

이때

$\mathrm{P}(-1 \leq Y \leq 1)$

$=\mathrm{P}\left(\frac{-1-t}{5} \leq \frac{Y-t}{5} \leq \frac{1-t}{5}\right)$

$=\mathrm{P}\left(\frac{-1-t}{5} \leq Z \leq \frac{1-t}{5}\right)$

\begin{center}
\includegraphics[max width=\textwidth]{2024_08_01_b4f29a259b9b99d8349ag-5(2)}
\end{center}

이때 닫힌구간 $\left[\frac{-1-t}{5}, \frac{1-t}{5}\right]$ 의 길이는

$\frac{1-t}{5}-\frac{-1-t}{5}=\frac{2}{5}$ 로 일정하므로

$\mathrm{P}\left(\frac{-1-t}{5} \leq Z \leq \frac{1-t}{5}\right)$ 의 값은

$\left|\frac{\frac{-1-t}{5}+\frac{1-t}{5}}{2}\right|=\frac{t}{5}$ 의 값이 클수록 작다.

…… 줄숄

$0<t \leq 4$ 이므로 $t=4$ 일 때 $\frac{t}{5}$ 의 값은 최대가 되고

이때 $\mathrm{P}\left(\frac{-1-t}{5} \leq Z \leq \frac{1-t}{5}\right)$ 의 최솟값은

$\mathrm{P}(-1 \leq Z \leq-0.6)$

$=\mathrm{P}(0.6 \leq Z \leq 1)$

$=\mathrm{P}(0 \leq Z \leq 1)-\mathrm{P}(0 \leq Z \leq 0.6)$

$=0.3413-0.2257$

$=0.1156$

곧, $\mathrm{P}(-1 \leq Y \leq 1)$ 의 최솟값은 0.1156 이다

\section*{률}
확률변수 $X$ 가 정규분포 $\mathrm{N}\left(m, \sigma^{2}\right)$ 을 따르고 $b-a$ 의 값이 일정하다고 하자.\\
이 때 $\mathrm{P}(a \leq X \leq b)$ 의 값은 $\frac{a+b}{2}$ 의 값이 $m$ 에 가까울수록 크고 $\frac{a+b}{2}=m$ 일 때 최대가 된다.

\begin{center}
\includegraphics[max width=\textwidth]{2024_08_01_b4f29a259b9b99d8349ag-5(3)}
\end{center}

\begin{enumerate}
  \setcounter{enumi}{28}
  \item $(b-a)(b-a-3) \leq 0$ 에서
\end{enumerate}

$0 \leq b-a \leq 3$

$(c-b)(c-b-5) \leq 0$ 에서

$0 \leq c-b \leq 5$

따라서 세 자연수 $a, b, c$ 는

$1 \leq a \leq b \leq c \leq 10, b-a \leq 3, c-b \leq 5 \quad$ …‥ (ㄱ)

를 만족시켜야 한다.

$a-1=x, b-a=y, c-b=z, 10-c=w$ 라 하면

$1 \leq a \leq b \leq c \leq 10$ 과

$b-a \leq 3, c-b \leq 5$ 에서

$0 \leq x \leq 9,0 \leq y \leq 3,0 \leq z \leq 5,0 \leq w \leq 9$

….. (ㄴ)

이고,

$x+y+z+w=(a-1)+(b-a)+(c-b)+(10-c)$

에서

\begin{center}
\includegraphics[max width=\textwidth]{2024_08_01_b4f29a259b9b99d8349ag-5(1)}
\end{center}

이때 (ㄱ)을 만족시키는 순서쌍 $(a, b, c)$ 의 개수는

(ㄴ), (ㄷ)을 만족시키는 순서쌍 $(x, y, z, w)$ 의 개수와 같다.

$x+y+z+w=9$ 인 음이 아닌 정수 $x, y, z, w$ 의 순서쌍 $(x, y, z, w)$ 의 집합을 $A$,

$x+y+z+w=9$ 이고 $y \geq 4$ 인 음이 아닌 정수

$x, y, z, w$ 의 순처쌍 $(x, y, z, w)$ 의 집합을 $B$,

$x+y+z+w=9$ 이고 $z \geq 6$ 인 음이 아닌 정수

$x, y, z, w$ 의 순서쌍 $(x, y, z, w)$ 의 집합을 $C$ 라 하자.

(ㄴ), (ㄷ)을 만족시키는 순서쌍 $(x, y, z, w)$ 의 집합은 $A-(B \cup C)$ 이다.

이때 $y \geq 4, z \geq 6$ 이면 $x+y+z+w \geq 10$ 이므로 $B \cap C=\varnothing$

따라서 $n(A-(B \cup C))=n(A)-n(B)-n(C)$

$n(A)=(x+y+z+w=9$ 인 음이 아닌 정수

$x, y, z, w$ 의 순서쌍 $(x, y, z, w)$ 의 개수)

$={ }_{4} \mathrm{H}_{9}$

$={ }_{12} \mathrm{C}_{9}$

$=220$

$n(B)=(x+y+z+w=9$ 이고 $y \geq 4$ 인 음이 아닌 정수 $x, y, z, w$ 의 순서쌍 $(x, y, z, w)$ 의 개수)

$={ }_{4} \mathrm{H}_{9-4}$

$={ }_{8} \mathrm{C}_{5}$

$=56$,

$n(C)=(x+y+z+w=9$ 이고 $z \geq 6$ 인 음이 아닌 정수 $x, y, z, w$ 의 순서쌍 $(x, y, z, w)$ 의 개수)

$={ }_{4} \mathrm{H}_{9-6}$

$={ }_{6} \mathrm{C}_{3}$

$=20$

따라서 $n(A)-n(B)-n(C)=220-56-20=144$

곧, 구하는 순서쌍 $(a, b, c)$ 의 개수는 144 이다.

\section*{수학 영역}
\begin{enumerate}
  \setcounter{enumi}{29}
  \item [실행 A], [실행B], [실행 B]를 이 순서대로 시행한 후 상자에 들어 있는 공의 개수가 짝수인 사건을 $X$, 상자에 들어 있는 흰 공의 개수가 1 이하인 사건을 $Y$ 라 하면 구하는 확률은 $\mathrm{P}(Y \mid X)=\frac{\mathrm{P}(X \cap Y)}{\mathrm{P}(X)}$
\end{enumerate}

우선 사건 $X$ 가 일어나는 경우를 살펴보자 [실행 A]를 한 후 상자에 있는 공의 개수는 4 로, 짝수이다.

[실행 B]를 두 번 반복한 후 상자에 들어 있는 공의 개수가 짝수이기 위해서는 두 번의 [실행 B]에서 검은 공만 2 번 꺼내거나 흰 공만 2 번 꺼내야 한다 [실행 A ]에서 꺼낸 공 중 검은 공의 개수를 $a$ 라 하면 $a$ 의 값은 1 또는 2 또는 3 이다

( i ) $a=1$ 인 경우

\[
\text { [실행 } \mathrm{A} \text { ]에서 } a=1 \text { 일 확률은 } \frac{{ }_{3} \mathrm{C}_{1} \times{ }_{3} \mathrm{C}_{3}}{{ }_{6} \mathrm{C}_{4}}=\frac{1}{5}
\]

이때 상자에 들어 있는 검은 공과 흰 공의 개수는 각각 1,3 이다.

두 번의 [실행B]에서

검은 공만 2 번 꺼낼 확률은 $\frac{1}{4} \times \frac{1}{4}=\frac{1}{16}$,

흰 공만 2 번 꺼낼 확률은 $\frac{3}{4} \times \frac{2}{3}=\frac{1}{2}$

따라서 $a=1$ 이고 사건 $X$ 가 일어날 확률은

\[
\frac{1}{5} \times\left(\frac{1}{16}+\frac{1}{2}\right)=\frac{9}{80}
\]

(ii) $a=2$ 인 경우

[실행 A ]에서 $a=2$ 일 확률은 $\frac{{ }_{3} \mathrm{C}_{2} \times{ }_{3} \mathrm{C}_{2}}{{ }_{6} \mathrm{C}_{4}}=\frac{3}{5}$

이때 상자에 들어 있는 검은 공과 흰 공의 개수는

각각 2,2 이다.

두 번의 [실행B]에서

검은 공만 2 번 꺼낼 확률은 $\frac{2}{4} \times \frac{2}{4}=\frac{1}{4}$,

흰 공만 2 번 꺼낼 확률은 $\frac{2}{4} \times \frac{1}{3}=\frac{1}{6}$

따라서 $a=2$ 이고 사건 $X$ 가 일어날 확률은

\[
\frac{3}{5} \times\left(\frac{1}{4}+\frac{1}{6}\right)=\frac{1}{4}
\]

(iii) $a=3$ 인 경우

[실행 A ]에서 $a=3$ 일 확률은 $\frac{{ }_{3} \mathrm{C}_{3} \times{ }_{3} \mathrm{C}_{1}}{{ }_{6} \mathrm{C}_{4}}=\frac{1}{5}$

이때 상자에 들어 있는 검은 공과 흰 공의 개수는 각각 3,1 이다.

두 번의 [실행 B]에서

검은 공만 2 번 꺼낼 확률은 $\frac{3}{4} \times \frac{3}{4}=\frac{9}{16}$,

흰 공만 2 번 꺼낼 확률은 $\frac{1}{4} \times \frac{0}{3}=0$

따라서 $a=3$ 이고 사건 $X$ 가 일어날 확률은

\[
\frac{1}{5} \times\left(\frac{9}{16}+0\right)=\frac{9}{80}
\]

( i ) (iii)에 의하여

$\mathrm{P}(X)=\frac{9}{80}+\frac{1}{4}+\frac{9}{80}=\frac{19}{40}$

한편, 사건 $X \cap Y$ 가 일어나는 경우는 다음과 같이 세 가지 경우가 있다

( i )에서 $a=1$ 이고 흰 공만 2 번 꺼내는 경우

이 확률은 $\frac{1}{5} \times \frac{1}{2}=\frac{1}{10}$

(ii)에서 $a=2$ 이고 흰 공만 2 번 꺼내는 경우

이 확률은 $\frac{3}{5} \times \frac{1}{6}=\frac{1}{10}$

(iii)에서 $a=3$ 이고 검은 공만 2 번 꺼내는 경우

이 확률은 $\frac{1}{5} \times \frac{9}{16}=\frac{9}{80}$\\
곧, $\mathrm{P}(X \cap Y)=\frac{1}{10}+\frac{1}{10}+\frac{9}{80}=\frac{5}{16}$

따라서 구하는 확률은

$\mathrm{P}(Y \mid X)=\frac{\mathrm{P}(X \cap Y)}{\mathrm{P}(X)}=\frac{\frac{5}{16}}{\frac{19}{40}}=\frac{25}{38}$

곧, $p=\frac{25}{38}$ 이므로 $38 \times p=25$\\
미적분

\textbackslash section*\{\begin{tabular}{|l|l|l|l|l|l|l|l|l|l|}
\hline
23 & (4) & 24 & $(2)$ & 25 & (1) & 26 & (5) & 27 & (4) \\
\hline
28 & (5) & 29 & 90 & 30 & 15 &  &  &  &  \\
\hline
\end{tabular}\}

\section*{해설}
\begin{enumerate}
  \setcounter{enumi}{22}
  \item $\lim _{x \rightarrow 0} \frac{e^{5 x}-1-\sin x}{x}$
\end{enumerate}

$=\lim _{x \rightarrow 0}\left(\frac{e^{5 x}-1}{5 x} \times 5-\frac{\sin x}{x}\right)$

$=1 \times 5-1$

$=4$

\begin{enumerate}
  \setcounter{enumi}{23}
  \item $\int \frac{4-x}{e^{x}} d x$
\end{enumerate}

$=\int(4-x)\left(-e^{-x}\right)^{\prime} d x$

$=(4-x)\left(-e^{-x}\right)-\int(-1)\left(-e^{-x}\right) d x$

$=(x-4) e^{-x}+e^{-x}+C(C$ 는 적분상수

$=(x-3) e^{-x}+C$

이므로

\[
\begin{aligned}
\int_{0}^{3} \frac{4-x}{e^{x}} d x & =\left[(x-3) e^{-x}\right]_{0}^{3} \\
& =0-(-3) \\
& =3
\end{aligned}
\]

\begin{enumerate}
  \setcounter{enumi}{24}
  \item $b_{n}=(\sqrt{n+4}-\sqrt{n}) a_{n}$ 이라 하면
\end{enumerate}

$\lim _{n \rightarrow \infty} b_{n}=1$

$a_{n}=\frac{b_{n}}{\sqrt{n+4}-\sqrt{n}}=\frac{1}{4}(\sqrt{n+4}+\sqrt{n}) b_{n}$ 이므로

$\lim _{n \rightarrow \infty} \frac{a_{n}}{\sqrt{n}}=\lim _{n \rightarrow \infty} \frac{(\sqrt{n+4}+\sqrt{n}) b_{n}}{4 \sqrt{n}}$

$=\lim _{n \rightarrow \infty}\left(\frac{\sqrt{1+\frac{4}{n}}+1}{4} \times b_{n}\right)$

$=\frac{2}{4} \times 1$

$=\frac{1}{2}$

따라서 $\lim _{n \rightarrow \infty} \frac{a_{n}^{2}}{\dot{n}}=\lim _{n \rightarrow \infty}\left(\frac{a_{n}}{\sqrt{n}}\right)^{2}=\frac{1}{4}$

\begin{enumerate}
  \setcounter{enumi}{25}
  \item 주어진 입체도형을 직선 $x=t(1 \leq t \leq e)$ 를 포함 하고 $x$ 축에 수직인 평면으로 자른 단면의 넓이는 $\left(\frac{\ln t+1}{\sqrt{t}}\right)^{2}$ 이므로 이 입체도형의 부피는
\end{enumerate}

$\int_{1}^{e}\left(\frac{\ln t+1}{\sqrt{t}}\right)^{2} d t=\int_{1}^{e} \frac{(\ln t)^{2}+2 \ln t+1}{t} d t$ $u=\ln t$ 라 하면 $\frac{d u}{d t}=\frac{1}{t}$ 이고

$t=1$ 일 때 $u=0, t=e$ 일 때 $u=1$ 이므로

$\int_{1}^{e} \frac{(\ln t)^{2}+2 \ln t+1}{t} d t$

$=\int_{0}^{1}\left(u^{2}+2 u+1\right) d u$

$=\left[\frac{1}{3} u^{3}+u^{2}+u\right]_{0}^{1}$

$=\frac{7}{3}$

따라서 구하는 입체도형의 부피는 $\frac{7}{3}$ 이다.

\section*{수학 영역}
\begin{enumerate}
  \setcounter{enumi}{26}
  \item 매개변수 $t(t \neq 0)$ 으로 나타내어진 곡선 $x=t^{2}+2 t, y=k(t+\ln |t|)$ 위에 서로 다른 두 점 $(a, b),(a,-b)$ 가 있으므로 다음 조건을 만족시키는 0 이 아닌 두 실수 $t_{1}, t_{2}\left(t_{1} \neq t_{2}\right)$ 가 존재한다.
\end{enumerate}

$a=t_{1}^{2}+2 t_{1}$ ….. (7)

$b=k\left(t_{1}+\ln \left|t_{1}\right|\right)$ ….. (ㄴ)

$a=t_{2}^{2}+2 t_{2}$ ….. (ㄷ)

$-b=k\left(t_{2}+\ln \left|t_{2}\right|\right)$

…… (ㄹ)

(ㄱ), (ㄷ)에서 $t$ 에 대한 이차방정식 $a=t^{2}+2 t$, 곧 $t^{2}+2 t-a=0$ 은 0 이 아닌 두 실근 $t_{1}, t_{2}\left(t_{1} \neq t_{2}\right)$ 를 갖는다.

따라서 $a>-1$ 이고 $a \neq 0$ 이다.

또한 이차방정식의 근과 계수의 관계에 의하여

$t_{1}+t_{2}=-2, t_{1} t_{2}=-a$ ...... (a)

두 등식 (ㄴ), (ㄹ)을 변끼리 더하여 정리하면 $0=k\left(t_{1}+t_{2}+\ln \left|t_{1} t_{2}\right|\right)$

$0=-2+\ln |a|(\because k>0$, (D) $)$

곧, $a=e^{2}(\because a>-1)$

한편, $\frac{d x}{d t}=2 t+2, \frac{d y}{d t}=k\left(1+\frac{1}{t}\right)$ 이므로

$t \neq-1$ 이고 $t \neq 0$ 일 때

$\frac{d y}{d x}=\frac{\frac{d y}{d t}}{\frac{d x}{d t}}=\frac{k\left(1+\frac{1}{t}\right)}{2 t+2}=\frac{k}{2 t}$

따라서 곡선 위의 두 점 $(a, b),(a,-b)$ 에서의 접선의 기울기는 각각 $\frac{k}{2 t_{1}}, \frac{k}{2 t_{2}}$ 이다.

이 두 접선이 서로 수직이므로 $\frac{k}{2 t_{1}} \times \frac{k}{2 t_{2}}=-1$ 에서

$k^{2}=-4 t_{1} t_{2}=4 a=4 e^{2}$

이때 $k>0$ 이므로 $k=2 e$

\begin{enumerate}
  \setcounter{enumi}{27}
  \item 조건 (나)의 $\{f(x)\}^{2}+3 f(x)=a x+4$ 에서 $x=0$ 일 때 $\{f(0)\}^{2}+3 f(0)=4$
\end{enumerate}

$\{f(0)+4\}\{f(0)-1\}=0$

$f(0)>0$ 이므로 $f(0)=1$

조건 (가)의 $f(x+1)=f(x)+1$ 에서

$x=0$ 일 때 $f(1)=f(0)+1=2$

조건 (나)에서 $x=1$ 일 때

$\{f(1)\}^{2}+3 f(1)=a+4$

곧, $2^{2}+3 \times 2=a+4$ 에서 $a=6$

따라서 $\{f(x)\}^{2}+3 f(x)=6 x+4$

한편, 조건 (가)에 의하여

\[
\begin{aligned}
f(x+3) & =f(x+2)+1 \\
& =f(x+1)+2 \\
& =f(x)+3
\end{aligned}
\]

이므로

$\int_{0}^{3} \frac{1}{\{f(x)\}^{2}+f(x)} d x$

$=\int_{0}^{3} \frac{1}{f(x)\{f(x)+1\}} d x$

$=\int_{0}^{3} \frac{1}{f(x)} d x-\int_{0}^{3} \frac{1}{f(x)+1} d x$

$=\int_{0}^{3} \frac{1}{f(x)} d x-\int_{0}^{3} \frac{1}{f(x+1)} d x$

$=\int_{0}^{3} \frac{1}{f(x)} d x-\int_{1}^{4} \frac{1}{f(x)} d x$

$=\int_{0}^{1} \frac{1}{f(x)} d x-\int_{3}^{4} \frac{1}{f(x)} d x$

$=\int_{0}^{1} \frac{1}{f(x)} d x-\int_{0}^{1} \frac{1}{f(x+3)} d x$

$=\int_{0}^{1} \frac{f(x+3)-f(x)}{f(x) f(x+3)} d x$ $=\int_{0}^{1} \frac{3}{f(x) f(x+3)} d x(\because f(x+3)=f(x)+3)$

$=\int_{0}^{1} \frac{3}{\{f(x)\}^{2}+3 f(x)} d x$

$=\int_{0}^{1} \frac{3}{6 x+4} d x(\because$ (ㄱ) $)$

$=\left[\frac{1}{2} \ln (6 x+4)\right]_{0}^{1}$

$=\frac{1}{2}(\ln 10-\ln 4)$

$=\frac{1}{2} \ln \frac{5}{2}$

\begin{enumerate}
  \setcounter{enumi}{28}
  \item $a_{n+1}= \begin{cases}\frac{1}{2} a_{n} & \left(a_{n} \leq 3\right) \\ a_{n}-p & \left(a_{n}>3\right)\end{cases}$
\end{enumerate}

에 의하여 $a_{n}>3$ 일 때,

$a_{n+1}=a_{n}-p$ 이고 $a_{1}=12>3$ 이므로

$a_{m-1}>3 \geq a_{m}$ 을 만족시키는 2 이상의 자연수 $m$ 이 존재한다.

…… (ㄴ)

이때 (ㄱ)에 의하여 수열 $\left\{a_{n}\right\}$ 을 나열하면 다음과 같다.

$a_{1}, a_{2}, \cdots, a_{m-1}, a_{m}, \frac{a_{m}}{2}, \frac{a_{m}}{2^{2}}, \cdots$

여기서 $a_{1}, a_{2}, \cdots, a_{m-1}, a_{m}$ 은 공차가 $-p$ 인 등차 수열을 이루고,

$a_{m}, \frac{a_{m}}{2}, \frac{a_{m}}{2^{2}}, \cdots$ 은 공비가 $\frac{1}{2}$ 인 등비수열을 이룬다.

그런데 $\sum_{n=1}^{\infty} a_{n}<\sum_{n=1}^{\infty}\left|a_{n}\right|$ 이므로 수열 $\left\{a_{n}\right\}$ 의 항 중

에는 음수가 존재한다.

곧, $a_{m}<0$

이때 $a_{m}$ 은 정수이므로 $a_{m} \leq-1$

또한 $a_{m-1}$ 도 정수이므로 (ㄴ)에 의하여

$a_{m-1} \geq 4$

따라서 $p=a_{m-1}-a_{m} \geq 5$

$a_{1}=12$ 이므로

$a_{m-1}=a_{1}-p\{(m-1)-1\}$

\[
=12-p(m-2)
\]

$a_{m-1} \geq 4$ 에서 $12-p(m-2) \geq 4$

곧, $p(m-2) \leq 8$

이때 $p \geq 5$ 이므로 $5(m-2) \leq 8$

따라서 (ㄴ)에 의하여 $m=2$ 또는 $m=3$

( i ) $m=2$ 인 경우

$a_{m}=a_{2}=a_{1}-p=12-p$

$a_{m} \leq-1$ 이므로

$12-p \leq-1$ 에서 $p \geq 13$

이 경우 수열 $\left\{a_{n}\right\}$ 은

$a_{1}, a_{2}, \frac{a_{2}}{2}, \frac{a_{2}}{2^{2}}, \cdots$ 이므로

$\sum_{n=1}^{\infty} a_{n}=a_{1}+\sum_{n=2}^{\infty} a_{n}$

$=a_{1}+\frac{a_{2}}{1-\frac{1}{2}}$

$=a_{1}+2 a_{2}$

$=12+2(12-p)$

$=36-2 p$

$\sum_{n=1}^{\infty} a_{n}>0$ 에서 $36-2 p>0$

곧, $p<18$

따라서 $13 \leq p<18$

(ii) $m=3$ 인 경우

$a_{m-1}=a_{2}=a_{1}-p=12-p$ $a_{m-1} \geq 4$ 이므로

$12-p \geq 4$ 에서 $p \leq 8$

또한 $a_{m}=a_{3}=a_{1}-p \times 2=12-2 p$

$a_{m} \leq-1$ 이므로

$12-2 p \leq-1$ 에서 $p \geq \frac{13}{2}$

따라서 $\frac{13}{2} \leq p \leq 8$

이 경우 수열 $\left\{a_{n}\right\}$ 은

$a_{1}, a_{2}, a_{3}, \frac{a_{3}}{2}, \frac{a_{3}}{2^{2}}, \cdots$ 이므로

\[
\begin{aligned}
\sum_{n=1}^{\infty} a_{n} & =\left(a_{1}+a_{2}\right)+\sum_{n=3}^{\infty} a_{n} \\
& =a_{1}+a_{2}+\frac{a_{3}}{1-\frac{1}{2}} \\
& =a_{1}+a_{2}+2 a_{3} \\
& =12+(12-p)+2(12-2 p) \\
& =48-5 p
\end{aligned}
\]

이때 $p \leq 8$ 이므로 $48-5 p>0$ 이 되어

$\sum_{n=1}^{\infty} a_{n}>0$ 을 만족시킨다.

따라서 $\frac{13}{2} \leq p \leq 8$

( i ), (ii)에 의하여 주어진 조건을 모두 만족시키는 자연수 $p$ 는 $13,14,15,16,17$ 과 7,8 이고 그 합은 $(13+14+15+16+17)+(7+8)=90$

\begin{enumerate}
  \setcounter{enumi}{29}
  \item 함수 $g(x)=\frac{4(x-1)^{4}}{x}(x>1)$ 이라 하자.
\end{enumerate}

\[
x>1 \text { 일 때 } g^{\prime}(x)=\frac{4(x-1)^{3}(3 x+1)}{x^{2}}>0 \text { 이므로 }
\]

함수 $g(x)$ 는 $x>1$ 에서 증가한다.

또한 $\lim _{x \rightarrow 1+} g(x)=0, \lim _{x \rightarrow \infty} g(x)=\infty$ 이므로

양수 $t$ 에 대하여 곡선 $y=g(x)(x>1)$ 과 직선 $y=t$ 의 교점의 개수는 1 이고 이 점이 P 이다. 따라서 점 P 의 좌표를 $(s, t)(s>1, t>0)$ 으로 놓을 수 있다.

점 $\mathrm{P}(s, t)$ 는 곡선 $y=g(x)$ 위에 있으므로

$t=g(s)$, 곧 $t=\frac{4(s-1)^{4}}{s}$

직선 OP 가 $x$ 축의 양의 방향과 이루는 예각의 크기가 $f(t)$ 이고 직선 OP 의 기울기는 $\frac{t-0}{s-0}=\frac{t}{s}$ 이므로

$\tan f(t)=\frac{t}{s}$

$t=a$ 일 때 $s=s_{0}\left(s_{0}>1\right)$ 이라 하자.

(ㄴ)에 $s=s_{0}, t=a$ 를 대입하면

$\tan f(a)=\frac{a}{s_{0}}$

$f(a)=\frac{\pi}{4}$ 이므로 $1=\frac{a}{s_{0}}$, 곧 $s_{0}=a$

따라서 $t=a$ 일 때, $s=a$ 이고 $a>1$ 이다.

(ㄱ)에 $s=a, t=a$ 를 대입하면

\[
a=\frac{4(a-1)^{4}}{a}
\]

$a^{2}=\left\{2(a-1)^{2}\right\}^{2}$

$a=2(a-1)^{2}(\because a>1)$

$(a-2)(2 a-1)=0$

$a>1$ 이므로 $a=2$

한편, (ㄱ)의 양변을 $t$ 에 대하여 미분하면

$1=g^{\prime}(s) \times \frac{d s}{d t}$

곧, $1=\frac{4(s-1)^{3}(3 s+1)}{s^{2}} \times \frac{d s}{d t}$

\section*{수학 영역}
$s=2$ 를 대입하면

$1=7 \times \frac{d s}{d t}$, 곧 $\frac{d s}{d t}=\frac{1}{7}$

(ㄴ)의 양변을 $t$ 에 대하여 미분하면

$\sec ^{2} f(t) \times f^{\prime}(t)=\frac{1 \times s-t \times \frac{d s}{d t}}{s^{2}}$

$s=2, t=2, \frac{d s}{d t}=\frac{1}{7}$ 을 대입하면

$\sec ^{2} f(2) \times f^{\prime}(2)=\frac{1 \times 2-2 \times \frac{1}{7}}{2^{2}}$

$f(2)=\frac{\pi}{4}$ 이므로

$2 \times f^{\prime}(2)=\frac{3}{7}$

곧, $f^{\prime}(a)=f^{\prime}(2)=\frac{3}{14}$ 이므로

$70 \times f^{\prime}(a)=15$\\
기하

\begin{center}
\begin{tabular}{|l|l|l|l|l|l|l|l|l|l|}
\hline
23 & $(3)$ & 24 & (4) & 25 & $(2)$ & 26 & (1) & 27 & (4) \\
\hline
28 & $(1)$ & 29 & 81 & 30 & 25 &  &  &  &  \\
\hline
\end{tabular}
\end{center}

\textbackslash section*\{\begin{tabular}{llllll}
28 & (1) & 29 & 81 & 30 & 25 \\
\hline
\end{tabular}\}

\section*{해설}
\begin{enumerate}
  \setcounter{enumi}{22}
  \item 점 $(3,2,-4)$ 를 $y$ 축에 대하여 대칭이동한 점의 좌표는 $(-3,2,4)$ 이고, 이것이 $(a, b, c)$ 와 같으므로 $a=-3, b=2, c=4$
\end{enumerate}

따라서 $a+b+c=3$

\begin{enumerate}
  \setcounter{enumi}{23}
  \item 점 $(\sqrt{a},-1)$ 이 타원 $\frac{x^{2}}{a^{2}}+\frac{y^{2}}{2}=1$ 위에 있으므로 $\frac{(\sqrt{a})^{2}}{a^{2}}+\frac{(-1)^{2}}{2}=1$ 에서
\end{enumerate}

$a=2$

따라서 타원 $\frac{x^{2}}{4}+\frac{y^{2}}{2}=1$ 위의 점 $(\sqrt{2},-1)$ 에서의 접선은 $\frac{\sqrt{2}}{4} x-\frac{1}{2} y=1$ 이고, 이 직선의 기울기는 $\frac{\sqrt{2}}{2}$ 이다.

\begin{enumerate}
  \setcounter{enumi}{24}
  \item 점 $\mathrm{A}(4,2)$ 를 지나고 벡터 $\vec{u}=(1,2)$ 에 평행한 직선의 방정식은
\end{enumerate}

$\frac{x-4}{1}=\frac{y-2}{2}$

이 직선이 $x$ 축과 만나는 점 P 의 좌표는

$y=0$ 일 때 $x=3$ 이므로 $\mathrm{P}(3,0)$ 이다.

점 $\mathrm{A}(4,2)$ 를 지나고 벡터 $\vec{u}=(1,2)$ 에 수직인

직선의 방정식은

$1 \times(x-4)+2 \times(y-2)=0$

곧, $x+2 y-8=0$

이 직선이 $x$ 축과 만나는 점 Q 의 좌표는 $y=0$ 일 때 $x=8$ 이므로 $\mathrm{Q}(8,0)$ 이다.

\begin{center}
\includegraphics[max width=\textwidth]{2024_08_01_b4f29a259b9b99d8349ag-8(3)}
\end{center}

따라서 삼각형 APQ 의 넓이는

$\frac{1}{2} \times \overline{\mathrm{PQ}} \times \mid$ 점 A 의 $y$ 좌표 $\left\lvert\,=\frac{1}{2} \times 5 \times 2=5\right.$

\begin{enumerate}
  \setcounter{enumi}{25}
  \item 점 P 에서 평면 EFGH 에 내린 수선의 발을 R 이라 하면 선분 EG 를 $1: 3$ 으로 내분하는 점이 R 이다.
\end{enumerate}

\begin{center}
\includegraphics[max width=\textwidth]{2024_08_01_b4f29a259b9b99d8349ag-8(2)}
\end{center}

또한 점 P 에서 직선 FH 에 내린 수선의 발이 Q 이 므로 삼수선의 정리에 의하여 점 R 에서 직선 FH 에 내린 수선의 발도 Q 이다.

직사각형 EFGH 의 두 대각선의 교점을 I라 하자.

\begin{center}
\includegraphics[max width=\textwidth]{2024_08_01_b4f29a259b9b99d8349ag-8}
\end{center}

$\overline{\mathrm{RE}}: \overline{\mathrm{RG}}=1: 3$ 에서 $\overline{\mathrm{RI}}=\frac{1}{4} \overline{\mathrm{EG}}$ 이고,

$\overline{\mathrm{QF}}: \overline{\mathrm{QH}}=3: 5$ 에서 $\overline{\mathrm{QI}}=\frac{1}{8} \overline{\mathrm{FH}}$

$\overline{\mathrm{EG}}=\overline{\mathrm{FH}}$ 이므로

$\cos (\angle \mathrm{RIQ})=\frac{\overline{\mathrm{QI}}}{\overline{\mathrm{RI}}}=\frac{1}{2}$, 곧 $\angle \mathrm{RIQ}=\frac{\pi}{3}$

삼각형 EFI 는 한 변의 길이가 4 인 정삼각형이므로 $\overline{\mathrm{QI}}=1, \overline{\mathrm{RQ}}=\sqrt{3}$ 이고

$\overline{\mathrm{PR}}=\overline{\mathrm{AE}}=\overline{\mathrm{AD}}=\overline{\mathrm{EH}}=4 \sqrt{3}$

따라서 직각삼각형 PQR 에서

$\overline{\mathrm{PQ}}=\sqrt{\overline{\mathrm{PR}}^{2}+\overline{\mathrm{RQ}}^{2}}=\sqrt{51}$

\begin{enumerate}
  \setcounter{enumi}{26}
  \item $\mathrm{H}_{1}(0, a), \mathrm{H}_{2}(0, b)(0<a<b)$ 라 하자.
\end{enumerate}

포물선 $y^{2}=4 p x$ 의 초점은 $\mathrm{F}(p, 0)$ 이고,

삼각형 $\mathrm{FH}_{1} \mathrm{H}_{2}$ 의 무게중심의 $x$ 좌표와 $y$ 좌표는 각각

$\frac{p+0+0}{3}=1, \frac{0+a+b}{3}=3$ 이므로

\begin{center}
\includegraphics[max width=\textwidth]{2024_08_01_b4f29a259b9b99d8349ag-8(1)}
\end{center}

또한 삼각형 $\mathrm{FP}_{1} \mathrm{P}_{2}$ 의 둘레의 길이와 사각형

$\mathrm{P}_{1} \mathrm{P}_{2} \mathrm{H}_{2} \mathrm{H}_{1}$ 의 둘레의 길이가 같으므로

$\overline{\mathrm{FP}_{1}}+\overline{\mathrm{FP}_{2}}=\overline{\mathrm{P}_{1} \mathrm{H}_{1}}+\overline{\mathrm{P}_{2} \mathrm{H}_{2}}+\overline{\mathrm{H}_{1} \mathrm{H}_{2}}$ 이고, 포물선의 정의에 의하여

$\overline{\mathrm{FP}_{1}}=\overline{\mathrm{P}_{1} \mathrm{H}_{1}}+3, \overline{\mathrm{FP}_{2}}=\overline{\mathrm{P}_{2} \mathrm{H}_{2}}+3$ 이므로 $b-a=\overline{\mathrm{H}_{1} \mathrm{H}_{2}}=6$


\begin{align*}
& \left|{\overline{\mathrm{OH}_{1}}}^{2}-{\overline{\mathrm{OH}_{2}}}^{2}\right|=\left|a^{2}-b^{2}\right|  \tag{ㄴ}\\
& =|(a+b)(a-b)| \\
& =9 \times 6(\because \text { (ㄱ), (ㄴ) }) \\
& =54
\end{align*}


\begin{enumerate}
  \setcounter{enumi}{27}
  \item 조건 (가)에 의하여 점 P 는 선분 BC 위에 있고, $\overline{\mathrm{BP}}=k(0<k<1), \overline{\mathrm{CP}}=1-k$
\end{enumerate}

점 Q 에서 직선 BC 에 내린 수선의 발을 H 라 하면 $\overrightarrow{\mathrm{PQ}}=k \overrightarrow{\mathrm{AC}}$ 에서

$\overrightarrow{\mathrm{PH}}+\overrightarrow{\mathrm{HQ}}=k(\overrightarrow{\mathrm{AB}}+\overrightarrow{\mathrm{BC}})$

두 벡터 $\overrightarrow{\mathrm{PH}}, \overrightarrow{\mathrm{BC}}$ 의 방향이 같고

두 벡터 $\overrightarrow{\mathrm{HQ}}, \overrightarrow{\mathrm{AB}}$ 의 방향이 같으므로

$\overrightarrow{\mathrm{PH}}=k \overrightarrow{\mathrm{BC}}$

$\overrightarrow{\mathrm{HQ}}=k \overrightarrow{\mathrm{AB}}$

$\overrightarrow{\mathrm{BH}}=\overrightarrow{\mathrm{BP}}+\overrightarrow{\mathrm{PH}}=2 k \overrightarrow{\mathrm{BC}}$

따라서

$|\overrightarrow{\mathrm{PH}}|=|k \overrightarrow{\mathrm{BC}}|=k$,

$|\overrightarrow{\mathrm{HQ}}|=|k \overrightarrow{\mathrm{AB}}|=k$,

$|\overrightarrow{\mathrm{BH}}|=|2 k \overrightarrow{\mathrm{BC}}|=2 k$

\section*{수학 영역}
$\angle \mathrm{BHQ}$ 가 직각인 직각삼각형 BHQ 에서

$\overline{\mathrm{BQ}}^{2}=\overline{\mathrm{BH}}^{2}+\overline{\mathrm{HQ}}^{2}=(2 k)^{2}+k^{2}=5 k^{2}$ 곧, $\overline{\mathrm{BQ}}=\sqrt{5} k$

\begin{center}
\includegraphics[max width=\textwidth]{2024_08_01_b4f29a259b9b99d8349ag-9}
\end{center}

두 벡터 $\overrightarrow{\mathrm{BQ}}, \overrightarrow{\mathrm{PQ}}$ 가 이루는 각의 크기를 $\alpha(0<\alpha<\pi)$,

두 벡터 $\overrightarrow{\mathrm{CQ}}, \overrightarrow{\mathrm{PQ}}$ 가 이루는 각의 크기를 $\beta(0<\beta<\pi)$ 라 하자.

(이때 $\cos \alpha=\frac{3 \sqrt{10}}{10}$

$\overrightarrow{\mathrm{BQ}} \cdot \overrightarrow{\mathrm{PQ}}=|\overrightarrow{\mathrm{BQ}}||\overrightarrow{\mathrm{PQ}}| \cos \alpha$ 이고,

$\overrightarrow{\mathrm{CQ}} \cdot \overrightarrow{\mathrm{PQ}}=|\overrightarrow{\mathrm{CQ}}||\overrightarrow{\mathrm{PQ}}| \cos \beta$ 이므로

조건 (나)에 의하여 $\cos \alpha=\cos \beta$, 곧 $\alpha=\beta$

$\overline{\mathrm{BP}}=k, \overline{\mathrm{CP}}=1-k, \overline{\mathrm{BQ}}=\sqrt{5} k$ 이므로

$\overline{\mathrm{BP}}: \overline{\mathrm{CP}}=\overline{\mathrm{BQ}}: \overline{\mathrm{CQ}}$ 에서

$\overline{\mathrm{CQ}}=\sqrt{5}(1-k)$

$|\overrightarrow{\mathrm{CH}}|=|\overrightarrow{\mathrm{BH}}-\overrightarrow{\mathrm{BC}}|$

\[
=|(2 k-1) \overrightarrow{\mathrm{BC}}|
\]

\[
=|2 k-1| \text {, }
\]

$\overrightarrow{\mathrm{CH}} \cdot \overrightarrow{\mathrm{HQ}}=0$ 이므로

$|\overrightarrow{\mathrm{CQ}}|^{2}=|\overrightarrow{\mathrm{CH}}+\overrightarrow{\mathrm{HQ}}|^{2}$

\[
\begin{aligned}
& =|\overrightarrow{\mathrm{CH}}|^{2}+2 \times \overrightarrow{\mathrm{CH}} \cdot \overrightarrow{\mathrm{HQ}}+|\overrightarrow{\mathrm{HQ}}|^{2} \\
& =|2 k-1|^{2}+k^{2}
\end{aligned}
\]

$\{\sqrt{5}(1-k)\}^{2}=|2 k-1|^{2}+k^{2}$

$k=\frac{2}{3}$

따라서 삼각형 BPQ 의 넓이는

$\frac{1}{2} \times \overline{\mathrm{BP}} \times \overline{\mathrm{HQ}}=\frac{1}{2} \times \frac{2}{3} \times \frac{2}{3}=\frac{2}{9}$

\section*{[다른 풀이]}
좌표평면에서

$\mathrm{A}(0,1), \mathrm{B}(0,0), \mathrm{C}(1,0), \mathrm{D}(1,1)$ 이라 하면 $\overrightarrow{\mathrm{BC}}=(1,0), \overrightarrow{\mathrm{AC}}=(1,-1)$ 이므로

조건 (가)에 의하여

$\overrightarrow{\mathrm{BP}}=(k, 0)(0<k<1), \overrightarrow{\mathrm{PQ}}=(k,-k)$

따라서

$\overrightarrow{\mathrm{BQ}}=\overrightarrow{\mathrm{BP}}+\overrightarrow{\mathrm{PQ}}=(2 k,-k)$,

$|\overrightarrow{\mathrm{BQ}}|=\sqrt{(2 k)^{2}+(-k)^{2}}=\sqrt{5} k$,

$\overrightarrow{\mathrm{BQ}} \cdot \overrightarrow{\mathrm{PQ}}=2 k \times k+(-k) \times(-k)=3 k^{2}$ 이고,

$\overrightarrow{\mathrm{CQ}}=\overrightarrow{\mathrm{BQ}}-\overrightarrow{\mathrm{BC}}=(2 k-1,-k)$,

$|\overrightarrow{\mathrm{CQ}}|=\sqrt{(2 k-1)^{2}+(-k)^{2}}=\sqrt{5 k^{2}-4 k+1}$

$\overrightarrow{\mathrm{CQ}} \cdot \overrightarrow{\mathrm{PQ}}=(2 k-1) \times k+(-k) \times(-k)=3 k^{2}-k$ 조건 (나)에 의하여

$\frac{3 k^{2}}{\sqrt{5} k}=\frac{3 k^{2}-k}{\sqrt{5 k^{2}-4 k+1}}$ 이므로

$3 \sqrt{5 k^{2}-4 k+1}=\sqrt{5}(3 k-1)$

$9\left(5 k^{2}-4 k+1\right)=5\left(9 k^{2}-6 k+1\right)$

곧, $k=\frac{2}{3}$

$\mathrm{P}\left(\frac{2}{3}, 0\right), \mathrm{Q}\left(\frac{4}{3},-\frac{2}{3}\right)$ 이므로

삼각형 BPQ 의 넓이는 $\frac{1}{2} \times \overline{\mathrm{BP}} \times \mid \mathrm{Q}$ 의 $y$ 좌표 $\mid$

$=\frac{1}{2} \times \frac{2}{3} \times \frac{2}{3}$

$=\frac{2}{9}$

\begin{enumerate}
  \setcounter{enumi}{28}
  \item 쌍곡선 $H: \frac{x^{2}}{a^{2}}-\frac{y^{2}}{b^{2}}=1$ 에서 $a>0, b>0$ 이라 하고 두 초점을 $\mathrm{F}\left(\sqrt{a^{2}+b^{2}}, 0\right), \mathrm{F}^{\prime}\left(-\sqrt{a^{2}+b^{2}}, 0\right)$ 이라 하자.
\end{enumerate}

쌍곡선 $H$ 의 두 점근선 $y= \pm \frac{b}{a} x$ 의 기울기의 곱이 $-\frac{b^{2}}{a^{2}}=-8$, 곧 $b^{2}=8 a^{2}$ 이므로 $\mathrm{F}(3 a, 0)$ 이다.

쌍곡선 $H$ 의 꼭짓점 중 $x$ 좌표가 음수인 점을

$\mathrm{B}(-a, 0)$ 이라 하면 쌍곡선 $H$ 와 서로 다른 세 점 에서만 만나고 원점 O 와 점 F 를 두 초점으로 하는 타원은 점 B 를 지나므로 이 타원의 장축의 길이는 $\overline{\mathrm{OF}}+2 \overline{\mathrm{OB}}=5 a$ 이며, $\overline{\mathrm{FA}}=1$ 이므로 타원의 정의에 의하여 $\overline{\mathrm{OA}}=5 a-1$ 이다.

쌍곡선 $H$ 의 주축의 길이는 $2 a$ 이므로 쌍곡선의

정의에 의하여 $\overline{\mathrm{F}^{\prime} \mathrm{A}}=2 a+1$ 이다.

\begin{center}
\includegraphics[max width=\textwidth]{2024_08_01_b4f29a259b9b99d8349ag-9(1)}
\end{center}

$\angle \mathrm{AOF}=\theta$ 라 하면 $\angle \mathrm{AOF}^{\prime}=\pi-\theta$

두 삼각형 AOF 와 $\mathrm{AOF}^{\prime}$ 에서 코사인법칙에 의하여

$1^{2}=(3 a)^{2}+(5 a-1)^{2}-2 \times 3 a \times(5 a-1) \times \cos \theta$,

$(2 a+1)^{2}$

$=(3 a)^{2}+(5 a-1)^{2}-2 \times 3 a \times(5 a-1) \times \cos (\pi-\theta)$

두 식을 변끼리 더하여 정리하면

$4 a^{2}+4 a+2=68 a^{2}-20 a+2$

곧, $a=\frac{3}{8}(\because a>0)$

따라서 $a^{2}+b^{2}=9 a^{2}=\frac{81}{64}$ 이므로

$64\left(a^{2}+b^{2}\right)=81$

\begin{enumerate}
  \setcounter{enumi}{29}
  \item 구 $S: x^{2}+y^{2}+z^{2}=25$ 의 중심은 원점 O 이고 반지름의 길이는 5 이므로
\end{enumerate}

$\overline{\mathrm{OA}}=\overline{\mathrm{OB}}=\overline{\mathrm{OC}}=5$

구 $S$ 와 점 A 에서 접하는 평면을 $\alpha$ 라 하자. (직선 OA$) \perp($ 평면 $\alpha$ )이고,

조건 (가)에서 (평면 $\alpha$ )//(평면 OBC )이므로 (직선 OA$) \perp($ 평면 OBC ) …... (7)

따라서 구 $S$ 를 평면 OAC 로 자른 단면은 다음 그림과 같다.

\begin{center}
\includegraphics[max width=\textwidth]{2024_08_01_b4f29a259b9b99d8349ag-9(2)}
\end{center}

삼각형 OAC 는 직각이등변삼각형이므로 그 넓이는 $\frac{1}{2} \times 5 \times 5=\frac{25}{2}$\\
두 평면 $\mathrm{OAC}, \mathrm{ABC}$ 는 직선 AC 를 교선으로 갖는다. 평면 ABC 위의 점 B 에서 평면 OAC 및 직선 AC 에 내린 수선의 발을 각각 찾아보자.

(평면 OAC ) $\perp$ (평면 OBC )이고

두 평면 $\mathrm{OAC}, \mathrm{OBC}$ 의 교선은 직선 OC 이므로 점 B 에서 직선 OC 에 내린 수선의 발은 점 B 에서 평면 OAC 에 내린 수선의 발인 H 이다. 따라서 구 $S$ 를 평면 OBC 로 자른 단면은 다음 그림과 같다.

\begin{center}
\includegraphics[max width=\textwidth]{2024_08_01_b4f29a259b9b99d8349ag-9(5)}
\end{center}

점 H 에서 직선 AC 에 내린 수선의 발을 I 라 하면 $\overline{\mathrm{BH}} \perp$ (평면 OAC )이고 $\overline{\mathrm{HI}} \perp$ (직선 AC )이므로 삼수선의 정리에 의하여

$\overline{\mathrm{BI}} \perp$ (직선 AC )

따라서 두 평면 $\mathrm{OAC}, \mathrm{ABC}$ 가 이루는 예각의 크기를 $\theta$ 라 할 때, $\theta=\angle \mathrm{BIH}$

\begin{center}
\includegraphics[max width=\textwidth]{2024_08_01_b4f29a259b9b99d8349ag-9(4)}
\end{center}

직각삼각형 OBH 에서

$\overline{\mathrm{OH}}=\sqrt{\overline{\mathrm{OB}}^{2}-\overline{\mathrm{BH}}^{2}}$

\[
=\sqrt{5^{2}-4^{2}}
\]

\[
\begin{aligned}
& =3 \\
& =\overline{\mathrm{O}} \\
& =5 \\
& =2
\end{aligned}
\]

$\overline{\mathrm{HC}}=\overline{\mathrm{OC}}-\overline{\mathrm{OH}}$

\[
=5-3
\]

\begin{center}
\includegraphics[max width=\textwidth]{2024_08_01_b4f29a259b9b99d8349ag-9(3)}
\end{center}

삼각형 HIC 는 직각이등변삼각형이므로

$\overline{\mathrm{HI}}=\frac{\overline{\mathrm{HC}}}{\sqrt{2}}=\sqrt{2}$

직각삼각형 BIH 에서

$\tan \theta=\tan (\angle \mathrm{BIH})=\frac{\overline{\mathrm{BH}}}{\overline{\mathrm{HI}}}=\frac{4}{\sqrt{2}}=2 \sqrt{2}$

이때 $\cos \theta=\frac{1}{3}$

따라서 삼각형 OAC 의 평면 ABC 위로의 정사영의 넓이는

(삼각형 OAC 의 넓이) $\times \cos \theta=\frac{25}{2} \times \frac{1}{3}=\frac{25}{6}$

곧, $k=\frac{25}{6}$ 이므로 $6 \times k=25$


\end{document}